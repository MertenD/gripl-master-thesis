\NeedsTeXFormat{LaTeX2e}
\documentclass[a4paper,12pt,
headsepline,           % Linie zw. Kopfzeile und Text
oneside,               % einseitig
pointlessnumbers,      % keine Punkte nach den letzten Ziffern in Überschriften
bibtotoc,              % LV im IV
%DIV=15,               % Satzspiegel auf 15er Raster, schmalere Ränder
BCOR15mm               % Bindekorrektur
%,draft
]{scrbook}
\KOMAoptions{DIV=last} % Neuberechnung Satzspiegel nach Laden von Paket helvet

\pagestyle{headings}
\usepackage{blindtext}

% für Texte in deutscher Sprache
\usepackage[ngerman]{babel}
\usepackage[utf8]{inputenc}
\usepackage[T1]{fontenc}
\usepackage[printonlyused]{acronym}

\newenvironment{abstract}{}{}
\usepackage{abstract}

% table design
\usepackage[table]{xcolor}

\usepackage{booktabs}
\usepackage{threeparttable}
\usepackage{rotating}
\usepackage{tablefootnote}

\usepackage{graphicx}
\usepackage{caption}
\usepackage{subcaption}

\usepackage{adjustbox}

% Helvetica als Standard-Dokumentschrift
\usepackage[scaled]{helvet}
\renewcommand{\familydefault}{\sfdefault}

\usepackage{booktabs,tabularx,hyperref}
\newcolumntype{Y}{>{\raggedright\arraybackslash}X}

% Literaturverzeichnis mit BibLaTeX
\usepackage[babel,german=quotes]{csquotes}
\usepackage[backend=biber]{biblatex}
\bibliography{bibliography}

% Alternative mit Paket-Option backend=biber und \addbibresource
% \usepackage[backend=biber]{biblatex}
% \addbibresource{bibliography.bib}

% Für Tabellen mit fester Gesamtbreite und variabler Spaltenbreite
\usepackage{tabularx} 

% Besondere Schriftauszeichnungen
\usepackage{url}              % \url{http://...} in Schreibmaschinenschrift
\usepackage{color}            % zum Setzen farbigen Textes

\usepackage{amssymb, amsmath} % Pakete für Mathe-Umgebungen und -Symbole

\usepackage{setspace}         % Paket für div. Abstände, z.B. ZA
%\onehalfspacing              % nur dann, wenn gefordert; ist sehr groß!!
\setlength{\parindent}{0pt}   % kein linker Einzug der ersten Absatzzeile
\setlength{\parskip}{1.4ex plus 0.35ex minus 0.3ex} % Absatzabstand, leicht variabel

% Tiefe, bis zu der Überschriften in das Inhaltsverzeichnis kommen
\setcounter{tocdepth}{3}      % ist Standard

% Beispiele für Quellcode
\usepackage{listings}
\lstset{language=Java,
  showstringspaces=false,
  frame=single,
  numbers=left,
  basicstyle=\ttfamily,
  numberstyle=\tiny}

\usepackage{threeparttable}

\usepackage[T1]{fontenc}
\usepackage[scaled]{beramono}
\lstdefinelanguage{Kotlin}{
  morekeywords={data,class,object,val,var,fun,interface,override,package,import,null,true,false},
  sensitive=true,
  morecomment=[l]{//},
  morecomment=[s]{/*}{*/},
  morestring=[b]"
}
\lstset{
  basicstyle=\ttfamily\small,
  columns=fullflexible,
  showstringspaces=false,
  tabsize=2,
  language=Kotlin,
  breaklines=true,
  breakatwhitespace=false,
  columns=fullflexible,
  keepspaces=true,
  postbreak=\mbox{$\hookrightarrow$\space},
  breakindent=1em
}

% hier Namen etc. einsetzen
\newcommand{\fullname}{Merten Dieckmann}
\newcommand{\email}{merten.dieckmann@uni-ulm.de}
\newcommand{\titel}{Identifikation von DSGVO-kritischen Aktivitäten in Business Prozessen mittels Large Language Models}
\newcommand{\jahr}{2025}
\newcommand{\matnr}{1058340}
\newcommand{\gutachterA}{Prof.\,Dr.\, Manfred Reichert}
\newcommand{\gutachterB}{Prof.\,Dr.\, Rüdiger Pryss}
\newcommand{\betreuer}{Magdalena von Schwerin}

% hier die Fakultät auswählen
%\newcommand{\fakultaet}{---  Im Quellcode anpassen nicht vergessen! ---}
\newcommand{\fakultaet}{Ingenieurwissenschaften, Informatik und\\Psychologie}
%\newcommand{\fakultaet}{Mathematik und\\Wirtschafts-\\wissenschaften}
%\newcommand{\fakultaet}{Medizin}
%\newcommand{\fakultaet}{Naturwissenschaften}

% hier das Institut einsetzen
\newcommand{\institut}{Institut für Datenbanken und Informationssysteme (DBIS)}

% Informationen, die LaTeX in die PDF-Datei schreibt
\pdfinfo{
  /Author (\fullname)
  /Title (\titel)
  /Producer     (pdfeTex 3.14159-1.30.6-2.2)
  /Keywords ()
}

\usepackage{hyperref}
\hypersetup{
pdftitle=\titel,
pdfauthor=\fullname,
pdfsubject={Diplomarbeit},
pdfproducer={pdfeTex 3.14159-1.30.6-2.2},
colorlinks=false,
pdfborder=0 0 0	% keine Box um die Links!
}

% Trennungsregeln
\hyphenation{Sil-ben-trenn-ung}

\begin{document}
\frontmatter

% Titelseite
\thispagestyle{empty}
\begin{addmargin*}[4mm]{-10mm}

\hfill
\includegraphics[height=1.8cm]{images/logo_uulm_sw.png}\\[1em]

{\footnotesize
%{\bfseries Universität Ulm} \textbar ~89069 Ulm \textbar ~Germany
\hspace*{115mm}\parbox[t]{35mm}{\bfseries Fakultät für\\
\fakultaet\\
\mdseries \institut}\\[2cm]

\parbox{140mm}{\bfseries \LARGE \titel}\\[2.5em]
{\footnotesize Abschlussarbeit an der Universität Ulm}\\[3em]

{\footnotesize \bfseries Vorgelegt von:}\\
{\footnotesize \fullname\\ \email}\\ \matnr\\[2em]
{\footnotesize \bfseries Gutachter:}\\                     
{\footnotesize \gutachterA\\ \gutachterB}\\[2em]
{\footnotesize \bfseries Betreuer:}\\ 
{\footnotesize \betreuer}\\\\
{\footnotesize \jahr}
}
\end{addmargin*}


% Impressum
\clearpage
\thispagestyle{empty}
{ \small
  \flushleft
  Fassung \today \\\vfill
  \copyright~\jahr~\fullname\\[0.5em]
% Wenn Sie Ihre Arbeit unter einer freien Lizenz bereitstellen möchten, können Sie die nächste Zeile in Ihren Code aufnehmen. Bitte beachten Sie, dass Sie hierfür an allen Inhalten, inklusive enthaltener Abbildungen, die notwendigen Rechte benötigen! Beim Veröffentlichungsexemplar Ihrer Dissertation achten Sie bitte darauf, dass der Lizenztext nicht den Angaben in den Metadaten der genutzten Publikationsplattform widerspricht. Nähere Information zu den Creative Commons Lizenzen erhalten Sie hier: https://creativecommons.org/licenses/
%This work is licensed under the Creative Commons Attribution 4.0 International (CC BY 4.0) License. To view a copy of this license, visit \href{https://creativecommons.org/licenses/by/4.0/}{https://creativecommons.org/licenses/by/4.0/} or send a letter to Creative Commons, 543 Howard Street, 5th Floor, San Francisco, California, 94105, USA. \\
  Satz: PDF-\LaTeXe
}

\begin{abstract}
    In der EU ist die Einhaltung der Datenschutz-Grundverordnung (DSGVO) in Geschäftsprozessen essenziell, da bei Nichteinhaltung hohe Bußgelder drohen können. Jedoch ist eine manuelle Prüfung von Prozessmodellen in der Praxis aufwendig und fehleranfällig. Diese Arbeit untersucht daher, wie Large Language Models (LLMs) DSGVO-kritische Aktivitäten in BPMN-Prozessmodellen automatisiert identifizieren können, da sie großes Potenzial im Umgang mit komplexen Texten und Zusammenhängen zeigen. Hierzu wird (1) eine Klassifizierungspipeline mit Zero-Shot-Prompting und strikt strukturiertem JSON-Output, ergänzt um \emph{id}-Validierung/-Vervollständigung und einem automatischen Retry-Mechanismus, (2) ein Labeling-Tool für gelabelte BPMN-Testfälle sowie (3) ein Evaluationsframework mit deklarativer Konfiguration, standardisierter HTTP-Schnittstelle und Frontend für reproduzierbare, vergleichbare Experimente entwickelt. Diese Infrastruktur ermöglicht belastbare Modellvergleiche über mehrere Domänen und Sprachen hinweg.

    In einer empirischen Studie werden 13 LLMs (Europäische vs. Internationale, Offene vs. Proprietäre, Große vs. Kleine) auf 25 Testfälle über fünf Wiederholungen evaluiert. Daraufhin werden die Mittelwerte und Standardabweichungen für Precision, Recall, F1-Score und Accuracy berechnet. Neun von dreizehn Modellen erreichen einen F1-Score von $\ge 0{,}80$. Spitzenreiter sind \texttt{Qwen3-235B-A22B-Thinking-2507} (F1-Score $=0{,}874$, Recall $=0{,}932$), \texttt{GPT-OSS-20B} (F1-Score $=0{,}866$, Recall $=0{,}918$) und \texttt{DeepSeek-R1-Distill-Qwen-14B} (F1-Score $=0{,}848$, Precision $=0{,}829$). Unter den EU-Modellen stechen \texttt{Mistral Medium 3.1} und \texttt{Mistral-\linebreak~Large-Instruct-2411} mit F1-Scores von $0{,}843$ bzw. $0{,}823$ heraus.

    Die Ergebnisse zeigen zudem unterschiedliche Trade-offs: \texttt{GPT-4o} erzielt die höchste Precision ($0{,}892$), verfehlt jedoch mit einem Recall von $0{,}762$ die Mindestanforderung für ein Recall-orientiertes Screening. \texttt{Gemma-3-27B-it} erreicht umgekehrt einen sehr hohen Recall ($0{,}916$) bei niedriger Precision ($0{,}687$). Insgesamt ist die Varianz über Seeds für die meisten Modelle gering ($\text{SD}_{\text{F1-Score}}\le 0{,}02$).

    Fehleranalysen zeigen vor allem False Positives bei fehlendem Kontext (z.\,B. nicht explizit modellierte Anonymisierung von Daten) und False Negatives, wenn personenbezogene Datenflüsse über mehrere Schritte nicht sicher erkannt werden.

    Insgesamt eignen sich aktuelle LLMs gut für ein automatisiertes, Recall-orientiertes Vorscreening von BPMN-Prozessen, jedoch bleibt eine nachgelagerte fachliche Prüfung erforderlich.
\end{abstract}

% ab hier Zeilenabstand etwas größer 
\setstretch{1.2}

\tableofcontents

\cleardoublepage
\addchap{Abkürzungen}
\begin{acronym}[SVM]
    \acro{DSGVO}{Datenschutz-Grundverordnung}
    \acro{LLM}{Large Language Model}
    \acrodefplural{LLM}[LLMs]{Large Language Models}
    \acro{BPM}{Business Process Management}
    \acro{BPMN}{Business Process Model and Notation}
    \acro{EU}{Europäische Union}
\end{acronym}

\mainmatter

\chapter{Einleitung}\label{ch:einleitung}

\section{Motivation}\label{sec:motivation}
\begin{itemize}
    \item 1-2 Sätze auch noch DSGVO irgendwo hier erwähnen
    \item Businessprozesse sind überall.
    \item Datenschutz relevant in Europa.
    \item Datenschutzprüfungen in Prozessen sind kosten- und zeitintensiv.
    \item Fehlerhafte Unterentdeckung ist ggf. kritisch (False Negatives).
    \item LLMs versprechen schnelles Screening (vor allem europäische Open-Source-Modelle sind interessant); dafür werden jedoch noch belastbare Evidenz und reproduzierbare Vergleiche benötigt.
\end{itemize}
\section{Problemstellung}\label{sec:problemstellung}

Trotz der genannten Potenziale fehlt es bisher an standardisierten, reproduzierbaren Vergleichen verschiedener Modelle für die konkrete Aufgabe Aktivitäten in Geschäftsprozessen nach \enquote{kritisch} und \enquote{unkritisch} zu klassifizieren. Erste Ansätze, wie z.B. der von Nake et al. \cite{nake2023towards}, zeigen dass ML Ansätze grundsätzlich in der Lage sind \ac{DSGVO}-kritische Aktivitäten in textuellen Prozessbeschreibungen zu erkennen; dennoch existieren keine einheitlichen Benchmarks, die einen systematischen vergleich unterschiedlicher \acp{LLM} erlauben.

Auch von Schwerin et al. \cite{schwerin2024systematic} heben hervor, dass trotz großer Fortschritte im Einsatz von \acp{LLM} für juristische Aufgaben bislang erhebliche Lücken in der Evaluation für compliance-spezifische Anwendungen bestehen und geeignete \ac{DSGVO}-spezifische Benchmarks fehlen. Somit mangelt es derzeit an einer belastbaren empirischen Grundlage, um Modelle zuverlässig und vergleichbar zu bewerten.

Besonders interessant ist die Frage, wie sich Open-Source-Modelle - insbesondere mit Ursprung aus der \ac{EU} - im Vergleich zu internationalen außerhalb der \ac{EU} entwickelten Modellen schlagen und welche Trade-offs dabei entstehen \cite{schwerin2024systematic}. Diese Perspektive ist nicht nur aus Leistungs-, sondern auch aus Transparenz- und Regulierungsgründen relevant.

Eine zusätzliche Herausforderung ergibt sich aus der Natur von \ac{BPMN}-Modellen: Typischerweise konzentrieren sie sich auf den Kontrollfluss und vernachlässigen die Datenebene. Datenobjekte werden oftmals gar nicht explizit modelliert oder nur implizit in den Aktivitäten referenziert. Dadurch ist die Datennutzung von Aktivitäten nicht direkt erkennbar und muss aus textuellen Beschreibungen und dem Kontext erschlossen werden \cite{schneid2021uncovering}. Das erschwert die automatische Identifikation von \ac{DSGVO}-kritischen Aktivitäten, da Algorithmen personenbezogene Datenflüsse zunächst indirekt und über den Kontext ableiten müssen.
\section{Zielsetzung und Beiträge}\label{sec:zielsetzung-und-beitrage}

Ziel der Arbeit ist es, einen methodischen Beitrag zur automatisierten Identifikation von \ac{DSGVO}-kritischen Aktivitäten in Geschäftsprozessen zu leisten. Hierfür werden folgende Beiträge angestrebt:

\begin{itemize}
    \item Entwicklung einer Klassifizierungspipeline für Geschäftsprozesse, die Aktivitäten binär in datenschutzkritisch oder unkritisch einordnet.
    \item Konzeption und Umsetzung eines Evaluationsframeworks, das reproduzierbare Vergleiche verschiedener \acp{LLM} und Algorithmen über eine einheitliche Schnittstelle ermöglicht.
    \item Entwicklung einer Labeling-Software zur Erstellung und Annotation von Datensätzen für das Evaluationsframework.
    \item Aufbau eines repräsentativen Datensatzes aus gelabelten \ac{BPMN}-Prozessen, inklusive klar definierter Labeling-Kriterien.
    \item Bereitstellung überprüfbarer empirischer Befunde, inklusive Code, Konfigurationen der Experimente und Seeds, um Nachvollziehbarkeit und Reproduzierbarkeit zu gewährleisten.
\end{itemize}

Auf dieser Grundlage ergibt sich die zentrale Forschungsfrage dieser Arbeit:

\begin{description}
    \item[\textbf{FF1}] Wie zuverlässig identifizieren \acp{LLM} \ac{DSGVO}-kritische Aktivitäten in \ac{BPMN}-Prozessmodellen?
\end{description}

Um diese Frage differenziert beantworten zu können, werden außerdem folgende Unterfragen betrachtet:

\begin{description}
    \item[\textbf{UF1}] Wie gut schneiden europäische Modelle im Vergleich zu internationalen Modellen ab?
    \item[\textbf{UF2}] Wie unterscheiden sich große und kleine Modelle in ihrer Leistungsfähigkeit?
    \item[\textbf{UF3}] Welche Open-Source-Modelle (insbesondere aus der \ac{EU}) erzielen die besten Ergebnisse?
    \item[\textbf{UF4}] Wie gut schneiden Open-Source-Modelle gegenüber kommerziellen Modellen wie \texttt{GPT-4o} ab?
\end{description}

Für ein initiales Screening reicht, wie in \cite{nake2023towards}, eine binäre Klassifikation (kritisch vs. unkritisch). Eine tiefergehende rechtliche Prüfung kann in einem nachfolgenden Schritt durchgeführt werden und ist nicht Bestandteil dieser Arbeit.
\section{Forschungsfrage und Unterfragen}\label{sec:forschungsfrage-und-unterfragen}

Die zentrale Forschungsfrage dieser Arbeit lautet:

\begin{description}
    \item[\textbf{FF1}] Wie zuverlässig identifizieren \acp{LLM} \ac{DSGVO}-kritische Aktivitäten in \ac{BPMN}-Prozessmodellen?
\end{description}

Um diese Frage differenziert beantworten zu können werden außerdem folgende Unterfragen betrachtet:

\begin{description}
    \item[\textbf{UF1}] Wie gut schneiden europäische Open-Source-Modelle im Vergleich zu internationalen Modellen ab?
    \item[\textbf{UF2}] Wie unterscheiden sich große und kleine Modelle in ihrer Leistungsfähigkeit?
    \item[\textbf{UF3}] Welche Open-Source-Modelle (insbesondere aus der EU) erzielen die besten Ergebnisse?
    \item[\textbf{UF4}] Wie gut schneiden Open-Source-Modelle gegenüber kommerziellen Modellen wie GPT-4o ab?
\end{description}

Für ein initiales Screening reicht, wie in \cite{nake2023towards}, eine binäre Klassifikation (kritisch vs. unkritisch). Eine tiefergehende rechtliche Prüfung kann in einem nachfolgendem Schritt durchgeführt werden und ist nicht Bestandteil dieser Arbeit.
\section{Aufbau der Arbeit}\label{sec:aufbau-der-arbeit}
\begin{itemize}
    \item Kurzer Überblick über jedes Kapitel (vom Kontext über Algorithmus, Framework (Labeling \& Evaluierung) hin zu Durchführung, Ergebnisse und abschließend Diskussion).
\end{itemize}
\chapter{Hintergrund und verwandte Arbeiten}\label{ch:hintergrund-und-verwandte-arbeiten}

\section{Datenschutzgrundverordnung (DSGVO)}\label{sec:dsgvo}

\begin{itemize}
    \item Definitionen: Personenbezogene Daten, Verarbeitung, typische Auslöser für ``kritisch'' (Erhebung, Nutzung, Speichern, Übermitteln, Löschen, \ldots)
    \item Abgrenzung, dass es sich hier bei den Klassifizierungen durch die Algorithmen nur um ein Risiko-Screening handelt und nicht um eine Rechtsberatung (Sollte ich vielleicht erwähnen, damit auch die Ergebnisse am Ende besser eingeordnet werden können)
\end{itemize}
\section{Business Process Model and Notation (BPMN)}\label{sec:bpmn}

\ac{BPMN} ist ein Standard zur Modellierung von Geschäftsprozessen. Die Notation wurde entwickelt, um eine einheitliche Notation bereitzustellen, die sowohl von Geschäftsanalysten als auch von technischen Entwicklern verstanden wird. \ac{BPMN}-Modelle bestehen aus verschiedenen Elementen wie Aktivitäten, Ereignissen, Gateways und Verbindungen, die zusammen den Ablauf eines Geschäftsprozesses darstellen \cite{omgbpmn}.

\subsection*{Relevante BPMN-Elemente}
Für die Identifikation von \ac{DSGVO}-kritischen Aktivitäten sind insbesondere folgende Elemente relevant, da sie Hinweise auf den Umgang mit (personenbezogenen) Daten geben:

\begin{itemize}
    \item \textbf{Aktivitäten}: bilden die auszuführenden Arbeitsschritte eines Prozesses ab. Sie können Ein- und Ausgaben sowie Datenabhängigkeiten definieren \cite{omgbpmn}. Durch ihren Namen oder Kontext können Rückschlüsse auf die Verarbeitung personenbezogener Daten gezogen werden.
    \item \textbf{Sequenzflüsse}: verbinden Aktivitäten, Ereignisse und Gateways und zeigen die Reihenfolge der Ausführung im Prozess an \cite{omgbpmn}. Mit ihrer Hilfe kann eine einzelne Aktivität im Kontext des gesamten Prozesses betrachtet werden, indem der Pfad zu und von der Aktivität verfolgt wird.
    \item \textbf{Datenobjekte und Datenspeicher}: repräsentieren flüchtige oder persistente Daten, die im Prozess von z.B. Aktivitäten genutzt oder geschrieben werden können \cite{omgbpmn}. Sie können auch personenbezogene Daten enthalten.
    \item \textbf{Datenassoziationen}: Eingangs- und Ausgangsassoziationen verbinden Aktivitäten mit Datenobjekten und Datenspeichern und zeigen so Ein- und Ausgaben explizit an \cite{omgbpmn}. Sie sind ein wichtiges Signal für die Verarbeitung personenbezogener Daten, da sie den direkten Bezug einer Aktivität zu bestimmten Daten verdeutlichen (z.B. Lesezugriff auf eine Kundendatenbank).
    \item \textbf{Pools und Lanes}: Pools modellieren Organisationseinheiten oder Prozessbeteiligte, während Lanes Verantwortlichkeiten innerhalb eines Pools darstellen. Innerhalb eines Pools befinden sich die Aktivitäten und anderen Elemente des Prozesses \cite{omgbpmn}. Die Rollen und Verantwortlichkeiten, die durch Pools und Lanes dargestellt werden, können für die Bewertung der Datenverarbeitung relevant sein.
    \item \textbf{Nachrichtenflüsse}: stellen den Austausch von Nachrichten zwischen verschiedenen Pools dar \cite{omgbpmn}. Sie können auf eine Übermittlung personenbezogener Daten an Dritte hinweisen (z.B. Transfer von Kundendaten an einen externen Dienstleister).
    \item \textbf{Textannotationen und Assoziationen}: dienen dazu, zusätzliche Informationen zu Prozessmodellen hinzuzufügen, die nicht durch die standardmäßigen BPMN-Elemente abgedeckt sind \cite{omgbpmn}. Sie können genutzt werden, um die Art der Datenverarbeitung zu präzisieren (z.B. „enthält E-Mail-Adresse“).
\end{itemize}

\subsection*{\ac{BPMN}-XML}

\ac{BPMN}-Modelle werden in einer XML-Serialisierung gespeichert (\ac{BPMN} 2.0 XML) \cite{omgbpmn}. Diese Darstellung enthält alle relevanten Strukturinformationen (Elementtypen, Namen, Beziehungen, Zuordnungen, Positionen der Elemente) und wird von vielen Prozess-Engines und Modellierungswerkzeugen wie Camunda \cite{camunda} und BPMN.io \cite{bpmnio} unterstützt. Für diese Arbeit dient \ac{BPMN}-XML als Eingabeformat der Klassifizierungspipeline (siehe Kapitel \ref{ch:klassifizierungsalgorithmus-(design-und-implementierung)}).

Im Metamodell von \ac{BPMN} erben fast alle Elemente von \texttt{BaseElement} und damit ein \texttt{id}-Attribut. Dieses \texttt{id} dient der eindeutigen Referenzierung und ist \emph{erforderlich} \cite{omgbpmn}. Diese ID ist für die Klassifizierungspipeline wichtig, da sie eine stabile Referenzierung der Aktivitäten und anderer Elemente ermöglicht. Dies ist insbesondere dann relevant, wenn die Ergebnisse der Klassifizierung auf die ursprünglichen Prozessmodelle zurückgeführt werden müssen.

\subsection*{Beispiel einer Datenassoziation als Datenschutzsignal}

\begin{figure}
    \centering
    \begin{subfigure}{.5\textwidth}
        \centering
        \includegraphics[width=.7\linewidth]{images/process-models/data-association-example-uncritical}
        \caption{Ohne Datenassoziation}
        \label{fig:without-data-association}
    \end{subfigure}%
    \begin{subfigure}{.5\textwidth}
        \centering
        \includegraphics[width=.7\linewidth]{images/process-models/data-association-example-critical}
        \caption{Mit Datenassoziation}
        \label{fig:with-data-association}
    \end{subfigure}
    \caption{ Beispiel einer Datenassoziation als Datenschutzsignal.}
    \label{fig:data-association-gdpr-example}
\end{figure}

Abbildung \ref{fig:data-association-gdpr-example} zeigt ein einfaches Beispiel, wie eine Datenassoziation die \ac{DSGVO}-Relevanz einer Aktivität verdeutlichen kann. In Abbildung \ref{fig:without-data-association} ist die Aktivität \enquote{Daten prüfen} ohne Datenassoziation dargestellt, was wenig über die Art der verarbeiteten Daten aussagt. In Abbildung \ref{fig:with-data-association} hingegen zeigt die eingehende Datenassoziation von einem Datenspeicher \enquote{Kunden-DB}, dass die Aktivität personenbezogene Daten verarbeitet. Dies macht die Aktivität als potenziell datenschutzkritisch erkennbar. Dieses Beispiel unterstreicht die Notwendigkeit den gesamten Kontext einer Aktivität zu betrachten, um fundierte Rückschlüsse auf die Verarbeitung personenbezogener Daten ziehen zu können.
\section{LLMs}\label{sec:llms}

\begin{itemize}
    \item Prompting (System Message, User Prompt)
    \item Zero-/Few-Shot
    \item JSON-konforme Ausgaben (Schema-Enforcing)
    \item typische Fehlerbilder (Halluzination, ungültiges JSON) + Gegenmaßnahmen (Retry/Repair)
    \item Fine-Tuning (?) Auch wenn ich es nicht nutze?
\end{itemize}
\section{Verwandte Arbeiten}\label{sec:verwandte-arbeiten}

Dieses Kapitel bündelt Arbeiten zur automatisierten \emph{Klassifikation datenschutzkritischer Aktivitäten} in Geschäftsprozessen und zur \emph{Nutzung von \acp{LLM}} in Datenschutzaufgaben und Tätigkeiten im \ac{BPM}. Im Fokus stehen (i) frühe Klassifikations- und Modellierungsansätze, (ii) \ac{LLM}-basierte Analyse von Richtlinientexten bis hin zu strukturierter Extraktion, (iii) Qualitätssicherung, Prompting und Reproduzierbarkeit sowie (iv) der Einsatz von \acp{LLM} im \ac{BPM}-Lebenszyklus. Abschließend werden Forschungslücken abgeleitet.

\subsection*{Frühe Ansätze: Klassifikation und modellbasierte Kennzeichnung}

Die Identifikation von Prozessschritten mit Verarbeitung personenbezogener Daten ist Voraussetzung wirksamer \ac{DSGVO}-Konformität, da nur so technische und organisatorische Maßnahmen gemäß Art.~32 Abs.~1 (Vertraulichkeit, Integrität, Verfügbarkeit, Belastbarkeit) zielgerichtet festgelegt werden können \cite{GDPR2016}. Nake et al.\ \cite{nake2023towards} beschreiben einen ersten automatisierten Ansatz: Mit einem überwachten Verfahren (Lernen aus gelabelten Beispielen) klassifizieren sie Aktivitäten in \emph{textuellen} Prozessbeschreibungen als \ac{DSGVO}-kritisch vs.\ unkritisch. Der Datensatz umfasst 37 Prozesse mit 509 Aktivitäten. In der stärksten Konfiguration werden ein F1-Score von $0{,}81$ und ein Recall von $0{,}83$ erreicht. Die Generalisierbarkeit bleibt aufgrund des kleinen, nicht repräsentativen Datensatzes begrenzt. Fehler entstehen u.\,a.\ durch zu wenige Trainingsbeispiele für bestimmte Merkmalswerte. Der Ansatz ist daher als Assistenz für Datenschutzbeauftragte zu verstehen, nicht als vollständige Automatisierung.

Komplementär markieren Capodieci et al.\ \cite{Capodieci2023BPMNEnabledDP} \ac{BPMN}-Elemente mit \ac{DSGVO}-\linebreak~Metadaten via \emph{Tagged Values} (\texttt{GDPR:legalbasis}, \texttt{GDPR:Duration},\linebreak~\texttt{GDPR:risklevel}, \texttt{GDPR:ispersonaldataprocessing}/\texttt{GDPR:personaldata}),\linebreak~sodass Datenschutzaspekte bereits in der Pre-Implementation-Phase prüfbar werden. In eine ähnliche Richtung zielt die designorientierte Arbeit von Agostinelli et al.\ \cite{agostinelli2019achievingGDPRComliance}, die \ac{DSGVO}-Anforderungen als wiederverwendbare Muster (\enquote{Data Breach}, \enquote{Consent to Use the Data}, \enquote{Right to Access/Rectify}, \enquote{Portability}, \enquote{Right to be Forgotten}) für eine transparente Einbettung in \ac{BPMN} formalisiert.

\subsection*{\acp{LLM} für Policy-Analyse}

Ciaramella et al.\ \cite{ciaramella2022leveraging} nutzen BERT, RoBERTa und DistilBERT, um Sätze aus Online-Datenschutzerklärungen gezielt im Hinblick auf die Informationspflicht aus Art.~13 (2)(b) \ac{DSGVO} (Hinweis auf Berichtigung/Löschung) zu klassifizieren. Die Ergebnisse sind \emph{moderat} und zeigen, dass innerhalb einer Erklärung konforme und nicht konforme Passagen koexistieren. Daher schließen sie daraus, dass Konformität kein binäres Gesamteurteil nur auf Textebene ist.

Neuere Arbeiten gehen darüber hinaus: Hooda et al.\ \cite{hooda2024policylr} führen mit \emph{PolicyLR} eine logikbasierte Form für Richtlinien ein und übersetzen Texte mit einem zweistufigen \ac{LLM}-Compiler (Übersetzen $\to$ Entailment-Prüfung) in atomare Formeln. Auf ToS;DR (annotierte Nutzungsbedingungen) erzielen Open-Source-\acp{LLM} wie \texttt{gemma2-27b} eine Precision von $0{,}91$ und $0{,}88$ Recall. Damit werden Compliance-Checks, Konsistenzprüfungen und Policy-Vergleiche möglich. Rodriguez et al.\ \cite{rodriguez2024largelanguagemodels} optimieren Prompts, Parameter und die Kontextaufteilung (Chunking) für die feingranulare Extraktion von Erhebungs- und Weitergabepraktiken mit \texttt{GPT-4~Turbo}. Auf MAPP erreichen sie $0{,}935$ F1-Score, auf OPP-115 $0{,}93$ F1-Score, $0{,}949$ Precision, $0{,}912$ Recall und $0{,}904$ Accuracy, während \texttt{Llama-2-70B-Chat} mit $0{,}882$ F1 leicht darunter liegt.

\subsection*{Qualitätssicherung, Prompting und Reproduzierbarkeit}

Zur Reduktion von Halluzinationen koppeln Silva et al.\ \cite{silva2024entailment} einen erklärenden \ac{LLM}-Klassifikator mit einem Entailment-Prüfer, der nur Entscheidungen passieren lässt, die sich aus dem Text folgern lassen. Auf OPP-115 steigt der Macro-F1-Score auf $0{,}63$ (plus $11{,}2$\%). Die zusätzliche Prüfstufe erhöht die Präzision von $0{,}38$ auf $0{,}61$, senkt aber den Recall von $0{,}85$ auf $0{,}61$. Für Screening-Aufgaben mit nachgelagerter menschlicher Prüfung eignet sich der Entailment-Schritt damit eher als optionaler High-Precision-Filter. Prompting wirkt als weiterer Qualitätshebel: Von Zero-/Few-Shot-Grundlagen \cite{brown2020fewshot,liu2023prompting} über RoPA-Generierung \cite{pragyan2024toward} bis zu domänenspezifischen Kategorien \cite{schwerin2024systematic} zeigt sich, dass Beispielanzahl, Kontextaufbereitung und Modellwahl entscheidend sind. Reimers und Gurevych \cite{reimers2017reporting} empfehlen zudem, aufgrund seed-bedingter Varianz Score-Verteilungen statt Einzelwerte zu berichten.

\subsection*{\acp{LLM} im BPM-Lebenszyklus}

Über Richtlinientexte hinaus skizzieren Vidgof et al.\ \cite{vidgof2023largelanguagemodelsbusiness} zentrale Forschungsrichtungen für \ac{LLM}-gestütztes \ac{BPM}, darunter Best Practices, \ac{BPM}-spezifische Datensätze und Leitlinien zu Prompting und Modellauswahl. Kourani et al.\ \cite{kourani2025evaluating} vergleichen in einem Benchmark mit 20 Prozessen 16 \acp{LLM} zur Transformation von Prozessbeschreibungen in ausführbare Modelle. \texttt{Claude-3.5-Sonnet} erzielt die höchste durchschnittliche Qualitätsbewertung, während z.\,B.\ \texttt{Mixtral-8x22B} zurückfällt. Es wird ein positiver Zusammenhang zwischen Fehlerbehandlung und Modellqualität wird beobachtet und dass durch Output-Optimierungstechniken schwächere Modelle spürbar verbessert werden können. Für dialogorientierte Unterstützung kombinieren Bernardi et al.\ \cite{bernardi2024conversing} \ac{RAG} mit feingetunten \texttt{LLaMA-2}-Modellen (BPLLM) und erreichen bei ausreichender Kontextabdeckung präzise Antworten zu Aktivitäten und Sequenzflüssen.

\subsection*{Forschungslücken}

Aus der Literatur ergeben sich mehrere offene Fragen, die die vorliegende Arbeit adressiert:

\begin{enumerate}
    \item \textbf{Granularität und Domänenfokus.} Viele Studien fokussieren einzelne Artikel (z.\,B.\ Art.~13) oder allgemeine Privacy-Tasks. Eine systematische Klassifikation \emph{kompletter} Geschäftsprozesse nach datenschutzrelevanten Aktivitäten ist selten. Zudem sind Datensätze klein und wenig repräsentativ \cite{nake2023towards}.
    \item \textbf{\ac{LLM}-Anwendung in Geschäftsprozessen.} Während es Benchmarks für Prozessmodellierung gibt, fehlen reproduzierbare Benchmarks speziell für die Klassifikation datenschutzkritischer Prozessschritte. Positionsarbeiten wie\linebreak~Vidgof et al.\ \cite{vidgof2023largelanguagemodelsbusiness} fordern \ac{BPM}-spezifische Datensätze und Modelle. Öffentlich verfügbare, auf den europäischen Rechtsraum zugeschnittene Benchmarks sind rar.
    \item \textbf{Erklärbarkeit und Halluzinationen.} \acp{LLM} erzeugen teils überzeugende, aber unzutreffende Ausgaben. Ansätze wie Silva et al.\ \cite{silva2024entailment} und Hooda et al.\ \cite{hooda2024policylr} zeigen, dass Schlussfolgerungsprüfer oder formale Repräsentationen nötig sind, um Halluzinationen zu reduzieren.
    \item \textbf{Datenschutz- und Sicherheitsbedenken.} Studien nutzen häufig geschlossene Modelle wie \texttt{GPT-4}, deren Einsatz aufgrund möglicher Datenübermittlungen in die USA datenschutzrechtlich problematisch sein kann. Offene Modelle wie \texttt{Qwen2-7B} liefern vergleichbare Ergebnisse \cite{schwerin2024systematic} und sind für \ac{EU}-Organisationen potenziell vorteilhaft.
    \item \textbf{Prompt-Engineering-Leitlinien.} Obwohl mehrere Arbeiten den maßgeblichen Einfluss von Prompt-Gestaltung (z.\,B.\ Beispielanzahl, Kontexttrennung) belegen \cite{pragyan2024toward,liu2023prompting}, fehlen breit akzeptierte Leitfäden speziell für Datenschutz- und \ac{BPM}-Kontexte.
\end{enumerate}

Diese Lücken unterstreichen den Bedarf an umfassenden, reproduzierbaren Benchmarks sowie an robusten Methoden zur Klassifikation datenschutzkritischer Aktivitäten in Geschäftsprozessen unter Berücksichtigung der europäischen \ac{DSGVO}. Das folgende Kapitel präzisiert die Problemdefinition und die Qualitätsziele der vorliegenden Arbeit.


\chapter{Problemdefinition und Zielkriterien}\label{ch:problemdefinition-und-zielkriterien}

Dieses Kapitel präzisiert die Aufgabe der Arbeit, die Qualitätsziele der Klassifikation und steckt den fachlichen Geltungsbereich ab. Damit schafft es die Grundlage für die in Kapitel~\ref{ch:klassifizierungsalgorithmus-(design-und-implementierung)} beschriebene Klassifizierungspipeline sowie für die späteren Experimente und deren Auswertung.

\section{Aufgabenstellung}\label{sec:aufgabenstellung}

Ziel der Arbeit ist eine \emph{binäre Klassifikation} auf Ebene einzelner \ac{BPMN}-Aktivitäten: Für jede Aktivität eines Eingabemodells im \ac{BPMN}-XML-Format (Version 2.0.2) \cite{omgbpmn} soll entschieden werden, ob sie \emph{kritisch} im Sinne des Datenschutzrechts ist oder nicht.

\begin{itemize}
    \item \textbf{Eingabe} ist ein valides \ac{BPMN}-XML mit stabilen \texttt{id}-Attributen je Aktivität \cite{omgbpmn}.
    \item \textbf{Ausgabe} ist eine Menge von Aktivitäts-\texttt{id}s, die als \emph{kritisch} klassifiziert wurden. Optional wird zusätzlich eine natürlichsprachige Begründung für einzelne Entscheidungen ausgegeben. Im Fall der Klassifizierungspipeline dieser Arbeit werden die Begründungen vom \ac{LLM} generiert. Diese dienen ausschließlich der Nachvollziehbarkeit der gewählten Klassifizierungen, werden allerdings nicht in der Evaluation berücksichtigt.
\end{itemize}

Zur Veranschaulichung zeigt Listing~\ref{lst:running-example-bpmn} einen vereinfachten  Auszug aus dem \ac{BPMN}-XML des laufenden Beispiels. Die erwartete Ausgabe ist \{\texttt{Activity\_generate}, \texttt{Activity\_send}\}.

\begin{lstlisting}[caption={BPMN-XML-Auszug des laufenden Beispiels}, label={lst:running-example-bpmn}]
<task id="Activity_generate"  name="Generate tracking id"/>
<task id="Activity_template"  name="Load notification template"/>
<task id="Activity_send"      name="Send status update"/>
\end{lstlisting}

\subsection*{Begriffsbestimmung \enquote{kritisch}}

Eine Aktivität gilt in dieser Arbeit als \emph{kritisch}, wenn sie \emph{personenbezogene Daten} verarbeitet. Personenbezogene Daten sind, nach Art.~4~Abs.~1 \ac{DSGVO} \cite{GDPR2016}, alle Informationen, die sich auf eine identifizierte oder identifizierbare natürliche Person beziehen. Gemäß Art.~4~Abs.~2 \ac{DSGVO} \cite{GDPR2016} umfasst Verarbeitung jede mit personenbezogenen Daten vorgenommene Operation, wie z.\,B. Erheben, Speichern, Abrufen, Verwenden, Übermitteln und Löschen. Dies schließt auch die \emph{Nutzung bereits vorhandener Daten} (z.\,B. Lesen/Abgleichen) ein. Am laufenden Beispiel bedeutet dies konkret: \texttt{Activity\_send} ist kritisch, da beim Versand typischerweise personenbezogene Daten (z.\,B. E-Mail-Adresse) verarbeitet werden. \texttt{Activity\_generate} ist kritisch, weil die Tracking-\texttt{id} im Kontext des Kundenkontakts einer Person zugeordnet werden kann. \texttt{Activity\_template} ist im Grundfall nicht kritisch, sofern lediglich eine generische Vorlage geladen wird und keine personenbezogenen Daten einfließen.

Diese Aufgabenstellung reiht sich in Arbeiten zur Kennzeichnung kritischer/unkritischer Tätigkeiten in Prozessartefakten ein und bildet die Referenz für die Qualitätsziele im nächsten Abschnitt. \cite{nake2023towards}
\section{Qualitätsziele}\label{sec:qualitatsziele}

// TODO Hier prüfen, ob ich das so schon hier sagen kann oder das erst bei der Diskussion thematisiert werden soll, dass mir der hohe Recall eher positiv aufgefallen ist, weil es des öfteren plausible Erklärungen dazu gab

\begin{itemize}
    \item Hauptziel ist ein hoher Recall mit minimalen False Negatives, bei noch akzeptabler Precision und somit auch überschaubaren False Positives
\end{itemize}

// TODO Aktuell unterstützt meine Evaluierung noch keine Seeds, aber das sollte ich noch umsetzen können und dann Temperature bei den LLMs auf 0 setzen

\begin{itemize}
    \item Sekundäres Ziel ist Stabilität über mehrere Seeds hinweg
    \item Hier könnte ich noch bestimmte Wertebereiche vorschlagen, sie sinnvoll wären wie Recall $\geq 80\%$, False Positives $\leq 1{,}5/$Prozess. Unbedingt dafür noch in anderen Papern nach guten Werten schauen, wenn ich das machen sollte
\end{itemize}
\section{Scope und Annahmen}\label{sec:scope-und-annahmen}

Dieser Abschnitt definiert Geltungsbereich, Annahmen und Risiken des Ansatzes. Dadurch wird eine klare Einordnung der Ergebnisse und ihrer Reproduzierbarkeit ermöglicht.

\subsection*{Geltungsbereich}

Die folgenden Punkte definieren den Geltungsbereich der Arbeit:

\begin{itemize}
    \item Klassifiziert werden ausschließlich Aktivitäten. Dafür wird sinvoller Kontext (z.\,B. Prozessname, Pool/Lane, Datenobjekte) berücksichtigt.
    \item Labels und Artefakte liegen in Deutsch und Englisch vor.
    \item Es handelt sich um ein \emph{Screening}, nicht um eine Rechtsprüfung. Kritisch klassifizierte Aktivitäten sind anschließend juristisch zu prüfen.
\end{itemize}

\subsection*{Annahmen und Risiken}

Die folgenden Annahmen und potenziellen Risiken sind für die Interpretation der Ergebnisse relevant:

\begin{itemize}
    \item Bei fehlenden Datenobjekten oder mehrdeutigen Labels kann sich die Einschätzung verschlechtern. Das ist ein bekanntes Problem in ähnlichen Studien \cite{nake2023towards}.
    \item Optional generierte \ac{LLM}-Begründungen sind als \emph{Hilfetexte} zu verstehen, um die Entscheidung des \acp{LLM} besser einordnen zu können, bilden aber nicht zwingend die tatsächlichen Entscheidungsgründe des Modells ab.
    \item Ungültiges \ac{BPMN}-XML oder Laufzeitfehler werden als \enquote{technischer Fehler} erfasst und nicht in die Metrikzählung eingerechnet. Sie werden separat berichtet.
\end{itemize}

Dieses Kapitel definierte die binäre Klassifikation von \ac{BPMN}-Aktivitäten als kritisch/unkritisch mit Fokus auf maximalen Recall bei akzeptabler Precision als Aufgabe. Es legte Qualitätsziele, Metriken, Geltungsbereich und Annahmen fest. Darauf aufbauend geht es im nächsten Kapitel um das Design und die Implementierung der Klassifizierungspipeline, die diese Anforderungen umsetzt.

\chapter{Klassifizierungsalgorithmus (Design und Implementierung)}\label{ch:klassifizierungsalgorithmus-(design-und-implementierung)}

\begin{itemize}
    \item Im gesamten Kapitel werden die genutzten Technologien an geeigneter Stelle aufgezeigt und beschrieben (Camunda-BPMN, LangChain4j, Spring Boot, Kotlin, \ldots)
    \begin{itemize}
        \item TODO: Ggf. extra Unterkapitel für die genutzten Technologien oder doch immer da erwähnen wo es gerade passt
        \item Aufzählungen für die Technologien nutzen
    \end{itemize}
\end{itemize}

\section{Ziel und Annahmen}\label{sec:ziel-und-annahmen}

\begin{itemize}
    \item Eingabe ist BPMN-XML, Ausgabe sind die IDs der kritischen Aktivitäten mit ggf. Erklärung
    \item Getestet wird mit BPMN-Modellen aus Camunda und bpmn.io
    \item Der Algorithmus klassifiziert Aktivitäten in BPMN-Prozessen binär nach ``kritisch'' oder ``nicht kritisch''
    \item Ausgegeben werden ausschließlich die IDs der kritischen Aktivitäten ggf. mit Erklärung, warum sie als kritisch klassifiziert wurde
    \item Das Ziel ist ein robustes und reproduzierbares Verhalten über unterschiedliche Modelle
    \begin{itemize}
        \item Granularität der Prozesse (Bspw. $>$5 Aktivitäten, 20+ Aktivitäten)
        \item Gateways, Datenobjekte, Datenbanken
        \item Mehrere Pools/Lanes, Message Flows
        \item Evtl. (?) verschiedene Sprachen, Text-Annotationen
    \end{itemize}
\end{itemize}
\section{BPMN Preprocessing}\label{sec:bpmn-preprocessing}

Ziel der Vorverarbeitung (Preprocessing) ist es, für jedes Flow-Element einen \emph{strukturierten Kontext} zu erzeugen. Dieser Kontext umfasst die eigenen Attribute, wie \texttt{id}, \texttt{name} und \texttt{documentation}, sowie die Beziehungen zu anderen Elementen im \ac{BPMN}-Diagramm. Dazu gehören vorangehende und nachfolgende Flow-Elemente, Datenobjekte, assoziierte Elemente, sowie Informationen über den Pool und die Lane, in denen sich das Element befindet. Das Parsen des \ac{BPMN}-XML erfolgt mit der \emph{Camunda BPMN Model API}, die das XML in ein Objektmodell überführt \cite{camunda-bpmn-model-api, camunda-bpmn-model-read}. Auf dieser Basis werden die relevanten Informationen extrahiert und in der Datenklasse \texttt{BpmnElement} strukturiert abgelegt. Die Datenklasse ist in Listing \ref{lst:bpmn-element-class} zu sehen. Dadurch entsteht für jedes Flow-Element ein umfassender Kontext, der später im Prompt genutzt wird, um dem \ac{LLM} alle notwendigen Informationen strukturiert bereitzustellen. Außerdem werden durch das Format Tokens eingespart, da irrelevante Informationen, wie die Positionen der Elemente im XML, weggelassen werden. In Abbildung \ref{fig:architecture-diagram} ist dieser Schritt über die Aktivität \enquote{Kontext aller Elemente aufbauen} dargestellt.

\begin{lstlisting}[language=Kotlin,caption={Interne \ac{BPMN}-Repräsentation je Flow-Element.},label={lst:bpmn-element-class}]
data class BpmnElement(
   val type: String,
   val id: String,
   val name: String? = null,
   val documentation: String? = null,
   val poolName: String? = null,
   val laneName: String? = null,
   val outgoingFlowElementIds: List<String> = emptyList(),
   val incomingFlowElementIds: List<String> = emptyList(),
   val outgoingMessageFlowsToElementIds: List<String> = emptyList(),
   val incomingMessageFlowsFromElementIds: List<String> = emptyList(),
   val incomingDataFromElementIds: List<String> = emptyList(),
   val outgoingDataToElementIds: List<String> = emptyList(),
   val associatedElementIds: List<String> = emptyList()
)
\end{lstlisting}
\section{Prompt Engineering}\label{sec:prompt-engineering}

\begin{itemize}
    \item Referenz auf Zero-Shot/Few-Shot (?)
    \item Prompt-Sprache (Passendes Paper referenzieren)
    \item Erklärung des System-Prompt (Anleitung was als kritisch klassifiziert werden soll, Auflistung wichtiger DSGVO-kritischer Inhalte und Definitionen aus der DSGVO wie ``Verarbeitung'', Definition des Ausgabeformats mit LangChain4j) (System-Prompt im Anhang beifügen oder hier direkt integrieren)
    \item Was steht im User-Prompt drin
\end{itemize}
\section{Validierung der Ausgabe}\label{sec:validierung-der-ausgabe}

Zusätzlich zu den in Kapitel \ref{sec:prompt-engineering} beschriebenen Maßnahmen zur Sicherstellung strukturierter Ausgaben wird die Antwort des \ac{LLM} durch LangChain4j validiert. Wenn das erwartete Schema vom \ac{LLM} nicht eingehalten wird, wirft Langchain4j eine \texttt{OutputParsingException}.

\textcolor{orange}{// TODO Eventuell noch Retry Mechanismus wenn am Ende noch Zeit ist mit Nachrichtenverlauf und Parsing Fehlermeldung. Das muss aber auch erst einmal implementiert werden. Aktuell gibt es nur den Parsing Error.}

Ein weiterer Mechanismus zur Validierung der Ausgabe ist die Überprüfung der ausgegebenen \texttt{ids}. Da das \ac{LLM} theoretisch jede beliebige \texttt{id} ausgeben könnte, werden die ausgegebenen \texttt{ids} mit den tatsächlichen \texttt{ids} der Aktivitäten aus dem Prozess abgeglichen. Falls eine ausgegebene Aktivität nicht im Prozess existiert, wird diese aus der Antwort der Klassifizierung entfernt. Dadurch wird sichergestellt, dass nur gültige \texttt{ids} in der finalen Ausgabe enthalten sind.

Nachdem es nun möglich ist, \ac{BPMN}-Prozesse vorzuverarbeiten, zu klassifizieren und die Ausgabe zu validieren, folgt als nächster Schritt die Definition einer Schnittstelle zum Aufruf der Pipeline. Das nächste Kapitel beschreibt dafür das API-Design.
\section{API Design}\label{sec:api}

\begin{itemize}
    \item Klassifizierungspipeline ist über HTTP-Endpunkt aufrufbar, wo das BPMN-XML und ggf. Attribute zum Überschreiben des genutzten LLMs übergeben werden
    \item Klassifizierungspipeline kann außerdem lokal über CLI gestartet werden und legt Ergebnisse in Datei ab
    \item Dadurch kann die Klassifizierung leicht in das Evaluationsframework eingebunden werden, während das Evaluationsframework flexibel bleibt und beliebige Klassifizierungsalgorithmen benutzen kann, welche ebenfalls die gleiche HTTP-Schnittstelle anbieten
    \item Beispielaufruf des Klassifizierungsendpunkts
    \item Irgendwo hier oder auch in einem nächsten Abschnitt die ``Schnittstelle'' mit BPMN-XML rein und JSON mit Liste der kritischen Elemente (evtl. mit Begründung) wieder. Durch diese Vereinheitlichung ist es möglich verschiedene Algorithmen in dem Evaluationsframework zu vergleichen (Stichwort Standardisierung). Vielleicht gehe ich auch soweit, dass ich den Standard für zukünftige Arbeiten propose, damit spätere Arbeiten meinen Standard zum Vergleich nutzen können
\end{itemize}
\section{Nutzung über Webapp}\label{sec:nutzung-uber-webapp}

Zur interaktiven Nutzung der Klassifizierung wurde eine \emph{Sandbox} in Form einer Webapp entwickelt. Sie verbindet einen vollwertigen \ac{BPMN}-Editor auf Basis von \texttt{BPMN.js} \cite{bpmn-js} mit der in Kapitel~\ref{sec:api-design} beschriebenen HTTP-Schnittstelle und macht die Analyse damit ganz einfach bedienbar. In der Sandbox können \ac{BPMN}-Modelle erstellt, verändert, exportiert und importiert sowie auf Datenschutzrelevanz analysiert werden. Als kritisch klassifizierte Aktivitäten werden nach der Analyse direkt im Editor farblich hervorgehoben, wie in Abbildung \ref{fig:sandbox-frontend-analyzed-model} zu sehen.

\begin{figure}
    \centering
    \includegraphics[width=\linewidth]{images/sandbox/sandbox-analyzed-model}
    \caption{Sandbox im Frontend mit hervorgehobenen kritischen Aktivitäten nach Analyse.}
    \label{fig:sandbox-frontend-analyzed-model}
\end{figure}

Außerdem können die vom \ac{LLM} generierten Begründungen zu jeder als kritisch erkannten Aktivität im Editor eingesehen werden. Diese Erläuterungen werden gesammelt in einer aufklappbaren Karte im unteren Bereich des Editors angezeigt, siehe Abbildung \ref{fig:sandbox-frontend-ai-reasoning}.

\begin{figure}
    \centering
    \includegraphics[width=\linewidth]{images/sandbox/sandbox-ai-reasoning}
    \caption{Begründung der Klassifikation durch das LLM in der Sandbox.}
    \label{fig:sandbox-frontend-ai-reasoning}
\end{figure}

Um verschiedene \acp{LLM} vergleichen zu können, verfügt die Sandbox auf der rechten Seite über ein Einstellungsmenü mit konfigurierbaren \ac{LLM}-Parametern (siehe Abbildung \ref{fig:sandbox-frontend-analyzed-model}). Diese Parameter sind identisch zu den in Kapitel \ref{sec:api-design} beschriebenen \texttt{llmProps} und werden beim Absenden der Analyse in die API-Anfrage überführt.


\chapter{Labeling und Datensätze}\label{ch:labeling-und-datensatze}

Für die Evaluation der Klassifikation ist es nötig, zuvor entsprechende Testdatensätze mit Annotationen bereitzustellen. Ein solcher Datensatz besteht dabei aus mehreren Testfällen, wobei jeder Testfall ein \ac{BPMN}-Prozessmodell darstellt. Standardisierte Datensätze gewährleisten einheitliche Prüfbedingungen und ermöglichen so objektive Leistungsvergleiche.

\section{Labeling Tool}\label{sec:labeling-tool}

Hier brauche noch genaue Kapitel, aber hier soll auf jeden Fall folgendes rein:
\begin{itemize}
    \item Definition von Labeln hier, oder kommt das schon vorher? (Muss im Verlauf beim schreiben entschieden werden)
    \item Labeling Software wurde entwickelt
    \item Es können Datensätze mit Namen und Beschreibungen erstellt werden
    \item Für jeden Datensatz können beliebig viele Testcases erstellt werden
    \item Ein Testcase ist ein BPMN Diagramm, welches direkt in der App mithilfe von bpmn.io bearbeitet werden kann
    \item Zusätzlich bietet der Editor eine Labeling Funktionalität, mit welcher Aktivitäten, welche als kritisch erkannt werden sollen, gelabelt werden. Zusätzlich kann auch eine Erklärung angegeben werden
    \item Datensätze und gelabelte Testcases werden in Datenbank gespeichert und werden während der Evaluierung des Evaluationsframeworks benutzt
\end{itemize}
\section{Quellen und Eigenschaften der Datensätze}\label{sec:quellen-und-eigenschaften-der-datensatze}

Für die Evaluation wurden drei Gruppen von \ac{BPMN}-Datensätzen eingesetzt:

\begin{enumerate}
    \item Prozesse, die von der Universität Ulm bereitgestellt wurden (z.B.\ Lehrbeispiele aus Übungsaufgaben).
    \item Realistische, mittelgroße Szenarien aus verschiedenen Domänen. Diese Prozesse beinhalten Elemente wie Pools, Lanes, Datenobjekte und Gateways.
    \item Kleine, reduzierte Testfälle mit maximal fünf Aktivitäten und wenigen weiteren Elementen (z.B. einfacher Sequenzfluss ohne Pools).
\end{enumerate}

Diese heterogene Auswahl ist bewusst getroffen worden, da die Mischung aus verschiedenen Domänen und Modellkomplexitäten eine aussagekräftige Evaluation ermöglicht. In der Literatur wird betont, dass eine erhöhte Datensatzvielfalt die Robustheit der Bewertung steigert und einseitige Ergebnisse vermeidet \cite{blake2025datasetdiversity}. Tabelle \ref{tab:datensaetze-eckdaten} zeigt die Eckdaten der Datensätze.

\textcolor{orange}{// TODO Noch mehr Testfälle hinzufügen - insbesondere, um auf die Datenassoziationsthematik aus dem Kapitel \ref{sec:bpmn} einzugehen. Ich sollte also mindestens ein kleines Beipsiel jeweils mit und ohne Datenassoziation haben, um den Unterschied zu verdeutlichen. Das bedeutet aber auch, dass ich die Tabelle \ref{tab:datensaetze-eckdaten} neu berechnen muss.}

\begin{table}[htbp]
    \centering
    \begin{threeparttable}
    \caption{Eckdaten der verwendeten Datensätze}
    \label{tab:datensaetze-eckdaten}
    \begin{tabular}{l r r r}
        \toprule
        & Uni-Prozesse & Reale Szenarien & Kleine Testfälle \\
        \midrule
        Testfälle Gesamt                  & 5  & 5  & 10 \\
        Testfälle (DE)                    & 0 & 4 & 10 \\
        Testfälle (EN)                    & 5 & 1 & 0 \\
        Ø Aktivitäten $\pm$ SD\tnote{1}   & 13,4 $\pm$ 2,6 & 11,6 $\pm$ 4,2 & 5 $\pm$ 0 \\
        Ø Aktivitäten (kritisch) $\pm$ SD & 8,6 $\pm$ 3,6 & 6,6 $\pm$ 1,9 & 2,6 $\pm$ 1,5 \\
        Ø Datenobjekte $\pm$ SD           & 1,4 $\pm$ 1,9 & 3,6 $\pm$ 2,1 & 0 $\pm$ 0 \\
        Ø Datenassoziationen $\pm$ SD     & 2,4 $\pm$ 3,3 & 7 $\pm$ 4 & 0 $\pm$ 0 \\
        Ø Ereignisse $\pm$ SD             & 21 $\pm$ 13,8 & 8,2 $\pm$ 2,8 & 2 $\pm$ 0 \\
        Ø Gateways $\pm$ SD               & 13 $\pm$ 7,6 & 1,8 $\pm$ 1,5 & 0 $\pm$ 0 \\
        Ø Pools $\pm$ SD                  & 3,4 $\pm$ 1,1 & 3 $\pm$ 1 & 0 $\pm$ 0 \\
        Ø Lanes $\pm$ SD\tnote{2}         & 3 $\pm$ 1 & 4 $\pm$ 0,7 & 0 $\pm$ 0 \\
        Ø Nachrichtenflüsse $\pm$ SD      & 9,4 $\pm$ 5,3 & 5,2 $\pm$ 0,8 & 0 $\pm$ 0 \\
        Ø Annotationen $\pm$ SD           & 1 $\pm$ 1,7 & 0 $\pm$ 0 & 0 $\pm$ 0 \\
        \bottomrule
    \end{tabular}
    \begin{tablenotes}
        \item[1] SD = Standardabweichung $s$ der jeweiligen Anzahl pro Testfall.
        \item[2] Blackbox-Pools ohne Lanes wurden nicht mitgezählt, daher kann der Durchschnittswert der Lanes geringer ausfallen als der, der Pools.
    \end{tablenotes}
    \end{threeparttable}
\end{table}

\textcolor{orange}{// TODO Hier noch einen Abschnitt wo ich ein paar besondere Testfälle hervorhebe, z.B. mit speziellen Datenassoziationen, oder reicht der Überblick mit den Eckdaten? Spätestens in den Fallstudien der Ergebnisse werde ich ja nochmal auf einzelne Testfälle eingehen.}
\section{Labeling-Guide}\label{sec:labeling-guide}

Nachfolgend wird beschrieben, nach welchen Richtlinien die Daten für die Klassifizierung \ac{DSGVO}-kritischer Aktivitäten gelabelt wurden.

Die Aktivitäten in den Testfällen sollen mit dem Label \enquote{kritisch} versehen werden, wenn sie potenziell personenbezogene Daten verarbeiten und somit im Sinne der \ac{DSGVO} relevant sein könnten. Die wichtigsten Begriffe der \ac{DSGVO} wurden bereits in Abschnitt \ref{sec:dsgvo} definiert.

Beim Labeln einer Aktivität können Grenzfälle auftreten - etwa, wenn kein Datenobjekt vorhanden ist, der Name aber auf Datenverarbeitung hindeutet (z.\,B. \enquote{Verträge archivieren}). Solche Verträge können personenbezogen sein (z.\,B. Arbeitsverträge) oder rein geschäftlich zwischen Unternehmen. In diesen Fällen wird zunächst der Kontext geprüft: Gibt es Hinweise auf personenbezogene Daten, z.\,B. über Pools/Lanes oder angrenzende Aktivitäten im Prozess? Fehlen eindeutige Hinweise, wird die Aktivität als unkritisch gelabelt. Deutet der Kontext hingegen auf die Verarbeitung personenbezogener Daten hin, z.\,B. durch einen Prozessnamen wie \enquote{Mitarbeiterverwaltung} oder vorangehende Aktivitäten wie \enquote{Mitarbeiterdaten erfassen}, erhält die Aktivität das Label kritisch. Im Zweifel wird kritisch gelabelt, um eine höhere Sensitivität zu gewährleisten.

Tabelle \ref{tab:labeling-examples} listet beispielhaft einige Aktivitäten mit ihrer Klassifikation und der zugehörigen Begründung auf.

\begin{table}[htbp]
    \centering
    \caption{Beispielhafte Aktivitäten und Label.}
    \begin{tabularx}{\textwidth}{p{0.4\textwidth} c p{0.4\textwidth}}
        \toprule
        Aktivität & Kritisch? & Kommentar \\
        \midrule
        Lieferadresse eingeben & Ja & Name, Anschrift des Kunden werden aufgenommen. \\
        Rückfrage an Kunden senden & Ja & Kontaktinformationen werden verwendet. \\
        Fall anlegen & Ja & Aktivität befindet sich im Kundenservice-Kontext, personenbezogene Daten wahrscheinlich. \\
        Sprache zu Text verarbeiten & Ja & Im Kontext eines Sprachassistenten werden biometrische Daten des Nutzers verarbeitet. \\
        Produkt versenden & Nein\textsuperscript{*} & Logistik und Sachvorgänge sind nicht per se datenschutzkritisch, solange keine neue Datenverarbeitung, wie ein Labeldruck stattfindet. \\
        Systemprotokoll auslesen & Ja & Im Kontext einer technischen Wartung können personenbezogene Daten (z.\,B. Nutzer-\texttt{id}s) enthalten sein. \\
        Logdaten archivieren (anonym) & Nein & Keine personenbezogenen Daten enthalten. \\
        Gerät kalibrieren & Nein & Im Kontext einer technischen Wartung werden keine personenbezogenen Daten verarbeitet. \\
        \bottomrule
    \end{tabularx}
    \label{tab:labeling-examples}
\end{table}
\chapter{Evaluationsframework}\label{ch:evaluationsframework}

\section{Use-Cases und Anforderungen}\label{sec:anforderungen-und-use-cases}

Das Evaluationsframework richtet sich an Forschende und Entwickler, die \acp{LLM} und Klassifizierungsalgorithmen für die Identifikation \ac{DSGVO}-kritischer \ac{BPMN}-\linebreak~Aktivitäten auswerten und miteinander vergleichen möchten. Es bietet eine einheitliche Ausführungs- und Auswertungsumgebung mit klar definierten Schnittstellen und standardisierten Berichten. In diesem Kapitel werden die Use-Cases und funktionale Anforderungen des Evaluationsframeworks beschrieben.

\subsection*{Use-Cases}

Die wichtigsten Anwendungsfälle des Evaluationsframeworks sind:

\begin{itemize}
    \item \textbf{Benchmarking von \acp{LLM}.} Systematischer Vergleich mehrerer \acp{LLM} auf denselben Datensätzen, mit identischem Algorithmus und identischen Parametern.
    \item \textbf{A/B-Vergleich von Algorithmen.} Gegenüberstellung verschiedener Klassifizierungspipelines, mit z.B.\ alternativen Prompts oder anderem Preprocessing, über eine standardisierte HTTP-Schnittstelle, die in Kapitel~\ref{sec:api-design} definiert ist.
    \item \textbf{Explorative Analyse.} Detaillierte Einsicht pro Modell und Testfall (inklusive Begründungen und Visualisierungen), um Fehlklassifikationen gezielt zu untersuchen.
    \item \textbf{Berichterstellung.} Die Ergebnisse lassen sich als JSON oder Markdown exportieren und später wieder importieren, um sie erneut untersuchen zu können. Sie eignen sich zudem für die Publikation. Die Diagramme werden automatisch erzeugt und stehen ebenfalls zum Download bereit.
\end{itemize}

In dieser Arbeit werden keine A/B-Vergleiche unterschiedlicher Klassifizierungsalgorithmen durchgeführt, sondern lediglich verschiedene \acp{LLM} mit demselben Algorithmus verglichen. Das Framework ist jedoch so konzipiert, dass dies in zukünftigen Arbeiten möglich ist.

\textcolor{orange}{// TODO In dem Absatz drüber noch erklären, warum keine A/B-Vergleiche unterschiedlicher Klassifizierungsalgorithmen durchgeführt, sondern lediglich verschiedene \acp{LLM} mit demselben Algorithmus verglichen werden.}

\subsection*{Funktionale Anforderungen}

In der folgenden Tabelle sind die funktionalen Anforderungen an das Evaluationsframework aufgelistet, die notwendig sind um die definierten Use-Cases zu erfüllen:

\begin{center}
    \reqTable{01}
    {Nutzen gelabelter Testdatensätze}
    {Das Framework kann die gelabelten Testdatensätze benutzten, die mit dem Labeling-Tool aus \ref{sec:labeling-tool} erstellt worden sind.}
    {}
\end{center}

\begin{center}
    \reqTable{02}
    {Vergleichbarkeit von Modellen und Algorithmen}
    {Das Framework erlaubt den direkten Vergleich verschiedener \acp{LLM} sowie unterschiedlicher Klassifizierungsalgorithmen anhand gelabelter Testdaten. Die Anbindung an Klassifizierungsalgorithmen erfolgt über die in Kapitel~\ref{sec:api-design} definierte, standardisierte HTTP-Schnitstelle.}
    {01}
\end{center}

\begin{center}
    \reqTable{03}
    {Deklarative Konfiguration}
    {Ein Evaluationslauf ist vollständig über eine YAML-Datei konfigurierbar. Dazu zählen Modelle, Klassifizierungsendpunkte, Testdatensätze und Seed. Experimente werden dadurch portabel und wiederholbar.}
    {02}
\end{center}

\begin{center}
    \reqTable{04}
    {Detaillierte Ergebnisaufbereitung}
    {
    Das Framework gibt Ergebnisse auf zwei Ebenen aus.
    \begin{enumerate}
        \item Pro Testfall und pro Modell: Status (\enquote{bestanden}/\enquote{nicht bestanden}), klassifizierte Elemente mit Begründungen, \ac{TP}/\ac{FP}/\ac{FN}/\ac{TN} und eine Visualisierung der Klassifikation im \ac{BPMN}-Prozess.
        \item Pro Modell als Summe über alle Testfälle: Accuracy, Precision, Recall, F1-Score und die Konfusionsmatrix.
    \end{enumerate}
    Zusätzlich protokolliert das Framework Metadaten der Evaluation, z.\,B. Endpunkt, verwendete Modelle und den Seed.
    }
    {02}
\end{center}

\begin{center}
    \reqTable{05}
    {Frontend}
    {Für eine einfache Bedienung und Ansicht der Ergebnisse bietet das Evaluationsframework ein Frontend an.}
    {02,03,04}
\end{center}

\begin{center}
    \reqTable{06}
    {Visualisierung und Berichte der Gesamtresultate}
    {Kennzahlen werden als Side-by-Side-Diagramme und tabellarisch dargestellt. Zusätzlich stehen Export/Import der Ergebnisse als JSON sowie ein Markdown-Report zur Verfügung.}
    {05}
\end{center}
\section{Architektur und Komponenten}\label{sec:architektur-und-komponenten}

\textcolor{orange}{// TODO Hier noch Fehlerbehandlung behandeln? Wenn irgendwo ein Fehler passiert, wird der Testfall als fehlgeschlagen markiert und im Bericht vermerkt. Außerdem habe ich glaub ich am Anfang auch erwähnt, dass Fehlerhafte Testfälle nicht in die Metriken einfließen, oder? Das muss ich dann auch tatsächlich noch implmentieren, weil sie aktuell als falsch klassifiziert gezählt werden.}

Das Evaluationsframework ist modular aufgebaut und nutzt eine Pipeline-Architektur, um eine flexible und skalierbare Evaluierung zu ermöglichen, wie es in \hyperlink{FA02}{FA02} gefordert ist. Die Architektur ist in Abbildung~\ref{fig:evaluation-framework-architecture} dargestellt. Sie besteht aus mehreren Hauptkomponenten, die jeweils eine klar definierte Aufgabe erfüllen. Im Folgenden werden die Komponenten und ihr Zusammenspiel beschrieben.

\begin{figure}
    \centering
    \includegraphics[width=\linewidth]{images/evaluation/evaluation-framework-architecture.drawio}
    \caption{Architektur des Evaluationsframeworks}
    \label{fig:evaluation-framework-architecture}
\end{figure}

\subsection*{Einstiegspunkte}

Das Framework bietet zwei Einstiegspunkte zur Ausführung einer Evaluierung:

\begin{itemize}
    \item \textbf{EvaluationController} als HTTP-Controller stellt REST-Endpunkte bereit, über die Evaluierungen gestartet werden können. Er akzeptiert eine YAML-Konfiguration und gibt die Ergebnisse entweder auf einmal als Markdown-Bericht oder als JSON-Stream zurück. Der Controller ermöglicht die Nutzung des Frameworks über die Weboberfläche, die in Kapitel \ref{sec:visualisierung-im-frontend} gezeigt wird, sowie über HTTP-APIs. Durch das Streamen der Ergebnisse können bereits abgeschlossene Testfälle sofort angezeigt werden, ohne auf das Ende der gesamten Evaluierung warten zu müssen.
    \item \textbf{EvaluationCommand} ist ein CLI-Befehl, der die Ausführung von Evaluierungen über die Kommandozeile erlaubt. Er liest eine YAML-Konfigurationsdatei ein, führt die Evaluierung aus und schreibt die Ergebnisse in eine Markdown-Datei. Dies eignet sich besonders für automatisierte Ausführungen, Continuous Integration oder die lokale Entwicklung.
\end{itemize}

Beide Einstiegspunkte akzeptieren die Konfiguration aus Kapitel \ref{sec:konfiguration-einer-evaluierung}, lösen ggf.\ Umgebungsvariablen auf und delegieren die Ausführung der Evaluation an den \texttt{MultiEvaluationRunner}.

\subsection*{Orchestrierung mit \texttt{MultiEvaluationRunner}}

Der \texttt{MultiEvaluationRunner} ist für die Orchestrierung der gesamten Evaluierung verantwortlich. Er verarbeitet die Konfiguration, die mehrere Modelle und Datensätze beschreibt, und koordiniert die sequenzielle Evaluierung aller konfigurierten Modelle. Für jedes Modell ruft der \texttt{MultiEvaluationRunner} den \texttt{EvaluationRunner} auf und übergibt diesem alle Informationen zur Ausführung der Evaluation eines einzelnen Modells.

Der \texttt{MultiEvaluationRunner} stellt zudem sicher, dass alle Modelle denselben Seed verwenden, um reproduzierbare Ergebnisse zu gewährleisten. Falls in der Konfiguration kein Seed angegeben wurde, wird an dieser Stelle einer erzeugt. Im Anschluss werden die Metadaten über die verwendeten Datensätze, Modelle, Endpunkte und den Seed gesammelt und als \texttt{MetadataReport} als Teil des Evaluationsberichts zurückgegeben. Die Teile des Berichts werden als \texttt{Flow} von Berichtartefakten zurückgegeben, wodurch eine streambasierte Verarbeitung ermöglicht wird. Welche Arten von Berichtartefakten es gibt, wird in kapitel \ref{sec:generierte-resultate} beschrieben.

\subsection*{Ausführung mit \texttt{EvaluationRunner}}

Der \texttt{EvaluationRunner} führt die Evaluierung für ein einzelnes Modell durch. Er lädt die Testfälle der angegebene Testdatensätze aus der Datenbank, führt die Klassifizierung für jeden Testfall aus und sammelt die Ergebnisse. Die Testfälle werden parallel verarbeitet, wobei die Anzahl der gleichzeitigen Ausführungen durch den Parameter \texttt{maxConcurrent} in der Konfiguration gesteuert wird. Dies ermöglicht es, Rate-Limits von \acp{LLM}-Diensten einzuhalten und die Auslastung der Ressourcen zu kontrollieren.

Für jeden Testfall delegiert der \texttt{EvaluationRunner} die eigentliche Klassifizierung an den \texttt{HttpEvaluator}. Anschließend vergleicht er die erwarteten mit den tatsächlichen Ergebnissen und berechnet die Klassifikationsmetriken wie \ac{TP}, \ac{FP}, \ac{FN} und \ac{TN}. Die Ergebnisse werden in \texttt{TestCaseReport}-Objekten zusammengefasst, als Teilergebnis zurückgegeben, und an den \texttt{MetricsAccumulator} weitergeleitet.

Der \texttt{EvaluationRunner} gibt die Ergebnisse ebenfalls als \texttt{Flow} zurück, wodurch eine frühzeitige Rückgabe von Teilergebnissen ermöglicht wird. Dies ist besonders vorteilhaft für die Live-Ansicht in der Weboberfläche, da Testfallergebnisse sofort nach ihrer Fertigstellung angezeigt werden können.

\subsection*{Klassifizierung mit \texttt{HttpEvaluator}}

Der \texttt{HttpEvaluator} ist für die Kommunikation mit dem Klassifizierungsendpunkt verantwortlich, der das in Kapitel \ref{sec:api-design} beschriebene Interface implementiert. Er nimmt das \ac{BPMN}-Modell aus dem aktuellen Testfall und die \texttt{llmProps} von dem aktuellen Modell aus der Konfiguration entgegen, baut einen HTTP-Request auf und sendet diesen an den konfigurierten Endpunkt. Nach erfolgreicher Klassifizierung extrahiert er die Liste der als kritisch identifizierten Aktivitäten aus der Antwort und gibt diese an den \texttt{EvaluationRunner} zurück.

\subsection*{Akkumulierung mit \texttt{MetricsAccumulator}}

Der \texttt{MetricsAccumulator} sammelt die Metriken aller Testfälle eines Modells und berechnet daraus aggregierte Werte. Er ist thread-sicher implementiert und kann gleichzeitig von mehreren parallelen Evaluierungen genutzt werden. Das ist wichtig, da der \texttt{EvaluationRunner} die Testfälle parallel ausführt und somit mehrere Threads gleichzeitig auf den \texttt{MetricsAccumulator} zugreifen können.

Nach Abschluss aller Testfälle erzeugt der \texttt{MetricsAccumulator} ein\break \texttt{EvaluationReportSummary}-Objekt, das alle Metriken für die Evaluation eines Modells über mehrere Testfälle hinweg enthält.

\subsection*{Zusammenfassung}

Die Architektur trennt Zuständigkeiten strikt:
\texttt{MultiEvaluationRunner} koordiniert Modellläufe,
\texttt{EvaluationRunner} verarbeitet Testfälle und sammelt Metriken,
\texttt{HttpEvaluator} kommuniziert mit der Klassifizierungs-Pipeline,
\texttt{MetricsAccumulator} aggregiert Ergebnisse pro Modell über mehrere Testfälle.

\section{Konfiguration einer Evaluierung}\label{sec:konfiguration-einer-evaluierung}

Die funktionale Anforderung \hyperlink{FA03}{FA03} fordert, dass Evaluationsläufe deklarativ konfiguriert werden können. Das Framework unterstützt dies auf zwei Wegen:
Erstens bietet die Weboberfläche, die in \ref{sec:visualisierung-im-frontend} gezeigt wird, die Möglichkeit, Evaluationsläufe interaktiv zu konfigurieren und zu starten.
Zweitens lässt sich eine Evaluierung über eine YAML-Datei beschreiben, die entweder in der Weboberfläche hochgeladen oder per CLI an das Evaluationsframework übergeben wird. Auf diese Weise werden Reproduzierbarkeit und Versionierung der Evaluationsläufe sichergestellt. Listing \ref{lst:evaluation-config} zeigt ein Beispiel für eine solche YAML-Konfiguration. Ein ausführliches JSON-Schema ist im Anhang (Listing \ref{lst:evaluation-config-schema}) zu finden.

Die Evaluierungskonfiguration umfasst die folgenden Bausteine:

\begin{itemize}
    \item \texttt{defaultEvaluationEndpoint} ist der Standardendpunkt für die Klassifizierung. Er wird verwendet, wenn für ein Modell kein eigener Endpunkt angegeben ist. Der Endpunkt muss die in Kapitel \ref{sec:api-design} beschriebene API-Spezifikation erfüllen und kann relativ (gegen die Basis-URL des Evaluationsframeworks) oder absolut (für einen externen Dienst) angegeben werden.
\end{itemize}

\begin{lstlisting}[caption={Beispiel einer Evaluierungskonfiguration in YAML.},label={lst:evaluation-config}]
defaultEvaluationEndpoint: /gdpr/analysis/prompt-engineering
maxConcurrent: 10
repetitions: 3
seed: 42
models:
  - label: Mistral Medium 3.1
    llmProps:
      baseUrl: https://openrouter.ai/api/v1
      modelName: mistralai/mistral-medium-3.1
      apiKey: ${OPEN_ROUTER_API_KEY}
      topP: 1
  - label: Deepseek Chat v3.1
    llmProps:
      baseUrl: https://openrouter.ai/api/v1
      modelName: deepseek/deepseek-chat-v3.1
      apiKey: ${OPEN_ROUTER_API_KEY}
      temperature: 0.1
  - label: GPT oss 120b
    llmProps:
      baseUrl: https://openrouter.ai/api/v1
      modelName: openai/gpt-oss-120b
      apiKey: ${OPEN_ROUTER_API_KEY}
datasets:
  - 2
  - 7
\end{lstlisting}

\begin{itemize}
    \item \texttt{maxConcurrent} gibt die maximale Anzahl parallel auszuführender Testfälle an. So lassen sich beispielsweise \emph{Rate Limits}\footnote{Providerseitige Begrenzungen, etwa \enquote{Requests pro Minute} oder maximale Parallelität. Bei Überschreitung antworten viele Anbieter mit HTTP~429 (\enquote{Too Many Requests}). Zudem drohen strengere Drosselungen.} der angebundenen \acp{LLM} einhalten, um technische Fehler in den Ergebnissen zu vermeiden.
    \item \texttt{repetitions} bestimmt, wie oft die Evaluierung pro Modell wiederholt wird. Die Ergebnisse werden später über alle Wiederholungen aggregiert (siehe Abschnitt~\ref{sec:generierte-resultate}).
    \item \texttt{seed} legt einen Startwert (Seed) für reproduzierbare Evaluationsläufe fest. Auf Basis des Seeds und der Wiederholungsnummer wird für jede Wiederholung deterministisch ein eigener Seed generiert, um unterschiedliche, aber reproduzierbare Ergebnisse zu erzielen. Er wird bei jedem Modell an die \texttt{llmProps} weitergereicht und bei der Kommunikation mit den \acp{LLM} verwendet, sofern diese einen Seed unterstützen.
    \item \texttt{models} enthält die zu evaluierenden Modelle. Jedes Modell besitzt ein \texttt{label} zur Identifikation und optional spezifische \texttt{llmProps}, um die Eigenschaften des verwendeten \acp{LLM} zu definieren. Diese sind identisch zu den in Kapitel \ref{sec:api-design} beschriebenen \texttt{llmProps}.
    \item \texttt{datasets} ist eine Liste von Datensatz-\texttt{ids}, die jeweils eine Menge von Testfällen beinhalten.
\end{itemize}

Wie im Schema in Listing \ref{lst:evaluation-config-schema} gezeigt, kann jedem Modell optional ein eigener\linebreak~\texttt{evaluationEndpoint} zugewiesen werden, der den in \texttt{defaultEvaluation-\linebreak~Endpoint} definierten Standard überschreibt. Dadurch lassen sich unterschiedliche Klassifizierungsalgorithmen oder -versionen gezielt pro Modell vergleichen. Ist kein spezifischer Endpunkt angegeben, greift automatisch der Standardendpunkt.

API-Keys in den \texttt{llmProps} können optional als Umgebungsvariablen referenziert werden, wie im Beispiel in Listing \ref{lst:evaluation-config} gezeigt. So lassen sich sensible Daten sicher handhaben, ohne sie direkt in der Konfigurationsdatei zu speichern. Die Umgebungsvariablen werden zur Laufzeit aufgelöst und müssen daher im Kontext der Anwendung verfügbar sein.
\section{Testdaten}\label{sec:testdaten}

Wie in \hyperlink{FA01}{FA01} beschrieben, kann das Evaluierungsframework die mit dem Labeling-Tool erzeugten Testdatensätze unmittelbar verwenden. Da das Tool die Testdaten in einer Datenbank ablegt, lassen sie sich unkompliziert auslesen und für die Evaluierung heranziehen. In der Konfiguration des Frameworks wird festgelegt, welche Datensätze genutzt werden, wodurch sich die Auswertung gezielt auf einen bestimmten Anwendungsfall zuschneiden lässt. Wie die Konfiguration einer Evaluierung funktioniert, wird im nächsten Kapitel erläutert.
\section{Evaluationsergebnisse}\label{sec:generierte-resultate}

Im vorherigen Abschnitt wurde erwähnt, dass die Komponenten des Evaluationsframeworks Berichtartefakte zurückgeben. Diese sind im im Architekturbild \ref{fig:evaluation-framework-architecture} als Beschriftungen über den gestrichelten Pfeilen dargestellt. Im Folgenden werden die Berichtartefakte beschrieben:

\begin{description}
    \item[\texttt{MetadataReport}] Berichtsartefakt, das der \texttt{MultiEvaluationRunner} zu Beginn der Evaluierung erzeugt. Es enthält Metadaten zur Evaluierung, z.\,B.\ Informationen über die Testdatensätze, die Anzahl der Testfälle sowie den verwendeten Seed. Das \texttt{MetadataReport}-Artefakt wird zuerst zurückgegeben, damit die Weboberfläche bereits Metadaten anzeigen kann, während die Evaluierung noch läuft.

    \item[\texttt{TestCaseReport}] Berichtsartefakt, das der \texttt{EvaluationRunner} für jeden abgeschlossenen Testfall erzeugt. Es enthält u.\,a.\ die Testfall-\texttt{id}, das Klassifizierungsergebnis und die für diesen Testfall berechneten Metriken. \texttt{TestCaseReport}-Artefakte werden fortlaufend bereitgestellt, sobald ein Testfall abgeschlossen ist, sodass die Weboberfläche Ergebnisse unmittelbar anzeigen kann.

    \item[\texttt{EvaluationReportSummary}] Berichtsartefakt, das der \texttt{MetricsAccumulator} am Ende der Evaluierung eines Modells erzeugt. Es fasst die aggregierten Metriken, wie z.\,B.\ \emph{Precision}, \emph{Recall}, \emph{F1-Score} und \emph{Accuracy}, sowie die Konfusionsmatrix zusammen. Das \texttt{EvaluationReportSummary}-Artefakt wird als letztes Berichtsartefakt pro Modell zurückgegeben und dient dem Modellvergleich in der Weboberfläche.
\end{description}

Die Informationen dieser Berichtsartefakte ermöglichen die Generierung eines ausführlichen Evaluierungsberichts, wie in \hyperlink{FA04}{FA04} gefordert. Im Folgenden ist dargestellt, welche Informationen nach Abschluss einer Evaluierung vorliegen.

\subsection*{Pro Testfall und Modell}

Für jeden Testfall eines Modells liegen vor: die von der Klassifizierungs-Pipeline zurückgegebenen klassifizierten Aktivitäten (mit optionalen Begründungen), die gelabelten erwarteten Aktivitäten, die Zählwerte für \emph{\ac{TP}}, \emph{\ac{FP}}, \emph{\ac{FN}} und \emph{\ac{TN}} sowie eine Bild-URL zur Visualisierung des \ac{BPMN}-Modells mit hervorgehobenen Aktivitäten. Aus diesen Informationen lässt sich ableiten, ob der Testfall erfolgreich war. Ein Testfall gilt als erfolgreich, wenn die klassifizierten Aktivitäten exakt den erwarteten Aktivitäten entsprechen. Technische Probleme, die während der Klassifizierung auftreten, werden ebenfalls erfasst, z.\,B.\ Parsing-Fehler, ungültiges \ac{BPMN}, Token-Limit-Überschreitungen oder Zeitüberschreitungen.

\subsection*{Pro Modell über alle Testfälle}

Auf Modellebene stehen die Gesamtergebnisse über alle Testfälle zur Verfügung. Dazu gehören die aggregierten Kennzahlen \emph{Precision}, \emph{Accuracy}, \emph{Recall} und \emph{F1-Score} sowie eine Konfusionsmatrix mit den Gesamtwerten für \emph{\ac{TP}}, \emph{\ac{FP}}, \emph{\ac{FN}} und \emph{\ac{TN}}. Zusätzlich sind die Anzahlen der korrekt bzw.\ falsch klassifizierten sowie der technisch fehlgeschlagenen Testfälle aufgeführt.

\subsection*{Über alle Modelle}

Abschließend sind die Metadaten der gesamten Evaluierung verfügbar: die verwendeten Testdatensätze, die Anzahl der Testfälle, die konfigurierten Modelle, der für die Reproduzierbarkeit verwendete Seed sowie ein Zeitstempel der Evaluierung. Zum unmittelbaren Vergleich werden die aggregierten Kennzahlen aller Modelle nebeneinander dargestellt.

\section{Nutzung über Webapp}\label{sec:nutzung-uber-webapp}

Zur interaktiven Nutzung der Klassifizierung wurde eine \emph{Sandbox} in Form einer Webapp entwickelt. Sie verbindet einen vollwertigen \ac{BPMN}-Editor auf Basis von \texttt{BPMN.js} \cite{bpmn-js} mit der in Kapitel~\ref{sec:api-design} beschriebenen HTTP-Schnittstelle und macht die Analyse damit ganz einfach bedienbar. In der Sandbox können \ac{BPMN}-Modelle erstellt, verändert, exportiert und importiert sowie auf Datenschutzrelevanz analysiert werden. Als kritisch klassifizierte Aktivitäten werden nach der Analyse direkt im Editor farblich hervorgehoben, wie in Abbildung \ref{fig:sandbox-frontend-analyzed-model} zu sehen.

\begin{figure}
    \centering
    \includegraphics[width=\linewidth]{images/sandbox/sandbox-analyzed-model}
    \caption{Sandbox im Frontend mit hervorgehobenen kritischen Aktivitäten nach Analyse.}
    \label{fig:sandbox-frontend-analyzed-model}
\end{figure}

Außerdem können die vom \ac{LLM} generierten Begründungen zu jeder als kritisch erkannten Aktivität im Editor eingesehen werden. Diese Erläuterungen werden gesammelt in einer aufklappbaren Karte im unteren Bereich des Editors angezeigt, siehe Abbildung \ref{fig:sandbox-frontend-ai-reasoning}.

\begin{figure}
    \centering
    \includegraphics[width=\linewidth]{images/sandbox/sandbox-ai-reasoning}
    \caption{Begründung der Klassifikation durch das LLM in der Sandbox.}
    \label{fig:sandbox-frontend-ai-reasoning}
\end{figure}

Um verschiedene \acp{LLM} vergleichen zu können, verfügt die Sandbox auf der rechten Seite über ein Einstellungsmenü mit konfigurierbaren \ac{LLM}-Parametern (siehe Abbildung \ref{fig:sandbox-frontend-analyzed-model}). Diese Parameter sind identisch zu den in Kapitel \ref{sec:api-design} beschriebenen \texttt{llmProps} und werden beim Absenden der Analyse in die API-Anfrage überführt.
\section{Erweiterbarkeit}\label{sec:erweiterbarkeit}

\begin{itemize}
    \item Neue Modelle können über Konfiguration ergänzt werden (Endpunkt, Modellname, API Key)
    \item Neue Klassifizierungsalgorithmen/pipelines können ergänzt werden, wenn sie das vorgegebene HTTP-Schnittstelle unterstützen
    \item Die Testdatensätze werden aus Datenbank ausgelesen und können für jeden Evaluations-Run neu gesetzt werden
\end{itemize}
\chapter{Modellauswahl}\label{ch:modellauswahl}

\section{Kriterien}\label{sec:kriterien}

Dieser Abschnitt legt die Auswahlkriterien der \acp{LLM} offen, nach denen die Modelle ausgewählt und kategorisiert wurden. Tabelle \ref{tab:kriterien} zeigt eine Übersicht der Kriterien. Diese Kriterien helfen, die Modelle systematisch zu vergleichen und ihre Eignung für die Klassifizierungsaufgabe zu bewerten.

\begin{table}[htbp]
    \centering
    \caption{Übersicht der Kriterien zur Modellauswahl}
    \label{tab:kriterien}
    \begin{tabularx}{\textwidth}{p{0.3\textwidth} p{0.65\textwidth}}
        \toprule
        \textbf{Kriterium} & \textbf{Beschreibung} \\
        \midrule
        Herkunft & Das Land in dem das Modell entwickelt wurde bzw. der Hauptsitz des Anbieters \\
        Lizenz & Art der Lizenz, wie bspw. Open-Source oder proprietär \\
        Größe & Anzahl der Parameter in Milliarden (B) \\
        Kontext & Maximale Anzahl der Token, die das Modell verarbeiten kann \\
        Letztes Update & Datum der letzten Aktualisierung des Modells bei Hugging Face \cite{huggingface} \\
        Downloads & Anzahl der Downloads des Modells bei Hugging Face, sofern verfügbar \\
        \bottomrule
    \end{tabularx}
\end{table}

Ein wesentliches Auswahlkriterium ist die geografische \textbf{Herkunft} der Modelle. Als \emph{\ac{EU}-Modell} gelten Modelle, deren Anbieter ihren Hauptsitz in der \ac{EU} haben, deren Veröffentlichung in der \ac{EU} erfolgt oder die schwerpunktmäßig in der \ac{EU} entwickelt oder verfeinert wurden. Alle anderen Modelle werden als \emph{international} eingeordnet. Diese Unterscheidung ist relevant, da europäische Modelle sowohl beim Training als auch beim Betrieb stärker den europäischen Datenschutzbestimmungen unterliegen und somit potenziell besser für den Einsatz in datensensiblen Bereichen wie der Klassifizierung von \ac{BPMN}-Modellen geeignet sind. \textcolor{orange}{// TODO Hier noch erwähnen, dass \ac{DSGVO} ja ein europäisches Gesetz ist}

Ein weiteres zentrales Kriterium ist die \textbf{Lizenzierung} der Modelle. Als \emph{Open-Source} oder \emph{Open-Weights} veröffentlichte Modelle bieten mehr Flexibilität und Transparenz. Sie sind häufig lizenzkostenfrei nutzbar und lassen sich eigenständig betreiben, was insbesondere ein Vorteil für Unternehmen und Organisationen mit strengen Datenschutzanforderungen ist. Proprietäre Modelle sind dagegen an spezifische Nutzungsbedingungen gebunden und bringen teils Einschränkungen bei Datenverarbeitung und -speicherung mit sich. \textcolor{orange}{// TODO Im folgendes ist zum einen nicht ganz klar was genau die Definition von Open-Source ist und der SPrung von Open-Source zu open-Weights ist nicht ganz günstig. Vielleicht hier noch kurz erklären was der Unterschied zwiscehn den beiden ist.} Im engeren Sinne definiert die \ac{OSI} Open-Source-Lizenzen über die \emph{Open Source Definition} \cite{OSI_OSD}. Viele aktuelle \acp{LLM} erscheinen als \emph{Open-Weights}, was bedeutet, dass die Gewichte frei beziehbar sind. Die Lizenz kann jedoch restriktive Klauseln enthalten (z.\,B. die Meta-Llama\,3 Community Lizenz mit Nutzungs- und Output-Beschränkungen \cite{Llama3_License}). In dieser Arbeit gilt ein Modell als \emph{offen} bzw.\ \emph{Open-Source-nah}, wenn

\begin{enumerate}
    \item die Gewichte frei zugänglich sind und
    \item eine \emph{permissive} Lizenz (z.\,B. Apache-2.0 oder MIT) eine breite kommerzielle Nutzung erlaubt (z.\,B. Mistral 7B \cite{HF_Mistral7B_2025}, GPT-OSS \cite{OpenAI_GPTOSS_ModelCard_2025, OpenAI_GPTOSS_Blog_2025} oder DeepSeek V3.1 \cite{HF_DeepSeek_V3_1_2025}).
\end{enumerate}

Modelle mit \emph{Community}- oder \emph{Eigennutzer}-Lizenzen (z.\,B. Mistral Large Instruct unter Mistral Research License \cite{HF_MistralLargeInstruct_2025, MRL_Research_License}) werden rechtlich \emph{nicht} als \ac{OSI}‑Open‑Source gewertet, können aber technisch als Vergleich herangezogen werden.

Die \textbf{Modellgröße} wird in \emph{Anzahl der Parameter} angegeben. Meist in \emph{Milliarden} (Billionen, engl. \emph{Billion}) Parametern (B, engl. \emph{Billion}). 1\,B = \(10^9\) Parameter. Diese Zahl korreliert mit dem Ressourcenbedarf für Training und Inferenz sowie der Leistungsfähigkeit \cite{webdev-llm-sizes}. Für die Einordnung werden hier folgende Klassen verwendet:

\begin{itemize}
    \item \textbf{Klein} (\(\leq\)\,\textasciitilde{}25\,B Parameter): z.\,B. Mistral\,7B Instruct (\textasciitilde{}7.3\,B) \cite{HF_Mistral7B_2025}.
    \item \textbf{Groß} (\(>\)\,\textasciitilde{}25\,B Parameter): z.\,B. GPT‑OSS\,120B (\textasciitilde{}117\,B) \cite{OpenAI_GPTOSS_ModelCard_2025}.
\end{itemize}

Die Klassifikation dient als methodische Abgrenzung für die Experimente. Kleinere Modelle lassen sich häufig lokal ausführen, größere erfordern typischerweise mehrere GPUs. Parameterzahl ist dabei ein nützlicher, wenn auch unvollständiger Indikator für Ressourcenbedarf und erwartete Leistung. Dies ermöglicht konsistente Entscheidungen zu Deployment und Kosten.

Der \textbf{Kontext} gibt an, wie viele Token ein Modell gleichzeitig verarbeiten kann. Ein Token ist dabei eine Grundeinheit von Text, die ein Wort, einen Teil eines Wortes oder sogar ein einzelnes Zeichen darstellen kann. Die Größe des Kontextfensters beeinflusst maßgeblich, wie gut ein Modell längere Texte verstehen und darauf reagieren kann \cite{ibm-llm-context}.

Neben den genannten Hauptkriterien werden weitere Merkmale erfasst, um die Modelle umfassend zu charakterisieren. Dazu gehören das \textbf{letzte Update} des Modells, um einordnen zu können, wie aktuell das Modell ist, und wie viele \textbf{Downloads} das Modell hat, sofern verfügbar. Diese Informationen helfen, die Popularität und Akzeptanz der Modelle in der Community einzuschätzen.

Im nächsten Abschnitt werden auf Basis dieser Kriterien die ausgewählten Modelle vorgestellt.
\section{Modellvorstellung}\label{sec:modellvorstellung}

\begin{itemize}
    \item Tabellarische Auflistung der Modelle
    \item Begründung warum genau die Modelle ausgewählt worden sind
\end{itemize}
\chapter{Versuchsaufbau}\label{ch:versuchsaufbau}

Ich habe noch nicht die Unterkapitel (außer Metriken) aber folgendes soll hier u.\,a. rein:

\begin{itemize}
    \item Die Modelle werden jeweils mit dem gleichen Klassifizierungsalgorithmus und den gleichen Datensätzen benutzt
    \item Die Evaluierung wird jeweils für jeden vorhandenen Testdatensatz durchgeführt (Uni, reale größere Prozesse, kleine Prozesse)
    \item Die Evaluationen werden pro Konfiguration (Modelle, Datensätze) mehrfach durchgeführt (Unterschiedliche Seeds, falls ich bis dahin Seeds unterstütze)
    \item Es werden verschiedene LLM-Modelle vergleichen
    \item Es werden vom gleichen LLM-Modell die unterschiedlichen Größen verglichen
    \item \ldots
    \item Ein Testcase gilt als korrekt klassifiziert, wenn genau die als kritisch gelabelten Aktivitäten als kritisch klassifiziert worden sind. Sobald es False Positives oder False Negatives gibt, ist ein Testcase nicht korrekt klassifiziert worden
    \item LLM Temperature 0 vorstellen (Falls ich das mit den Seeds noch umsetze) für reproduzierbare Ergebnisse
\end{itemize}

\chapter{Durchführung}\label{ch:durchfuhrung}

\section{Experimente}\label{sec:experimente}

\begin{itemize}
    \item Dokumentation der einzelnen Runs
    \begin{itemize}
        \item Konfiguration aufzeigen
        \item Diagramme der Ergebnisse
    \end{itemize}
\end{itemize}
\section{Fehlerklassen}\label{sec:fehlerklassen}

\begin{itemize}
    \item Falls es Fehlerklassen gibt kann ich sie hier thematisieren (Timeout, Parse-Error, Token Limit, \ldots)
\end{itemize}
\chapter{Ergebnisse}\label{ch:ergebnisse}

\section{Gesamtübersicht}\label{sec:gesamtubersicht}

\begin{itemize}
    \item Hier allgemein die Ergebnisse der einzelnen Runs darstellen (Vor allem die Diagramme wo die einzelnen Metriken nebeneinander aufgelistet sind)
\end{itemize}
\section{Analyse nach Modellkategorien}\label{sec:analyse-nach-modellkategorien}

Für ein besseres Verständnis der Leistungsunterschiede werden die Modelle im Folgenden nach verschiedenen Kriterien gruppiert und verglichen. Dabei wird jeweils diskutiert, wie sich die Gruppen in Bezug auf die Qualitätsziele unterscheiden und welche Trends sich beobachten lassen. Die Implikationen für Praxis und Forschungsfragen werden am Ende dieses Kapitels in Abschnitt~\ref{sec:antworten-auf-forschungsfragen} gebündelt beantwortet.

\subsection*{Proprietäre versus Open-Weight Modelle}

Die beiden proprietären Modelle \texttt{GPT-4o} und \texttt{Mistral Medium 3.1} erreichen F1-Scores von $0{,}822$ bzw.\ $0{,}843$. Trotz seiner exzellenten Precision von $0{,}892$ verfehlt \texttt{GPT-4o} aufgrund des niedrigen Recalls von $0{,}762$ das Mindestziel und übersieht damit relativ viele kritische Aktivitäten. \texttt{Mistral Medium 3.1} bietet mit einem Recall von $0{,}877$ eine bessere Balance und erfüllt alle Qualitätsziele.

Die offene Kategorie zeigt ein heterogenes Bild. Mehrere Modelle wie \texttt{Qwen3-235B-\linebreak~A22B-Thinking-2507} mit einem F1-Score $= 0{,}874$, \texttt{GPT-OSS-20B} mit F1-Score $= 0{,}866$ und \texttt{DeepSeek-R1-Distill-Qwen-14B} mit F1-Score $= 0{,}848$ übertreffen die proprietären Modelle. Sie erkennen kritische Aktivitäten sehr zuverlässig und klassifizieren nur wenige unkritische Aktivitäten fälschlich als kritisch. Gleichzeitig gibt es mit \texttt{Mixtral-8x7B-Instruct-v0.1}, das F1-Score $= 0{,}596$ erzielte, auch klare Ausreißer nach unten, die weder genug kritische Aktivitäten erkennen noch eine akzeptable Precision bieten.

Insgesamt zeigt sich, dass hochwertige offene Modelle ein besseres Verhältnis von Recall und Precision aufweisen und die Qualitätsziele häufig klar erfüllen. Für die Praxis bedeutet dies, dass offene Modelle eine attraktive Alternative zu proprietären Lösungen darstellen, jedoch ist die Auswahl des Modells entscheidend, da die Leistungsunterschiede innerhalb der offenen Kategorie erheblich sind.

\subsection*{Kleine versus Große Modelle}

Tabelle \ref{tab:small-vs-large} vergleicht die Mittelwerte der Metriken für kleine Modelle ($\leq 25$\,B Parameter) und große Modelle ($>25$\,B Parameter). Im Durchschnitt unterscheiden sich die Gruppen nur geringfügig. Die kleinen Modelle erreichen einen mittleren F1-Score von $0{,}805$ und die großen Modelle $0{,}806$, wobei die großen Modelle ohne den Ausreißer \texttt{Mixtral-8x7B-Instruct-v0.1} einen leicht höheren Durchschnitt von $0{,}836$ erreichen. Der beste F1-Score unter den kleinen Modellen stammt von \texttt{GPT-OSS-20B} mit $0{,}866$, bei den großen Modellen führt \texttt{Qwen3-235B-A22B-\linebreak~Thinking-2507} mit $0{,}874$. Bemerkenswert ist der leicht höhere durchschnittliche Recall der kleinen Modelle von $0{,}843$ gegenüber den großen mit $0{,}839$, wohingegen Precision und Accuracy annähernd identisch sind.

\begin{table}[htbp]
 \centering
 \caption{Kleine vs. große Modelle: Durchschnittswerte pro Gruppe und jeweils bestes Modell.}
 \label{tab:small-vs-large}
 \begin{adjustbox}{width=\textwidth}
  \begin{threeparttable}[width=\textwidth]
   \begin{tabular}[width=\textwidth]{l r r}
    \toprule
    \textbf{Metrik} & \textbf{Klein} ($\leq 25B$) & \textbf{Groß} ($> 25B$) \\
    \midrule
    Anzahl Modelle\tnote{1}             & 5                         & 8 \\
    Ø F1-Score $\pm$ SD\tnote{2}      & 0,805 $\pm$ 0,057                     & 0,806 $\pm$ 0,089 \\
    Ø Precision $\pm$ SD    & 0,774 $\pm$ 0,050                     & 0,779 $\pm$ 0,085 \\
    Ø Recall $\pm$ SD       & 0,843 $\pm$ 0,086                     & 0,839 $\pm$ 0,128 \\
    Ø Accuracy $\pm$ SD     & 0,744 $\pm$ 0,067                     & 0,749 $\pm$ 0,099 \\
    Bester F1-Score & 0,866                     & 0,874 \\
    Bestes Modell (F1-Score)   & GPT-OSS-20B               & Qwen3-235B-A22B-Thinking-2507 \\
    Bester Precision & 0,829                     & 0,892 \\
    Bestes Modell (Precision) & DeepSeek-R1-Distill-Qwen-14B        & GPT-4o \\
    Bester Recall & 0,918                     & 0,932 \\
    Bestes Modell (Recall)      & GPT-OSS-20B      & Qwen3-235B-A22B-Thinking-2507 \\
    Beste Accuracy & 0,821                     & 0,830 \\
    Bestes Modell (Accuracy)     & GPT-OSS-20B               & Qwen3-235B-A22B-Thinking-2507 \\
    \bottomrule
   \end{tabular}
   \begin{tablenotes}
    \footnotesize
    \item[1] Einteilung nach gesamten Milliarden Parametern bei \ac{MoE}. Die Proprietären Modelle \texttt{GPT-4o} und \texttt{Mistral Medium 3.1} wurden trotz fehlender Parameterangabe als große Modelle eingeordnet.
    \item[2] Ohne \texttt{Mixtral-8x7B-Instruct-v0.1} beträgt der Durchschnitt der großen Modelle $\pm$ SD $0.836 \pm 0.029$.
   \end{tablenotes}
  \end{threeparttable}
 \end{adjustbox}
\end{table}

Diese Ergebnisse bestätigen, dass die Modellgröße allein kein Garant für eine bessere Klassifikationsleistung ist. Kleinere Modelle wie \texttt{GPT-OSS-20B} liefern sehr starke Screening-Leistung bei geringeren Kosten und lassen sich leichter On-\linebreak~Premises\footnote{
On-Premises bezeichnet den Betrieb von IT-Systemen im eigenen Rechenzentrum statt in der Cloud.
} betreiben. In der Praxis sollte daher das Auswahlkriterium für ein Modell die \emph{Balance aus Recall, Precision und Betriebskosten} sein, nicht die Parameterzahl.

\subsection*{Europäische vs. internationale Modelle}

Tabelle \ref{tab:eu-vs-international} stellt die Mittelwerte der europäischen Mistral-Modelle den übrigen internationalen Modellen gegenüber. Die europäischen Modelle zeigen eine größere Streuung: das kommerzielle \texttt{Mistral Medium 3.1} erfüllt mit einem F1-Score von $0{,}843$ und Recall von $0{,}877$ alle Zielkriterien und liegt knapp vor dem Referenzmodell \texttt{GPT-4o}. Ähnlich sieht es bei dem Open-Weight-Modell \texttt{Mistral-Large-\linebreak~Instruct-2411} aus. Dagegen verfehlen \texttt{Mistral-7B-Instruct-v0.3} und insbesondere \texttt{Mixtral-8x7B-Instruct-v0.1} die Qualitätsziele deutlich. Im Durchschnitt bleiben die europäischen Modelle hinter den internationalen Spitzenreitern zurück. Letztere – allen voran \texttt{Qwen3-235B-A22B-Thinking-2507} mit einem F1-Score von $0{,}874$ und \texttt{GPT-OSS-20B} mit $0{,}866$ – erreichen im Mittel einen höheren F1-Score sowie Recall und weisen eine geringere Varianz auf.

\begin{table}[htbp]
 \centering
 \caption{Europäische vs. internationale Modelle: Durchschnittswerte pro Gruppe und jeweils bestes Modell.}
 \label{tab:eu-vs-international}
 \begin{adjustbox}{width=\textwidth}
  \begin{threeparttable}[width=\textwidth]
   \begin{tabular}[width=\textwidth]{l r r}
    \toprule
    \textbf{Metrik} & \textbf{\ac{EU}-Modelle} & \textbf{Internationale Modelle} \\
    \midrule
    Anzahl Modelle              & 4                           & 9 \\
    Ø F1-Score $\pm$ SD         & 0{,}760 $\pm$ 0{,}098       & 0{,}826 $\pm$ 0{,}045 \\
    Ø Precision $\pm$ SD        & 0{,}738 $\pm$ 0{,}061       & 0{,}797 $\pm$ 0{,}056 \\
    Ø Recall $\pm$ SD           & 0{,}789 $\pm$ 0{,}138       & 0{,}864 $\pm$ 0{,}076 \\
    Ø Accuracy $\pm$ SD         & 0{,}694 $\pm$ 0{,}101       & 0{,}771 $\pm$ 0{,}057 \\
    Bester F1-Score             & 0{,}843                     & 0{,}874 \\
    Bestes Modell (F1-Score)    & Mistral Medium 3.1          & Qwen3-235B-A22B-Thinking-2507 \\
    Bester Precision            & 0{,}811                     & 0{,}892 \\
    Bestes Modell (Precision)   & Mistral Medium 3.1          & GPT-4o \\
    Bester Recall               & 0{,}877                     & 0{,}932 \\
    Bestes Modell (Recall)      & Mistral Medium 3.1          & Qwen3-235B-A22B-Thinking-2507 \\
    Beste Accuracy              & 0{,}794                     & 0{,}830 \\
    Bestes Modell (Accuracy)    & Mistral Medium 3.1          & Qwen3-235B-A22B-Thinking-2507 \\
    \bottomrule
   \end{tabular}
   \begin{tablenotes}
    \footnotesize
    \item Die \ac{EU}‑Modelle umfassen \texttt{Mistral‑7B‑Instruct‑v0.3}, \texttt{Mixtral‑8x7B‑Instruct‑v0.1}, \texttt{Mistral‑Large‑Instruct‑2411} und \texttt{Mistral Medium 3.1}. Die internationalen Modelle sind die übrigen in Kapitel \ref{sec:ueberblick} betrachteten Modelle.
   \end{tablenotes}
  \end{threeparttable}
 \end{adjustbox}
\end{table}

Diese Vergleiche belegen, dass ein europäischer Ursprung nicht zwangsläufig mit einer geringeren Leistung einhergeht – \texttt{Mistral Medium 3.1} erreicht gute Werte, dicht gefolgt von \texttt{Mistral-Large-Instruct-2411}. Allerdings zeigen die Ergebnisse auch, dass einige europäische Modelle hinter den internationalen Konkurrenten zurückbleiben. Insgesamt sind die internationalen Modelle im Durchschnitt leistungsfähiger und stabiler, da sie die europäischen Modelle in jeder Metrik im Durchschnitt übertreffen und eine geringere Varianz aufweisen. Zudem ist in jeder Metrik ein internationales Modell führend.
\section{Beantwortung der Forschungsfragen}\label{sec:antworten-auf-forschungsfragen}

Auf Basis der vorhergehenden Auswertungen lassen sich die in Abschnitt~\ref{sec:zielsetzung-und-beitrage} formulierten Forschungsfragen beantworten. Die folgenden Antworten berücksichtigen sowohl die quantitativen Ergebnisse als auch die qualitative Beobachtungen aus den Fallstudien und ordnen sie unter Berücksichtigung der Qualitätsziele ein.

\paragraph{UF1: Wie gut schneiden europäische Modelle im Vergleich zu internationalen Modellen ab?}

Die europäischen Modelle zeigen eine große Bandbreite in ihrer Leistungsfähigkeit. \texttt{Mistral Medium 3.1} erfüllt mit einem F1-Score $= 0{,}843$, einem Recall $= 0{,}877$ und einer Precision $= 0{,}811$ sämtliche Qualitätsziele und übertrifft das Referenzmodell \texttt{GPT-4o}. \texttt{Mistral-Large-Instruct-2411} erreicht mit einem F1-Score $= 0{,}823$ ebenfalls alle Zielwerte. Dagegen schneiden \texttt{Mistral-\linebreak~7B-Instruct-v0.3} mit F1-Score $= 0{,}777$ und insbesondere \texttt{Mixtral-8x7B-\linebreak~Instruct-v0.1} mit F1-Score $= 0{,}596$ deutlich schlechter ab. Im Durchschnitt liegen die internationalen Modelle - insbesondere die Qwen- und GPT-OSS-Varianten - vor den europäischen und bieten eine robustere Balance aus Recall und Precision. Dennoch zeigen \texttt{Mistral Medium 3.1} und \texttt{Mistral-Large-Instruct-\linebreak~2411}, dass leistungsfähige europäische Alternativen existieren.

\paragraph{UF2: Wie unterscheiden sich große und kleine Modelle in ihrer Leistungsfähigkeit?}

Der direkte Vergleich zeigt, dass sich kleine ($\leq 25$B Parameter) und große Modelle kaum im Durchschnitt ihrer Metriken unterscheiden. Beide Größenklassen erreichen praktisch identische mittlere F1-Scores von etwa $0{,}80$. Ohne das Ausreißermodell \texttt{Mixtral-8x7B-Instruct-v0.1} liegt der Durchschnitt der großen Modelle mit $0{,}836$ zwar etwas höher, doch belegen Modelle wie \texttt{GPT-OSS-\linebreak~20B}, dass kleinere Modelle mit den großen mithalten können. Entscheidend sind Trainingsdaten, Feinabstimmung und Architektur, nicht allein die Parameteranzahl.

\paragraph{UF3: Welche Open-Source-Modelle (insbesondere aus der EU) erzielen die besten Ergebnisse?}

Unter den offenen Modellen dominieren die chinesischen Qwen-Varianten und die GPT-OSS-Modelle. \texttt{Qwen3-235B-A22B-Thinking-2507} erreicht mit einem F1-Score $= 0{,}874$ und einem Recall $= 0{,}932$ die Spitzenposition, gefolgt von \texttt{GPT-OSS-20B} mit F1-Score $= 0{,}866$ und Recall $= 0{,}918$ und \texttt{DeepSeek-R1-Distill-Qwen-14B} mit F1-Score $= 0{,}848$ und Precision $= 0{,}829$. Diese Modelle übertreffen die proprietären Benchmarks deutlich. Das leistungsstärkste offene \ac{EU}-Modell ist \texttt{Mistral-Large-Instruct-2411} mit F1-Score $= 0{,}823$, während \texttt{Mistral-7B-Instruct-v0.3} und \texttt{Mixtral-8x7B-Instruct-\linebreak~v0.1} die Zielwerte verfehlen.

\paragraph{UF4: Wie gut schneiden Open-Source-Modelle gegenüber kommerziellen Modellen wie GPT-4o ab?}

Mehrere Open-Source-Modelle übertreffen die kommerziellen Vertreter. \texttt{Qwen3-235B-A22B-Thinking-2507}, \texttt{GPT-OSS-20B} und\linebreak\texttt{DeepSeek-R1-Distill-Qwen-14B} erreichen höhere F1- und Recall-Werte als sowohl \texttt{GPT-4o} als auch \texttt{Mistral Medium 3.1}. \texttt{GPT-4o} überzeugt mit einer außergewöhnlich hohen Precision von $0{,}892$, verfehlt aber das Recall-Mindestziel. \texttt{Mistral Medium 3.1} bietet einen ausgewogenen Kompromiss und erfüllt alle Zielwerte, liegt aber hinter den besten Open-Source-Modellen. Insgesamt zeigen hochwertige Open-Source-Modelle die beste Balance zwischen hohem Recall und akzeptabler Precision.

Auf Basis der durchgeführten Experimente, Analysen und Antworten auf die Unterfragen lässt sich die Hauptforschungsfrage im Folgenden beantworten.

\paragraph{FF1: Wie zuverlässig identifizieren \acp{LLM} DSGVO-kritische Aktivitäten in\linebreak~BPMN-Prozessmodellen?}

Die überwiegende Mehrheit der Modelle kann kritische Aktivitäten mit hoher Zuverlässigkeit erkennen. Neun von dreizehn Modellen erreichen einen F1-Score von mindestens $0{,}80$ und erfüllen damit den Zielwert. Die Spitzenmodelle \texttt{Qwen3-235B-A22B-Thinking-2507}, \texttt{GPT-OSS-20B}, \texttt{DeepSeek-\linebreak~R1-Distill-Qwen-14B} und \texttt{Mistral Medium 3.1} erzielen F1-Scores zwischen\linebreak$0{,}843$ und $0{,}874$ bei Recall-Werten von $0{,}868$ bis $0{,}932$. Gleichzeitig gibt es Modelle wie \texttt{Mixtral-8x7B-Instruct-v0.1} und \texttt{Qwen2.5-7B-Instruct}, die deutlich abfallen.

Die Robustheitsanalyse zeigt, dass die meisten Modelle, mit einer Standardabweichung der F1-Scores von $\leq 0{,}02$, eine geringe Varianz über verschiedene Seeds aufweisen und häufig keine Retries benötigen, um eine korrekte JSON-Ausgabe zu produzieren. Ausreißer wie \texttt{Mistral Medium 3.1} (höhere Varianz) oder \texttt{Mistral-\linebreak~7B-Instruct-v0.3} (viele Retries) sollten im praktischen Einsatz sorgfältig überprüft werden.

Die Fallstudien unterstreichen, dass \ac{FP} vor allem dann entstehen, wenn im Prozessmodell wichtige Kontextinformationen fehlen, wie z.\,B.\ Anonymisierung von\linebreak~Klickraten, und Modelle daher konservativ entscheiden. \ac{FN} treten auf, wenn Datenflüsse über mehrere Aktivitäten nicht korrekt erkannt werden. Trotz dieser Fehlerbilder zeigen die Experimente, dass \acp{LLM} für ein automatisiertes Screening von \ac{BPMN}-Prozessen sehr gut geeignet sind. Eine nachgelagerte menschliche Prüfung bleibt jedoch sinnvoll, um verbleibende \ac{FP} und \ac{FN} zu adressieren und die Ergebnisse kontextsensitiv zu bewerten.
\section{Fallstudien}\label{sec:fallstudien}

Neben den aggregierten Metriken aus den Ergebnissen bieten einzelne Testfälle wichtige Einblicke in die Stärken und Schwächen der Modelle. Im Folgenden werden drei exemplarische Szenarien vorgestellt, die jeweils typische Fehlklassifikationen illustrieren: ...

\subsection*{Sales Warehouse}

Bei dem Testfall \enquote{Sales Warehouse} handelt es sich um einen englischen Prozess aus dem Testdatensatz \enquote{Universität}. Der Prozess ist in Abbildung \ref{fig:qwen3-fall} zu sehen. Im Testfall sind vier Aktivitäten als kritisch markiert. Das Modell \texttt{Qwen3-235B-A22B-\linebreak~Thinking-2507} erkennt alle vier korrekt, markiert jedoch zusätzlich die Aktivität \enquote{Ship product} als kritisch. Die manuell festgelegten Labels ordnen das Versenden eines Produkts als unkritisch ein, da Logistikvorgänge in der Regel ohne Verarbeitung personenbezogener Daten erfolgen (vgl. Tabelle \ref{tab:labeling-examples}). \texttt{Qwen3-235B-A22B-\linebreak~Thinking-2507} begründet die Entscheidung mit der Nutzung der Kundenadresse zum Versand und zur Zustellung. Diese Begründung zeigt, dass das Modell mögliche Datenflüsse im Hintergrund berücksichtigt und daher zu einer vorsichtigeren Klassifikation gelangt. Angesichts der hohen Strafen bei übersehenen Datenschutzverstößen und des angestrebten hohen Recalls kann dieses \ac{FP} als vertretbar gelten.

\begin{figure}
    \centering
    \includegraphics[height=.41\textheight]{images/results/examples/qwen3-235B-run-3-uni-sales-warehouse}
    \caption{Ergebnis des Testfalls \enquote{Sales Warehouse} mit farblich hervorgehobenen Aktivitäten. Grün markierte Aktivitäten sind korrekt als kritisch erkannt, rot markierte stellen \acp{FP} dar.}
    \label{fig:qwen3-fall}
\end{figure}

Das Beispiel verdeutlicht eine grundsätzliche Limitierung der Klassifizierung: Fehlen in einem \ac{BPMN}-Modell explizite Informationen über Verarbeitungsschritte, ist es für das System schwierig, eine eindeutige Klassifikation vorzunehmen.

\subsection*{Marketing-Kampagne}

Im deutschen Testfall \enquote{Marketing-Kampagne}, aus dem Testdatensatz \enquote{Kleine Szenarien}, sind drei Aktivitäten als kritisch gelabelt: \enquote{Leads sammeln}, \enquote{Newsletter versenden} und \enquote{CRM aktualisieren}. \texttt{GPT‑OSS‑20B} identifiziert diese korrekt, markiert aber zusätzlich die Aktivität \enquote{Klickraten auswerten} als kritisch. Die Prozessmodellierung sah vor, dass die Klickdaten komplett anonymisiert werden und daher keine personenbezogenen Daten verarbeitet werden. Da diese Information im \ac{BPMN}-Diagramm jedoch nicht explizit hinterlegt ist, stuft das Modell die Analyse der Klickraten als potenziell personenbezogen ein und führt als Begründung die Nutzung der E‑Mail‑Adresse an. \texttt{Qwen3-235B-A22B-Thinking-2507} und einige weitere Modelle bewerteten diesen Schritt ebenfalls als kritisch, während \texttt{Mistral‑7B‑\linebreak~Instruct‑v0.3} in zwei von fünf Wiederholungen und die Gemma‑Modelle in keiner der Wiederholungen eine kritische Klassifikation vornahmen. Der Prozess inklusive farblich hervorgehobener Aktivitäten ist in Abbildung \ref{fig:gptoss-fall} zu sehen.

Dieses Beispiel zeigt, dass ohne genaue Kontextangaben zur Anonymisierung selbst scheinbar unbedenkliche Auswertungen als datenschutzrelevant erscheinen können. Es unterstreicht, dass die \acp{LLM} im Zweifel eher ein kritisches Label vergeben, um \acp{FN} zu vermeiden, wie es das Hauptziel der Klassifikation aus Abschnitt \ref{sec:qualitatsziele} vorsieht.

\begin{figure}
    \centering
    \includegraphics[width=\textwidth]{images/results/examples/oss-20b-run-1-small-marketing}
    \caption{Ergebnis des Testfalls \enquote{Marketing-Kampagne} mit farblich hervorgehobenen Aktivitäten. Die Aktivität \enquote{Klickraten auswerten} wurde als zusätzliches kritisches Element markiert.}
\label{fig:gptoss-fall}
\end{figure}

\subsection*{Karten App - Standort Erfassen}

Im Fall \enquote{Karten-App – Standort Erfassen}, ebenfalls aus dem Testdatensatz \enquote{Kleine Szenarien}, treten zwei Aktivitäten auf: \enquote{Standort erfassen} und \enquote{Route berechnen}. Beide sollten als kritisch gekennzeichnet werden, da im zweiten Schritt der zuvor erfasste Benutzerstandort zur Berechnung der Route verwendet wird. \texttt{Mistral-Large} erkennt jedoch in drei von fünf Läufen nur die erste Aktivität als kritisch und die Aktivität \enquote{Route berechnen} wird trotz der Datenassoziation nicht als kritisch eingestuft. Die Begründung des Modells erklärt zwar, dass \enquote{Standort erfassen} personenbezogene Daten verarbeitet, überträgt diese Argumentation aber nicht auf den unmittelbar folgenden Schritt. Dieses \ac{FN} ist problematisch, da es dem gewünschten hohen Recall entgegensteht und dieser Testfall zeigt, dass selbst mit vorhandenen Datenobjekten manche Modelle Schwierigkeiten haben, Datenflüsse über mehrere Aktivitäten hinweg zu erfassen. Es verdeutlicht auch, dass unterschiedliche Seeds zu unterschiedlichen Klassifikationen führen können. Der Prozess inklusive farblich hervorgehobener Aktivitäten ist in Abbildung \ref{fig:mistral-fall} zu sehen.

\begin{figure}
    \centering
    \includegraphics[width=.55\textwidth]{images/results/examples/mistral-large-run-3-small-maps-app}
    \caption{Ergebnis des Testfalls \enquote{Karten-App – Standort Erfassen} mit farblich hervorgehobenen Aktivitäten. Die Aktivität \enquote{Route berechnen} wurde fälschlicherweise nicht als kritisch markiert.}
    \label{fig:mistral-fall}
\end{figure}
\section{Robustheit}\label{sec:robustheit2}

Die Robustheitsanalyse über mehrere Seeds unterstreicht die Praxistauglichkeit der meisten Modelle in Kombination mit der entwickelten Klassifizierungspipeline. Für die Mehrzahl der \acp{LLM} liegt die Standardabweichung des F1-Scores über fünf Wiederholungen bei $\leq 0{,}02$, was auf eine geringe Varianz und konsistente Leistung hinweist. Vereinzelt zeigen Modelle eine höhere Varianz oder benötigen mehr Wiederholungen zur Korrektur von Parsing-Fehlern. Solche Unterschiede sind für den operativen Einsatz relevant, da sie sich direkt in Durchsatz, Latenz und Stabilität der Gesamtpipeline niederschlagen. Modelle mit erhöhter Varianz sollten daher im produktiven Betrieb sorgfältig überwacht und validiert werden, um unerwartete Leistungseinbußen zu vermeiden.

Wesentlich zur Zuverlässigkeit trägt die entwickelte Klassifizierungspipeline bei. \emph{Structured Output} via Langchain4j mit explizitem JSON-Schema (und, wo verfügbar, API-seitig erzwungenem \texttt{response\_format}) erhöht die Format-Treue, ein explizites \texttt{isRelevant}-Flag mit nachgelagertem Relevanz-Filter entschärft Widersprüche zwischen Begründung und Klassifikation, die \emph{id-Validierung/-Vervoll-\linebreak~ständigung} reduziert typische Ausgabefehler und der \emph{Retry-Mechanismus} behebt Parsing-Fehler automatisiert. Diese Maßnahmen tragen wesentlich zur Ergebnisstabilität bei und sollten in produktiven Systemen implementiert werden, um die Zuverlässigkeit der Modellausgaben zu gewährleisten.

Ob das Preprocessing der Klassifizierungspipeline zur Leistung beiträgt, lässt sich nicht abschließend beurteilen. Die im nächsten Abschnitt beschriebenen Fallstudien legen nahe, dass trotz des im Preprocessing bereitgestellten Kontexts Datenflüsse und Prozesszusammenhänge durch \ac{LLM} weiterhin unberücksichtigt bleiben oder falsch interpretiert werden. Hier könnten künftige Anpassungen der Pipeline ansetzen, um den Kontext für die Modelle weiter zu verbessern.
\chapter{Diskussion}\label{ch:diskussion}

\section{Interpretation der Befunde}\label{sec:interpretation-der-befunde}

\begin{itemize}
    \item Einordnung der Rangfolge der LLMs
    \item Besonderheiten der Modellfamilien (Bspw. wie groß ist der Unterschied von Groß gegen Klein?)
    \item Es gab Prozesse in denen ich eine Aktivität nicht als kritisch gelabelt habe, aber das LLM in der Evaluierung das als kritisch mit einer validen Begründung klassifiziert hat, man könnte die Testdaten also noch anpassen, wenn die Begründung des LLMs überzeugt (Ich weiß nicht wie sinnvoll das ist hier zu thematisieren)
\end{itemize}
\section{Hoher Recall vs. Präzision}\label{sec:hoher-recall-vs.-prazision}

\begin{itemize}
    \item Beobachtung ausformulieren, dass einige Testfälle als fehlerhaft eingeordnet wurden, weil es False-Positives gab, obwohl es keine False-Negatives gab. Es wurden also alle Aktivitäten gefunden, die gefunden werden sollten, nur halt noch mehr on top.
    \item Einordnung der FP-Last pro Prozess (Lieber False Positives, als dass etwas übersehen wird, Ziel war sowieso ein Vorscreening), Diskussion darüber wie nützlich hohe Recall Werte sind
\end{itemize}
\section{EU-Modelle}\label{sec:eu-modelle}

\begin{itemize}
    \item Analyse der EU-Open-Source-Modelle in Bezug auf Precision, Recall und Stabilität in Bezug auf die anderen Modelle
    \item Wie gut haben sich die EU Modelle im Vergleich zu den anderen geschlagen
\end{itemize}
\section{Open-Source Modelle}\label{sec:open-source-modelle}

\begin{itemize}
    \item Analyse der Open-Source-Modelle in Bezug auf Precision, Recall und Stabilität in Bezug auf kommerzielle Modelle
    \item Wie gut haben sich die Open-Source-Modelle im Vergleich zu den anderen geschlagen
\end{itemize}
\section{Modellgrößen}\label{sec:modellgroen}

\begin{itemize}
    \item Selbst hosten von Modellen diskutieren. Ist es realistisch die Modelle selbst zu hosten, welche gut performt haben? Reichen die kleinen Varianten der Modelle oder muss man schon die großen Modelle benutzen, um gute Ergebnisse zu erzielen
\end{itemize}
\section{Grenzen}\label{sec:grenzen}

\begin{itemize}
    \item Wären Grenzen wie BPMN-Modellgröße im Zusammenhang mit der Kontextlänge des LLM interessant?
    \item Keine aussagekräftige Rechtsberatung, sondern stand jetzt eher ein Vorscreening, was nochmal überprüft werden muss
    \item ggf. notwendige Anonymisierung von Prozessen diskutieren (Wenn das in BPMN Modellen überhaupt ein Problem ist)
\end{itemize}
\chapter{Zusammenfassung}\label{ch:zusammenfassung}

Hier neben der allgemeinen Zusammenfassung unbedingt noch die erste Forschungsfrage explizit beantworten
\chapter{Aussicht}

Unter anderem das hier, evtl noch mehr:
\begin{itemize}
    \item Jetzt gibt es ein einheitliches Evaluationsframework mit einer einheitlich definierten Schnittstelle für Klassifizierungsalgorithmen -> Zukünftige Arbeiten können sich mit der Entwicklung besserer Klassifizierungsalgorithmen/Pipelines (Bspw. noch RAG einbauen) beschäftigen und diese mit diesem Framework vergleichen/benchmarken
    \item Außerdem können in Zukunft auch noch mehr Modelle vergleichen werden, da sich die Welt der LLMs rasant weiterentwickelt
    \item Auch Finetunen ist etwas was interessant gewesen wäre für diese Masterarbeit, aber den Rahmen gesprengt hätte
\end{itemize}


\appendix
% hier Anhänge einbinden
\chapter{Quelltexte}\label{ch:quelltexte}

In diesem Anhang sind mehrere Quellcode-Ausschnitte aufgeführt.

\begin{lstlisting}[caption={System-Prompt fuer die DSGVO-Klassifikation von BPMN-Aktivitäten},label={lst:system-prompt}]
You are an expert in analysing Business Process Model and Notation (BPMN) diagrams for GDPR compliance. Your task is to identify and return a list of the IDs of all Activity (Task) elements that process personal data. Ignore all other element types. Always consider every activity in the process; do not omit any activity from your assessment.

Use all available context for each activity - including the activity's name, description, annotations, associated data objects, and message or data associations - to determine whether the activity processes personal data. Under Article 4 of the GDPR, personal data is any information relating to an identified or identifiable natural person, including names, addresses, email addresses, phone numbers, identification numbers, payment or bank details, employment records, academic records, location data, IP addresses, online identifiers, images, audio/video recordings, biometric identifiers, health data or other information that can be linked to a specific person. "Processing" includes any operation performed on personal data, such as collecting, recording, organising, structuring, storing, retrieving, consulting, using, analysing, transmitting, printing, disseminating, aligning, combining, altering, restricting, erasing or destroying the data.

Classify an activity as GDPR-relevant whenever it performs or enables processing of personal data. Indicators include (but are not limited to):

- **Collection and entry of personal data**: Activities that collect or capture personal information, for example entering contact details, addresses, payment information, job applications, health information, student enrolments, membership data, tax declarations, registration forms or other forms with personally identifiable information.
- **Creation, storage and updating of records**: Activities that create, save or update records containing personal data, such as opening customer accounts, storing order or appointment details, creating personnel files, enrolling students, setting up insurance cases or filing a medical record.
- **Transmission or disclosure of personal data**: Activities that send, print or otherwise disclose personal data to another participant, system or third party. Examples include printing shipping labels or prescriptions, sending orders or personal data to logistics partners, pharmacies, insurers or authorities, generating payroll reports for external providers, notifying universities about student records, transmitting tax or social security data, sending confirmations or queries that rely on a person's contact details, or transferring data to non-EU locations.
- **Payments and financial transactions**: Activities that process personal financial data, such as initiating or verifying payments, processing bank account or credit-card information, executing payroll, handling reimbursements or insurance payouts, managing expense claims or collecting membership fees.
- **Use of health, biometric or other special categories of data**: Activities that handle medical diagnoses, prescriptions, insurance claims, disability information, photos of damages or patients, biometric identifiers (fingerprints, facial images, voice), racial or ethnic data, political opinions, religious beliefs or union membership. Processing these "special categories" always triggers GDPR relevance.
- **Audio/Video and communications**: Activities that initiate or join audio or video calls, record calls or meetings, capture surveillance footage, or communicate directly with a data subject via email, chat, SMS or other channels. Simply using a person's contact data to send reminders, marketing messages or notifications is processing.
- **Profiling, scoring and decision-making**: Activities that analyse or evaluate a person's performance, behaviour or characteristics for purposes such as credit scoring, hiring, admissions, insurance underwriting, marketing segmentation, customer value analysis or automated decision-making.
- **Logging, tracking and location data**: Activities that log user activity, record access or usage data, track geolocation (e.g. telematics, fleet or mobile tracking), monitor attendance or timekeeping, or collect IP addresses or device identifiers.
- **Consent and data-subject rights**: Activities that obtain, record or manage consent; respond to requests for access, rectification, restriction, erasure, data portability or objections; or document lawful bases for processing.
- **Deletion, anonymisation or pseudonymisation**: Activities that erase, anonymise or pseudonymise personal data, even if the goal is to remove identifiers, because these operations manipulate personal data.

When assessing an activity, consider synonyms or domain-specific terms: activities referring to customers, patients, applicants, employees, students, voters, taxpayers, residents or members often imply personal data processing, even if names like "address" or "contact" are absent. Use context - data objects, annotations or typical process semantics - to infer personal data involvement. Do not rely solely on explicit data-object links; many process names ("Anmeldung pruefen", "Aufnahmeantrag bearbeiten", "Kundeninfo aktualisieren", "Registrierung bestaetigen", "Kreditwuerdigkeit berechnen") themselves indicate personal data processing.

Do **not** classify an activity as GDPR-relevant when it only performs administrative or logistic tasks that do not involve personal data. Examples include picking or packing goods, routing vehicles without using specific addresses, printing generic pick lists, moving items in inventory, or checking if a document exists without viewing its contents. Likewise, activities using truly aggregated or irreversibly anonymised data can be ignored if no individual can be reidentified.

In your output, return only the IDs of activities you classify as GDPR-relevant. For each, provide a clear explanation in englisch using the activity’s name and description to justify why it processes personal data. Do not reference element IDs in your explanation; use the activity names instead. Exclude from your result any activities that do not process personal data and any elements that are not activity/task elements.
\end{lstlisting}

\begin{lstlisting}[language=Kotlin,caption={Antworttyp fuer die Klassifizierung},label={lst:bpmn-analysis-result}]
data class BpmnAnalysisResult(
    @Description("List of Activity Elements that are classified as relevant for GDPR compliance")
    var elements: List<Element>
) {

    init {
        elements = elements.filter { it.isRelevant }
    }

    @Description("Represents an Activity/Task Element that is classified as relevant for GDPR compliance")
    data class Element(
        @Description("The ID of the Activity Element")
        val id: String,
        @Description("The detailed reason why the Activity Element is relevant for GDPR compliance and why you think personal data is processed.")
        val reason: String,
        @Description("Indicates whether the Activity Element is relevant for GDPR compliance")
        val isRelevant: Boolean = true
    )

    /* Andere Methoden dieser Klasse sind weggelassen */
}
\end{lstlisting}

\begin{lstlisting}[language=Kotlin,caption={Kern der \texttt{id}-Validierung und -Vervollständigung},label={lst:activity-id-resolution}]
fun resolveActivityIds(actualBpmnElements: Set<BpmnElement>): BpmnAnalysisResult {
    val existingActivityIds = actualBpmnElements
        .filter { it.type.lowercase().contains("task") }
        .map { it.id }.toSet()

    val resolvedDistinct = elements.mapNotNull { element ->
        val resolvedId = resolveActivityIdUniquely(element.id, existingActivityIds)
        resolvedId?.let { if (it == element.id) element else element.copy(id = it) }
    }.distinctBy { it.id }

    return BpmnAnalysisResult(elements = resolvedDistinct)
}

private fun resolveActivityIdUniquely(partialId: String, existingActivityIds: Set<String>): String? {
    if (partialId in existingActivityIds) return partialId
    existingActivityIds.filter { it.startsWith(partialId) }.singleOrNull()?.let { return it }
    return existingActivityIds.filter { it.contains(partialId) }.singleOrNull()
}
\end{lstlisting}

\begin{lstlisting}[caption={Schema der YAML-Evaluationskonfiguration},label={lst:evaluation-config-schema}]
{
    "$schema": "https://json-schema.org/draft/2020-12/schema",
    "$ref": "#/definitions/Configuration",
    "definitions": {
        "Configuration": {
            "type": "object",
            "additionalProperties": false,
            "properties": {
                "defaultEvaluationEndpoint": {
                    "type": "string"
                },
                "maxConcurrent": { "type": "integer" },
                "repititions": { "type": "integer" },
                "models": {
                    "type": "array",
                    "items": { "$ref": "#/definitions/Model" }
                },
                "datasets": {
                    "type": "array",
                    "items": { "type": "integer" }
                },
                "seed": { "type": "integer" }
            },
            "required": [
                "defaultEvaluationEndpoint",
                "models"
                "datasets",
            ],
            "title": "Configuration"
        },
        "Model": {
            "type": "object",
            "additionalProperties": false,
            "properties": {
                "label": { "type": "string" },
                "evaluationEndpoint": { "type": "string" },
                "llmProps": { "$ref": "#/definitions/LlmProps" }
            },
            "required": [ "label" ],
            "title": "Model"
        },
        "LlmProps": {
            "type": "object",
            "additionalProperties": false,
            "properties": {
                "baseUrl": {
                    "type": "string",
                    "format": "uri",
                    "qt-uri-protocols": [ "https" ]
                },
                "modelName": { "type": "string" },
                "apiKey": { "type": "string"},
                "timeoutSeconds": { "type": "number" },
                "temperature": { "type": "number" },
                "topP": { "type": "number" },
            },
            "required": [],
            "title": "LlmProps"
        }
    }
}
\end{lstlisting}

\begin{lstlisting}[caption={Zusammengefasster Logauszug zum Retry-Mechanismus}, label={lst:retry-log}]
2025-10-03T19:11:51.152+02:00  INFO  BpmnExtractor  : Extracting BPMN elements from XML

# 1) Erste Anfrage an das LLM (gekuerzt: Prompt/Headers/Body)
2025-10-03T19:11:51.156+02:00  INFO  LoggingHttpClient  : HTTP POST https://openrouter.ai/api/v1/chat/completions
  model: openai/gpt-oss-20b
  messages: [system: (System-Prompt), user: (User-Prompt mit BpmnElement-Liste und Format-Anweisung)]

# 2) Antwort des LLM mit fehlerhaftem JSON (verkuerzt)
2025-10-03T19:11:56.671+02:00  INFO  LoggingHttpClient  : HTTP 200
assistant:
{
  "elements": [
    { "id": "Activity_09ehuii", "reason": "...", "isRelevant": true },
    { "id": "Activity_1la5hsp", "reason": "...", "isRelevant":  }   <-- fehlender Bool-Wert
    { "id": "Activity_0rfgrlm", "reason": "...", "isRelevant": true }
  ]
}

# 3) Parser-Fehler + Retry-Ankuendigung (gekuerzt)
2025-10-03T19:11:56.691+02:00  WARN  SafetyNet  : Parsing failed. Attempting to fix JSON and retry... (Attempt 1 of 2)
dev.langchain4j.service.output.OutputParsingException:
  Caused by: com.fasterxml.jackson.core.JsonParseException:
  Unexpected character ('}') ... at elements[1].isRelevant

# 4) Zweite Anfrage zum beheben des JSON mit Chat-Verlauf und Fehlermeldung (n-mal wiederholt, bis erfolgreich)
2025-10-03T19:11:56.721+02:00  INFO  LoggingHttpClient  : HTTP POST https://openrouter.ai/api/v1/chat/completions
  messages: [
    system: (System-Prompt),
    user: (User-Prompt mit BpmnElement-Liste und Format-Anweisung),
    assistant: (Fehlerhafte JSON-Antwort),
    system: (Fix-JSON System-Prompt),
    user: (Fehlermeldung)
  ]

# 5) Korrigierte JSON-Antwort des LLM
2025-10-03T19:12:01.519+02:00  INFO  LoggingHttpClient  : HTTP 200
assistant:
{
  "elements": [
    { "id": "Activity_09ehuii", "reason": "...", "isRelevant": true },
    { "id": "Activity_1la5hsp", "reason": "...", "isRelevant": true }, <-- jetzt mit Bool-Wert
    { "id": "Activity_0rfgrlm", "reason": "...", "isRelevant": true }
  ]
}

# 6) Erfolgreiches Parsing und Weiterverarbeitung
2025-10-03T19:12:01.519+02:00  INFO  PromptBpmnAnalyzer : BPMN Analysis Result: elements=[... isRelevant=true ...]
\end{lstlisting}

\begin{lstlisting}[caption={Konfigurationsdatei des Experiments mit Gemma Modellen}, label={lst:gemma-experiment-config}]
defaultEvaluationEndpoint: /gdpr/analysis/prompt-engineering
seed: 24523833
maxConcurrent: 10
repetitions: 5
models:
  - label: Gemma-3-12B-it
    llmProps:
      baseUrl: https://openrouter.ai/api/v1
      modelName: google/gemma-3-12b-it
      apiKey: ${OPEN_ROUTER_API_KEY}
  - label: Gemma-3-27B-it
    llmProps:
      baseUrl: https://openrouter.ai/api/v1
      modelName: google/gemma-3-27b-it
      apiKey: ${OPEN_ROUTER_API_KEY}
datasets:
  - 2
  - 7
  - 1
\end{lstlisting}

\begin{lstlisting}[caption={Konfigurationsdatei des Experiments mit DeepSeek Modellen}, label={lst:deepseek-experiment-config}]
defaultEvaluationEndpoint: /gdpr/analysis/prompt-engineering
seed: 24523833
maxConcurrent: 10
repetitions: 5
models:
  - label: DeepSeek-V3.1
    llmProps:
      baseUrl: https://openrouter.ai/api/v1
      modelName: deepseek/deepseek-chat-v3.1
      apiKey: ${OPEN_ROUTER_API_KEY}
  - label: DeepSeek-R1-Distill-Qwen-14B
    llmProps:
      baseUrl: https://openrouter.ai/api/v1
      modelName: deepseek/deepseek-r1-distill-qwen-14b
      apiKey: ${OPEN_ROUTER_API_KEY}
datasets:
  - 2
  - 7
  - 1
\end{lstlisting}

\begin{lstlisting}[caption={Konfigurationsdatei des Experiments mit Qwen Modellen}, label={lst:qwen-experiment-config}]
defaultEvaluationEndpoint: /gdpr/analysis/prompt-engineering
seed: 24523833
maxConcurrent: 10
repetitions: 5
models:
  - label: Qwen2.5-7B-Instruct
    llmProps:
      baseUrl: https://openrouter.ai/api/v1
      modelName: qwen/qwen-2.5-7b-instruct
      apiKey: ${OPEN_ROUTER_API_KEY}
  - label: Qwen3-235B-A22B-Thinking-2507
    llmProps:
      baseUrl: https://openrouter.ai/api/v1
      modelName: qwen/qwen3-vl-235b-a22b-thinking
      apiKey: ${OPEN_ROUTER_API_KEY}
datasets:
  - 2
  - 7
  - 1
\end{lstlisting}

\begin{lstlisting}[caption={Konfigurationsdatei des Experiments mit GPT Modellen}, label={lst:gpt-experiment-config}]
defaultEvaluationEndpoint: /gdpr/analysis/prompt-engineering
seed: 24523833
maxConcurrent: 10
repetitions: 5
models:
  - label: GPT-OSS-20B
    llmProps:
      baseUrl: https://openrouter.ai/api/v1
      modelName: openai/gpt-oss-20b
      apiKey: ${OPEN_ROUTER_API_KEY}
  - label: GPT-OSS-120B
    llmProps:
      baseUrl: https://openrouter.ai/api/v1
      modelName: openai/gpt-oss-120b
      apiKey: ${OPEN_ROUTER_API_KEY}
  - label: GPT-4o (2024-11-20)
    llmProps:
      baseUrl: https://openrouter.ai/api/v1
      modelName: openai/gpt-4o-2024-11-20
      apiKey: ${OPEN_ROUTER_API_KEY}
datasets:
  - 2
  - 7
  - 1
\end{lstlisting}

\listoffigures
\lstlistoflistings
\listoftables

\backmatter
\nocite{Knappen2009}
\nocite{Mittelbach2005}
\nocite{Schlosser2014}
\nocite{Sturm2012}
\nocite{Voss2010}

\setcounter{biburllcpenalty}{7000}
\setcounter{biburlucpenalty}{8000}
\printbibliography

\clearpage
\thispagestyle{empty}

Name: \fullname \hfill Matrikelnummer: \matnr \vspace{2cm}

\minisec{Erklärung}

Ich erkläre, dass ich die Arbeit selbständig verfasst und keine anderen als die angegebenen Quellen und Hilfsmittel verwendet habe.\vspace{2cm}

Ulm, den \dotfill

\hspace{10cm} {\footnotesize \fullname}
\end{document}
