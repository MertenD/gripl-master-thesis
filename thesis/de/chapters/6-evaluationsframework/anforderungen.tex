\section{Use-Cases und Anforderungen}\label{sec:anforderungen-und-use-cases}

Das Evaluationsframework richtet sich an Forschende und Entwickler, die \acp{LLM} und Klassifizierungsalgorithmen für die Identifikation \ac{DSGVO}-kritischer \ac{BPMN}-\linebreak~Aktivitäten auswerten und miteinander vergleichen möchten. Es bietet eine einheitliche Ausführungs- und Auswertungsumgebung mit klar definierten Schnittstellen und standardisierten Berichten. In diesem Kapitel werden die Use-Cases und funktionalen Anforderungen des Evaluationsframeworks beschrieben.

\subsection*{Use-Cases}

Die wichtigsten Anwendungsfälle des Evaluationsframeworks sind:

\begin{itemize}
    \item \textbf{Benchmarking von \acp{LLM}.} Systematischer Vergleich mehrerer \acp{LLM} auf denselben Datensätzen, mit identischem Algorithmus und identischen Parametern.
    \item \textbf{A/B-Vergleich von Algorithmen.} Gegenüberstellung verschiedener Klassifizierungspipelines, mit z.\,B. alternativen Prompts oder anderem Preprocessing, über eine standardisierte HTTP-Schnittstelle, die in Kapitel~\ref{sec:api-design} definiert ist.
    \item \textbf{Explorative Analyse.} Detaillierte Einsicht pro Modell und Testfall (inklusive Begründungen und Visualisierungen), um Fehlklassifikationen gezielt zu untersuchen.
    \item \textbf{Berichterstellung.} Die Ergebnisse lassen sich als JSON oder Markdown exportieren und später wieder importieren, um sie erneut untersuchen zu können. Sie eignen sich zudem für die Publikation. Die Diagramme werden automatisch erzeugt und stehen ebenfalls zum Download bereit.
\end{itemize}

In dieser Arbeit werden keine A/B-Vergleiche unterschiedlicher Klassifizierungsalgorithmen durchgeführt, sondern lediglich verschiedene \acp{LLM} mit demselben Algorithmus verglichen. Dies ist eine bewusste Einschränkung des Untersuchungsrahmens, um die Analyse auf die Leistungsfähigkeit der \acp{LLM} zu fokussieren. Die in \ref{sec:api-design} definierte Schnittstelle erlaubt es jedoch, in zukünftigen Arbeiten alternative Klassifizierungsalgorithmen mit geringem Aufwand einzubinden.

\subsection*{Funktionale Anforderungen}

In der folgenden Tabelle sind die funktionalen Anforderungen an das Evaluationsframework aufgelistet, die notwendig sind, um die definierten Use-Cases zu erfüllen:

\begin{center}
    \reqTable{01}
    {Nutzen gelabelter Testdatensätze}
    {Das Framework kann die gelabelten Testdatensätze benutzen, die mit dem Labeling-Tool aus \ref{sec:labeling-tool} erstellt worden sind.}
    {}
\end{center}

\begin{center}
    \reqTable{02}
    {Vergleichbarkeit von Modellen und Algorithmen}
    {Das Framework erlaubt den direkten Vergleich verschiedener \acp{LLM} sowie unterschiedlicher Klassifizierungsalgorithmen anhand gelabelter Testdaten. Die Anbindung an Klassifizierungsalgorithmen erfolgt über die in Kapitel~\ref{sec:api-design} definierte, standardisierte HTTP-Schnittstelle.}
    {01}
\end{center}

\begin{center}
    \reqTable{03}
    {Deklarative Konfiguration}
    {Ein Evaluationslauf ist vollständig über eine YAML-Datei konfigurierbar. Dazu zählen Modelle, Klassifizierungsendpunkte, Testdatensätze und Seed. Experimente werden dadurch portabel und wiederholbar.}
    {02}
\end{center}

\begin{center}
    \reqTable{04}
    {Detaillierte Ergebnisaufbereitung}
    {
    Das Framework gibt Ergebnisse auf zwei Ebenen aus.
    \begin{enumerate}
        \item Pro Testfall und pro Modell: Status (\enquote{bestanden}/\enquote{nicht bestanden}), klassifizierte Elemente mit Begründungen, \ac{TP}/\ac{FP}/\ac{FN}/\ac{TN} und eine Visualisierung der Klassifikation im \ac{BPMN}-Prozess.
        \item Pro Modell als Summe über alle Testfälle: Accuracy, Precision, Recall, F1-Score und die Konfusionsmatrix.
    \end{enumerate}
    Zusätzlich protokolliert das Framework Metadaten der Evaluation, z.\,B. Endpunkt, verwendete Modelle und den Seed.
    }
    {02}
\end{center}

\begin{center}
    \reqTable{05}
    {Frontend}
    {Für eine einfache Bedienung und Ansicht der Ergebnisse bietet das Evaluationsframework ein Frontend an.}
    {02,03,04}
\end{center}

\begin{center}
    \reqTable{06}
    {Visualisierung und Berichte der Gesamtresultate}
    {Kennzahlen werden als Side-by-Side-Diagramme und tabellarisch dargestellt. Zusätzlich stehen Export/Import der Ergebnisse als JSON sowie ein Markdown-Report zur Verfügung.}
    {05}
\end{center}