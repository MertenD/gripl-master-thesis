% Functional-Requirement Table Design
%
% Oberhalb und unterhalb des Zeilen Contents muss noch irgendwie Platzeingefügt werden
\newcounter{linkcount}
\newcounter{linktotal}

\newcommand{\countLinks}[1]{%
    \setcounter{linkcount}{0}%
    \setcounter{linktotal}{0}%
    \renewcommand*{\do}[1]{\stepcounter{linktotal}}%
    \docsvlist{#1}%
}

\newcommand{\createHyperlinks}[1]{%
    \ifthenelse{\equal{#1}{-}}%
    {-}%
    {%
        \countLinks{#1}%
        \renewcommand*{\do}[1]{%
            \stepcounter{linkcount}%
            \hyperlink{FA##1}{FA##1}%
            \ifnum\value{linkcount}<\value{linktotal}%
            ,\ %
            \fi%
        }%
        \docsvlist{#1}%
    }%
}

\newcommand{\reqTable}[4]{
        {\rowcolors{2}{gray!50!white!20}{gray!40!white!10}
    \phantomsection\hypertarget{FA#1}{}
    \begin{tabular}{ p{3cm} p{10.2cm} }
        \rowcolor{darkgray}\textcolor{white}{\textbf{ID:}} & \textcolor{white}{\textbf{FA#1}} \\
        Titel: & #2 \\
        Beschreibung: & #3 \\
        Abhängigkeit: & \createHyperlinks{#4} \\
    \end{tabular}
    }
}

\newcommand{\createHyperlinksNFA}[1]{%
    \ifthenelse{\equal{#1}{-}}%
    {-}%
    {%
        \countLinks{#1}%
        \renewcommand*{\do}[1]{%
            \stepcounter{linkcount}%
            \hyperlink{NFA##1}{NFA##1}%
            \ifnum\value{linkcount}<\value{linktotal}%
            ,\ %
            \fi%
        }%
        \docsvlist{#1}%
    }%
}
\chapter{Evaluationsframework}\label{ch:evaluationsframework}

Nachdem nun Daten gelabelt werden können und der Testdatensatz für diese Arbeit erstellt wurde, wird in diesem Kapitel das Evaluationsframework vorgestellt. Das Framework nutzt die in Kapitel~\ref{ch:klassifizierungsalgorithmus-(design-und-implementierung)} entwickelte Klassifizierungspipeline, um verschiedene \acp{LLM} anhand gelabelter Testdaten systematisch, reproduzierbar und transparent zu vergleichen. Leitendes Gestaltungsprinzip ist die Entkopplung: Modelle, Klassifizierungsendpunkte und Testdaten werden zur Laufzeit über eine deklarative Konfiguration und eine standardisierte HTTP-Schnittstelle, siehe Abschnitt~\ref{sec:api-design}, angebunden. Dadurch sind sie austauschbar und erweiterbar, ohne Codeänderungen vornehmen zu müssen, zum Beispiel durch die Einbindung eines neuen Modells mit Endpunkt, Modellname und Parametern wie \texttt{temperature} oder \texttt{topP} sowie durch zusätzliche Datensätze. So wird ein fairer und reproduzierbarer Vergleich unterschiedlicher Modelle unter identischen Rahmenbedingungen ermöglicht.

\section{Use-Cases und Anforderungen}\label{sec:anforderungen-und-use-cases}

Das Evaluationsframework richtet sich an Forschende und Entwickler, die \acp{LLM} und Klassifizierungsalgorithmen für die Identifikation \ac{DSGVO}-kritischer \ac{BPMN}-\linebreak~Aktivitäten auswerten und miteinander vergleichen möchten. Es bietet eine einheitliche Ausführungs- und Auswertungsumgebung mit klar definierten Schnittstellen und standardisierten Berichten. In diesem Kapitel werden die Use-Cases und funktionale Anforderungen des Evaluationsframeworks beschrieben.

\subsection*{Use-Cases}

Die wichtigsten Anwendungsfälle des Evaluationsframeworks sind:

\begin{itemize}
    \item \textbf{Benchmarking von \acp{LLM}.} Systematischer Vergleich mehrerer \acp{LLM} auf denselben Datensätzen, mit identischem Algorithmus und identischen Parametern.
    \item \textbf{A/B-Vergleich von Algorithmen.} Gegenüberstellung verschiedener Klassifizierungspipelines, mit z.B.\ alternativen Prompts oder anderem Preprocessing, über eine standardisierte HTTP-Schnittstelle, die in Kapitel~\ref{sec:api-design} definiert ist.
    \item \textbf{Explorative Analyse.} Detaillierte Einsicht pro Modell und Testfall (inklusive Begründungen und Visualisierungen), um Fehlklassifikationen gezielt zu untersuchen.
    \item \textbf{Berichterstellung.} Die Ergebnisse lassen sich als JSON oder Markdown exportieren und später wieder importieren, um sie erneut untersuchen zu können. Sie eignen sich zudem für die Publikation. Die Diagramme werden automatisch erzeugt und stehen ebenfalls zum Download bereit.
\end{itemize}

In dieser Arbeit werden keine A/B-Vergleiche unterschiedlicher Klassifizierungsalgorithmen durchgeführt, sondern lediglich verschiedene \acp{LLM} mit demselben Algorithmus verglichen. Das Framework ist jedoch so konzipiert, dass dies in zukünftigen Arbeiten möglich ist.

\textcolor{orange}{// TODO In dem Absatz drüber noch erklären, warum keine A/B-Vergleiche unterschiedlicher Klassifizierungsalgorithmen durchgeführt, sondern lediglich verschiedene \acp{LLM} mit demselben Algorithmus verglichen werden.}

\subsection*{Funktionale Anforderungen}

In der folgenden Tabelle sind die funktionalen Anforderungen an das Evaluationsframework aufgelistet, die notwendig sind um die definierten Use-Cases zu erfüllen:

\begin{center}
    \reqTable{01}
    {Nutzen gelabelter Testdatensätze}
    {Das Framework kann die gelabelten Testdatensätze benutzten, die mit dem Labeling-Tool aus \ref{sec:labeling-tool} erstellt worden sind.}
    {}
\end{center}

\begin{center}
    \reqTable{02}
    {Vergleichbarkeit von Modellen und Algorithmen}
    {Das Framework erlaubt den direkten Vergleich verschiedener \acp{LLM} sowie unterschiedlicher Klassifizierungsalgorithmen anhand gelabelter Testdaten. Die Anbindung an Klassifizierungsalgorithmen erfolgt über die in Kapitel~\ref{sec:api-design} definierte, standardisierte HTTP-Schnitstelle.}
    {01}
\end{center}

\begin{center}
    \reqTable{03}
    {Deklarative Konfiguration}
    {Ein Evaluationslauf ist vollständig über eine YAML-Datei konfigurierbar. Dazu zählen Modelle, Klassifizierungsendpunkte, Testdatensätze und Seed. Experimente werden dadurch portabel und wiederholbar.}
    {02}
\end{center}

\begin{center}
    \reqTable{04}
    {Detaillierte Ergebnisaufbereitung}
    {
    Das Framework gibt Ergebnisse auf zwei Ebenen aus.
    \begin{enumerate}
        \item Pro Testfall und pro Modell: Status (\enquote{bestanden}/\enquote{nicht bestanden}), klassifizierte Elemente mit Begründungen, \ac{TP}/\ac{FP}/\ac{FN}/\ac{TN} und eine Visualisierung der Klassifikation im \ac{BPMN}-Prozess.
        \item Pro Modell als Summe über alle Testfälle: Accuracy, Precision, Recall, F1-Score und die Konfusionsmatrix.
    \end{enumerate}
    Zusätzlich protokolliert das Framework Metadaten der Evaluation, z.\,B. Endpunkt, verwendete Modelle und den Seed.
    }
    {02}
\end{center}

\begin{center}
    \reqTable{05}
    {Frontend}
    {Für eine einfache Bedienung und Ansicht der Ergebnisse bietet das Evaluationsframework ein Frontend an.}
    {02,03,04}
\end{center}

\begin{center}
    \reqTable{06}
    {Visualisierung und Berichte der Gesamtresultate}
    {Kennzahlen werden als Side-by-Side-Diagramme und tabellarisch dargestellt. Zusätzlich stehen Export/Import der Ergebnisse als JSON sowie ein Markdown-Report zur Verfügung.}
    {05}
\end{center}
\section{Konfiguration einer Evaluierung}\label{sec:konfiguration-einer-evaluierung}

Die funktionale Anforderung \hyperlink{FA03}{FA03} fordert, dass Evaluationsläufe deklarativ konfiguriert werden können. Das Framework unterstützt dies auf zwei Wegen:
Erstens bietet die Weboberfläche, die in \ref{sec:visualisierung-im-frontend} gezeigt wird, die Möglichkeit, Evaluationsläufe interaktiv zu konfigurieren und zu starten.
Zweitens lässt sich eine Evaluierung über eine YAML-Datei beschreiben, die entweder in der Weboberfläche hochgeladen oder per CLI an das Evaluationsframework übergeben wird. Auf diese Weise werden Reproduzierbarkeit und Versionierung der Evaluationsläufe sichergestellt. Listing \ref{lst:evaluation-config} zeigt ein Beispiel für eine solche YAML-Konfiguration. Ein ausführliches JSON-Schema ist im Anhang (Listing \ref{lst:evaluation-config-schema}) zu finden.

Die Evaluierungskonfiguration umfasst die folgenden Bausteine:

\begin{itemize}
    \item \texttt{defaultEvaluationEndpoint} ist der Standardendpunkt für die Klassifizierung. Er wird verwendet, wenn für ein Modell kein eigener Endpunkt angegeben ist. Der Endpunkt muss die in Kapitel \ref{sec:api-design} beschriebene API-Spezifikation erfüllen und kann relativ (gegen die Basis-URL des Evaluationsframeworks) oder absolut (für einen externen Dienst) angegeben werden.
\end{itemize}

\begin{lstlisting}[caption={Beispiel einer Evaluierungskonfiguration in YAML.},label={lst:evaluation-config}]
defaultEvaluationEndpoint: /gdpr/analysis/prompt-engineering
maxConcurrent: 10
repetitions: 3
seed: 42
models:
  - label: Mistral Medium 3.1
    llmProps:
      baseUrl: https://openrouter.ai/api/v1
      modelName: mistralai/mistral-medium-3.1
      apiKey: ${OPEN_ROUTER_API_KEY}
      topP: 1
  - label: Deepseek Chat v3.1
    llmProps:
      baseUrl: https://openrouter.ai/api/v1
      modelName: deepseek/deepseek-chat-v3.1
      apiKey: ${OPEN_ROUTER_API_KEY}
      temperature: 0.1
  - label: GPT oss 120b
    llmProps:
      baseUrl: https://openrouter.ai/api/v1
      modelName: openai/gpt-oss-120b
      apiKey: ${OPEN_ROUTER_API_KEY}
datasets:
  - 2
  - 7
\end{lstlisting}

\begin{itemize}
    \item \texttt{maxConcurrent} gibt die maximale Anzahl parallel auszuführender Testfälle an. So lassen sich beispielsweise \emph{Rate Limits}\footnote{Providerseitige Begrenzungen, etwa \enquote{Requests pro Minute} oder maximale Parallelität. Bei Überschreitung antworten viele Anbieter mit HTTP~429 (\enquote{Too Many Requests}). Zudem drohen strengere Drosselungen.} der angebundenen \acp{LLM} einhalten, um technische Fehler in den Ergebnissen zu vermeiden.
    \item \texttt{repetitions} bestimmt, wie oft die Evaluierung pro Modell wiederholt wird. Die Ergebnisse werden später über alle Wiederholungen aggregiert (siehe Abschnitt~\ref{sec:generierte-resultate}).
    \item \texttt{seed} legt einen Startwert (Seed) für reproduzierbare Evaluationsläufe fest. Auf Basis des Seeds und der Wiederholungsnummer wird für jede Wiederholung deterministisch ein eigener Seed generiert, um unterschiedliche, aber reproduzierbare Ergebnisse zu erzielen. Er wird bei jedem Modell an die \texttt{llmProps} weitergereicht und bei der Kommunikation mit den \acp{LLM} verwendet, sofern diese einen Seed unterstützen.
    \item \texttt{models} enthält die zu evaluierenden Modelle. Jedes Modell besitzt ein \texttt{label} zur Identifikation und optional spezifische \texttt{llmProps}, um die Eigenschaften des verwendeten \acp{LLM} zu definieren. Diese sind identisch zu den in Kapitel \ref{sec:api-design} beschriebenen \texttt{llmProps}.
    \item \texttt{datasets} ist eine Liste von Datensatz-\texttt{ids}, die jeweils eine Menge von Testfällen beinhalten.
\end{itemize}

Wie im Schema in Listing \ref{lst:evaluation-config-schema} gezeigt, kann jedem Modell optional ein eigener\linebreak~\texttt{evaluationEndpoint} zugewiesen werden, der den in \texttt{defaultEvaluation-\linebreak~Endpoint} definierten Standard überschreibt. Dadurch lassen sich unterschiedliche Klassifizierungsalgorithmen oder -versionen gezielt pro Modell vergleichen. Ist kein spezifischer Endpunkt angegeben, greift automatisch der Standardendpunkt.

API-Keys in den \texttt{llmProps} können optional als Umgebungsvariablen referenziert werden, wie im Beispiel in Listing \ref{lst:evaluation-config} gezeigt. So lassen sich sensible Daten sicher handhaben, ohne sie direkt in der Konfigurationsdatei zu speichern. Die Umgebungsvariablen werden zur Laufzeit aufgelöst und müssen daher im Kontext der Anwendung verfügbar sein.
\section{Architektur und Komponenten}\label{sec:architektur-und-komponenten}

\textcolor{orange}{// TODO Hier noch Fehlerbehandlung behandeln? Wenn irgendwo ein Fehler passiert, wird der Testfall als fehlgeschlagen markiert und im Bericht vermerkt. Außerdem habe ich glaub ich am Anfang auch erwähnt, dass Fehlerhafte Testfälle nicht in die Metriken einfließen, oder? Das muss ich dann auch tatsächlich noch implmentieren, weil sie aktuell als falsch klassifiziert gezählt werden.}

Das Evaluationsframework ist modular aufgebaut und nutzt eine Pipeline-Architektur, um eine flexible und skalierbare Evaluierung zu ermöglichen, wie es in \hyperlink{FA02}{FA02} gefordert ist. Die Architektur ist in Abbildung~\ref{fig:evaluation-framework-architecture} dargestellt. Sie besteht aus mehreren Hauptkomponenten, die jeweils eine klar definierte Aufgabe erfüllen. Im Folgenden werden die Komponenten und ihr Zusammenspiel beschrieben.

\begin{figure}
    \centering
    \includegraphics[width=\linewidth]{images/evaluation/evaluation-framework-architecture.drawio}
    \caption{Architektur des Evaluationsframeworks}
    \label{fig:evaluation-framework-architecture}
\end{figure}

\subsection*{Einstiegspunkte}

Das Framework bietet zwei Einstiegspunkte zur Ausführung einer Evaluierung:

\begin{itemize}
    \item \textbf{EvaluationController} als HTTP-Controller stellt REST-Endpunkte bereit, über die Evaluierungen gestartet werden können. Er akzeptiert eine YAML-Konfiguration und gibt die Ergebnisse entweder auf einmal als Markdown-Bericht oder als JSON-Stream zurück. Der Controller ermöglicht die Nutzung des Frameworks über die Weboberfläche, die in Kapitel \ref{sec:visualisierung-im-frontend} gezeigt wird, sowie über HTTP-APIs. Durch das Streamen der Ergebnisse können bereits abgeschlossene Testfälle sofort angezeigt werden, ohne auf das Ende der gesamten Evaluierung warten zu müssen.
    \item \textbf{EvaluationCommand} ist ein CLI-Befehl, der die Ausführung von Evaluierungen über die Kommandozeile erlaubt. Er liest eine YAML-Konfigurationsdatei ein, führt die Evaluierung aus und schreibt die Ergebnisse in eine Markdown-Datei. Dies eignet sich besonders für automatisierte Ausführungen, Continuous Integration oder die lokale Entwicklung.
\end{itemize}

Beide Einstiegspunkte akzeptieren die Konfiguration aus Kapitel \ref{sec:konfiguration-einer-evaluierung}, lösen ggf.\ Umgebungsvariablen auf und delegieren die Ausführung der Evaluation an den \texttt{MultiEvaluationRunner}.

\subsection*{Orchestrierung mit \texttt{MultiEvaluationRunner}}

Der \texttt{MultiEvaluationRunner} ist für die Orchestrierung der gesamten Evaluierung verantwortlich. Er verarbeitet die Konfiguration, die mehrere Modelle und Datensätze beschreibt, und koordiniert die sequenzielle Evaluierung aller konfigurierten Modelle. Für jedes Modell ruft der \texttt{MultiEvaluationRunner} den \texttt{EvaluationRunner} auf und übergibt diesem alle Informationen zur Ausführung der Evaluation eines einzelnen Modells.

Der \texttt{MultiEvaluationRunner} stellt zudem sicher, dass alle Modelle denselben Seed verwenden, um reproduzierbare Ergebnisse zu gewährleisten. Falls in der Konfiguration kein Seed angegeben wurde, wird an dieser Stelle einer erzeugt. Im Anschluss werden die Metadaten über die verwendeten Datensätze, Modelle, Endpunkte und den Seed gesammelt und als \texttt{MetadataReport} als Teil des Evaluationsberichts zurückgegeben. Die Teile des Berichts werden als \texttt{Flow} von Berichtartefakten zurückgegeben, wodurch eine streambasierte Verarbeitung ermöglicht wird. Welche Arten von Berichtartefakten es gibt, wird in kapitel \ref{sec:generierte-resultate} beschrieben.

\subsection*{Ausführung mit \texttt{EvaluationRunner}}

Der \texttt{EvaluationRunner} führt die Evaluierung für ein einzelnes Modell durch. Er lädt die Testfälle der angegebene Testdatensätze aus der Datenbank, führt die Klassifizierung für jeden Testfall aus und sammelt die Ergebnisse. Die Testfälle werden parallel verarbeitet, wobei die Anzahl der gleichzeitigen Ausführungen durch den Parameter \texttt{maxConcurrent} in der Konfiguration gesteuert wird. Dies ermöglicht es, Rate-Limits von \acp{LLM}-Diensten einzuhalten und die Auslastung der Ressourcen zu kontrollieren.

Für jeden Testfall delegiert der \texttt{EvaluationRunner} die eigentliche Klassifizierung an den \texttt{HttpEvaluator}. Anschließend vergleicht er die erwarteten mit den tatsächlichen Ergebnissen und berechnet die Klassifikationsmetriken wie \ac{TP}, \ac{FP}, \ac{FN} und \ac{TN}. Die Ergebnisse werden in \texttt{TestCaseReport}-Objekten zusammengefasst, als Teilergebnis zurückgegeben, und an den \texttt{MetricsAccumulator} weitergeleitet.

Der \texttt{EvaluationRunner} gibt die Ergebnisse ebenfalls als \texttt{Flow} zurück, wodurch eine frühzeitige Rückgabe von Teilergebnissen ermöglicht wird. Dies ist besonders vorteilhaft für die Live-Ansicht in der Weboberfläche, da Testfallergebnisse sofort nach ihrer Fertigstellung angezeigt werden können.

\subsection*{Klassifizierung mit \texttt{HttpEvaluator}}

Der \texttt{HttpEvaluator} ist für die Kommunikation mit dem Klassifizierungsendpunkt verantwortlich, der das in Kapitel \ref{sec:api-design} beschriebene Interface implementiert. Er nimmt das \ac{BPMN}-Modell aus dem aktuellen Testfall und die \texttt{llmProps} von dem aktuellen Modell aus der Konfiguration entgegen, baut einen HTTP-Request auf und sendet diesen an den konfigurierten Endpunkt. Nach erfolgreicher Klassifizierung extrahiert er die Liste der als kritisch identifizierten Aktivitäten aus der Antwort und gibt diese an den \texttt{EvaluationRunner} zurück.

\subsection*{Akkumulierung mit \texttt{MetricsAccumulator}}

Der \texttt{MetricsAccumulator} sammelt die Metriken aller Testfälle eines Modells und berechnet daraus aggregierte Werte. Er ist thread-sicher implementiert und kann gleichzeitig von mehreren parallelen Evaluierungen genutzt werden. Das ist wichtig, da der \texttt{EvaluationRunner} die Testfälle parallel ausführt und somit mehrere Threads gleichzeitig auf den \texttt{MetricsAccumulator} zugreifen können.

Nach Abschluss aller Testfälle erzeugt der \texttt{MetricsAccumulator} ein\break \texttt{EvaluationReportSummary}-Objekt, das alle Metriken für die Evaluation eines Modells über mehrere Testfälle hinweg enthält.

\subsection*{Zusammenfassung}

Die Architektur trennt Zuständigkeiten strikt:
\texttt{MultiEvaluationRunner} koordiniert Modellläufe,
\texttt{EvaluationRunner} verarbeitet Testfälle und sammelt Metriken,
\texttt{HttpEvaluator} kommuniziert mit der Klassifizierungs-Pipeline,
\texttt{MetricsAccumulator} aggregiert Ergebnisse pro Modell über mehrere Testfälle.

\section{Evaluationsergebnisse}\label{sec:generierte-resultate}

Im vorherigen Abschnitt wurde erwähnt, dass die Komponenten des Evaluationsframeworks Berichtartefakte zurückgeben. Diese sind im im Architekturbild \ref{fig:evaluation-framework-architecture} als Beschriftungen über den gestrichelten Pfeilen dargestellt. Im Folgenden werden die Berichtartefakte beschrieben:

\begin{description}
    \item[\texttt{MetadataReport}] Berichtsartefakt, das der \texttt{MultiEvaluationRunner} zu Beginn der Evaluierung erzeugt. Es enthält Metadaten zur Evaluierung, z.\,B.\ Informationen über die Testdatensätze, die Anzahl der Testfälle sowie den verwendeten Seed. Das \texttt{MetadataReport}-Artefakt wird zuerst zurückgegeben, damit die Weboberfläche bereits Metadaten anzeigen kann, während die Evaluierung noch läuft.

    \item[\texttt{TestCaseReport}] Berichtsartefakt, das der \texttt{EvaluationRunner} für jeden abgeschlossenen Testfall erzeugt. Es enthält u.\,a.\ die Testfall-\texttt{id}, das Klassifizierungsergebnis und die für diesen Testfall berechneten Metriken. \texttt{TestCaseReport}-Artefakte werden fortlaufend bereitgestellt, sobald ein Testfall abgeschlossen ist, sodass die Weboberfläche Ergebnisse unmittelbar anzeigen kann.

    \item[\texttt{EvaluationReportSummary}] Berichtsartefakt, das der \texttt{MetricsAccumulator} am Ende der Evaluierung eines Modells erzeugt. Es fasst die aggregierten Metriken, wie z.\,B.\ \emph{Precision}, \emph{Recall}, \emph{F1-Score} und \emph{Accuracy}, sowie die Konfusionsmatrix zusammen. Das \texttt{EvaluationReportSummary}-Artefakt wird als letztes Berichtsartefakt pro Modell zurückgegeben und dient dem Modellvergleich in der Weboberfläche.
\end{description}

Die Informationen dieser Berichtsartefakte ermöglichen die Generierung eines ausführlichen Evaluierungsberichts, wie in \hyperlink{FA04}{FA04} gefordert. Im Folgenden ist dargestellt, welche Informationen nach Abschluss einer Evaluierung vorliegen.

\subsection*{Pro Testfall und Modell}

Für jeden Testfall eines Modells liegen vor: die von der Klassifizierungs-Pipeline zurückgegebenen klassifizierten Aktivitäten (mit optionalen Begründungen), die gelabelten erwarteten Aktivitäten, die Zählwerte für \emph{\ac{TP}}, \emph{\ac{FP}}, \emph{\ac{FN}} und \emph{\ac{TN}} sowie eine Bild-URL zur Visualisierung des \ac{BPMN}-Modells mit hervorgehobenen Aktivitäten. Aus diesen Informationen lässt sich ableiten, ob der Testfall erfolgreich war. Ein Testfall gilt als erfolgreich, wenn die klassifizierten Aktivitäten exakt den erwarteten Aktivitäten entsprechen. Technische Probleme, die während der Klassifizierung auftreten, werden ebenfalls erfasst, z.\,B.\ Parsing-Fehler, ungültiges \ac{BPMN}, Token-Limit-Überschreitungen oder Zeitüberschreitungen.

\subsection*{Pro Modell über alle Testfälle}

Auf Modellebene stehen die Gesamtergebnisse über alle Testfälle zur Verfügung. Dazu gehören die aggregierten Kennzahlen \emph{Precision}, \emph{Accuracy}, \emph{Recall} und \emph{F1-Score} sowie eine Konfusionsmatrix mit den Gesamtwerten für \emph{\ac{TP}}, \emph{\ac{FP}}, \emph{\ac{FN}} und \emph{\ac{TN}}. Zusätzlich sind die Anzahlen der korrekt bzw.\ falsch klassifizierten sowie der technisch fehlgeschlagenen Testfälle aufgeführt.

\subsection*{Über alle Modelle}

Abschließend sind die Metadaten der gesamten Evaluierung verfügbar: die verwendeten Testdatensätze, die Anzahl der Testfälle, die konfigurierten Modelle, der für die Reproduzierbarkeit verwendete Seed sowie ein Zeitstempel der Evaluierung. Zum unmittelbaren Vergleich werden die aggregierten Kennzahlen aller Modelle nebeneinander dargestellt.

\section{Nutzung über Webapp}\label{sec:nutzung-uber-webapp}

Zur interaktiven Nutzung der Klassifizierung wurde eine \emph{Sandbox} in Form einer Webapp entwickelt. Sie verbindet einen vollwertigen \ac{BPMN}-Editor auf Basis von \texttt{BPMN.js} \cite{bpmn-js} mit der in Kapitel~\ref{sec:api-design} beschriebenen HTTP-Schnittstelle und macht die Analyse damit ganz einfach bedienbar. In der Sandbox können \ac{BPMN}-Modelle erstellt, verändert, exportiert und importiert sowie auf Datenschutzrelevanz analysiert werden. Als kritisch klassifizierte Aktivitäten werden nach der Analyse direkt im Editor farblich hervorgehoben, wie in Abbildung \ref{fig:sandbox-frontend-analyzed-model} zu sehen.

\begin{figure}
    \centering
    \includegraphics[width=\linewidth]{images/sandbox/sandbox-analyzed-model}
    \caption{Sandbox im Frontend mit hervorgehobenen kritischen Aktivitäten nach Analyse.}
    \label{fig:sandbox-frontend-analyzed-model}
\end{figure}

Außerdem können die vom \ac{LLM} generierten Begründungen zu jeder als kritisch erkannten Aktivität im Editor eingesehen werden. Diese Erläuterungen werden gesammelt in einer aufklappbaren Karte im unteren Bereich des Editors angezeigt, siehe Abbildung \ref{fig:sandbox-frontend-ai-reasoning}.

\begin{figure}
    \centering
    \includegraphics[width=\linewidth]{images/sandbox/sandbox-ai-reasoning}
    \caption{Begründung der Klassifikation durch das LLM in der Sandbox.}
    \label{fig:sandbox-frontend-ai-reasoning}
\end{figure}

Um verschiedene \acp{LLM} vergleichen zu können, verfügt die Sandbox auf der rechten Seite über ein Einstellungsmenü mit konfigurierbaren \ac{LLM}-Parametern (siehe Abbildung \ref{fig:sandbox-frontend-analyzed-model}). Diese Parameter sind identisch zu den in Kapitel \ref{sec:api-design} beschriebenen \texttt{llmProps} und werden beim Absenden der Analyse in die API-Anfrage überführt.