\section{Fallstudien}\label{sec:fallstudien}

Die aggregierten Kennzahlen liefern einen guten Überblick über die Modellqualität, verbergen jedoch individuelle Fehlklassifikationen.

Abbildung~\ref{fig:results-testcase-outcomes} vermittelt auf den ersten Blick den Eindruck, dass viele Modelle nur wenige der Testfälle bestehen. Eine genaue Analyse zeigt jedoch, dass hinter vielen dieser fehlgeschlagenen Testfälle lediglich ein oder wenige plausible \ac{FP} stehen. Das bedeutet: Oft markiert das Modell eine zusätzliche Aktivität als kritisch, während alle tatsächlich kritischen Aktivitäten korrekt identifiziert werden. Solche Fälle lassen den Testfall formal als \enquote{nicht bestanden} erscheinen, obwohl die Gesamtleistung des Modells durchaus hoch ist.

Im Folgenden verdeutlichen drei ausgewählte Fallstudien typische Muster von \ac{FP} und \ac{FN}.

\subsection*{Sales Warehouse}

Der englische Prozess \enquote{Sales Warehouse} aus dem Datensatz \enquote{Universität} enthält vier als kritisch gelabelte Aktivitäten und ist in Abbildung \ref{fig:qwen3-fall} zu sehen. Das Modell \texttt{Qwen3-235B-A22B-Thinking-\linebreak~2507} klassifiziert die gelabelten korrekt, markiert jedoch zusätzlich die Aktivität \enquote{Ship product} als kritisch, was in \Abbildung{fig:qwen3-fall} als rot markierte Aktivität dargestellt ist. Die Begründung verweist auf die Nutzung der Kundenadresse für Versand und Zustellung. Obwohl während der Modellierung nur ein rein logistischer Schritt vorgesehen war - der laut Tabelle \ref{tab:labeling-examples} in dem Labeling-Guide nicht kritisch ist - interpretiert das Modell den möglichen Datenfluss und wählt eine konservative Einstufung. Angesichts des Zielkriteriums eines hohen Recalls ist dieses \ac{FP} vertretbar.

\begin{figure}
    \centering
    \includegraphics[height=.41\textheight]{images/results/examples/qwen3-235B-run-3-uni-sales-warehouse}
    \caption{Ergebnis des Testfalls \enquote{Sales Warehouse} mit farblich hervorgehobenen Aktivitäten. Grün markierte Aktivitäten sind korrekt als kritisch erkannt, rot markierte stellen \acp{FP} dar.}
    \label{fig:qwen3-fall}
\end{figure}

Das Beispiel verdeutlicht eine grundsätzliche Limitierung der Klassifizierung. Wenn in einem \ac{BPMN}-Modell explizite Informationen über Datenverarbeitungen fehlen, ist es für das das \ac{LLM} schwierig, eine eindeutige Klassifikation vorzunehmen.

\subsection*{Marketing-Kampagne}

Im deutschen Testfall \enquote{Marketing-Kampagne}, aus dem Datensatz \enquote{Kleine Szenarien}, sind drei Aktivitäten als kritisch gelabelt: \enquote{Leads sammeln}, \enquote{Newsletter versenden} und \enquote{CRM aktualisieren}. \texttt{GPT-OSS-20B} identifiziert diese korrekt, markiert aber zusätzlich die Aktivität \enquote{Klickraten auswerten} als kritisch. Im Prozessmodell wird davon ausgegangen, dass die Klickdaten vollständig anonymisiert werden, jedoch ist diese Information nicht explizit hinterlegt. Mehrere Modelle – darunter \texttt{Qwen3-235B-A22B-Thinking-2507} – stufen diesen Schritt daher als potenziell personenbezogen ein, wie in Abbildung \ref{fig:gptoss-fall} zu sehen ist. Die Anonymisierung der Daten wurde nur von \texttt{Mistral-7B-Instruct-v0.3} in zwei von fünf und von den Gemma-Modellen in allen Wiederholungen korrekt antizipiert und die Aktivität als unkritisch klassifiziert. Dieses Beispiel zeigt, wie fehlende Kontextinformationen zu vorsichtiger Klassifikation und damit zu \ac{FP} führen können.

\begin{figure}
    \centering
    \includegraphics[width=\textwidth]{images/results/examples/oss-20b-run-1-small-marketing}
    \caption{Ergebnis des Testfalls \enquote{Marketing-Kampagne} mit farblich hervorgehobenen Aktivitäten. Die Aktivität \enquote{Klickraten auswerten} wurde als zusätzliches kritisches Element markiert.}
    \label{fig:gptoss-fall}
\end{figure}

Dieses Beispiel zeigt, dass ohne genaue Kontextangaben zur Anonymisierung selbst scheinbar unbedenkliche Auswertungen als datenschutzrelevant erscheinen können. Es unterstreicht, dass die \acp{LLM} im Zweifel eher ein kritisches Label vergeben, um \acp{FN} zu vermeiden, wie es das Hauptziel der Klassifikation aus Abschnitt \ref{sec:qualitatsziele} vorsieht.

\subsection*{Karten-App – Standort erfassen}

Der Testfall \enquote{Karten-App – Standort erfassen} besteht aus zwei Aktivitäten: \enquote{Standort erfassen} und \enquote{Route berechnen}. Beide sollten als kritisch klassifiziert werden, da im zweiten Schritt der erfasste Standort zur Berechnung der Route genutzt wird. \texttt{Mistral-Large-Instruct-2411} kennzeichnet jedoch in drei von fünf Wiederholungen nur die erste Aktivität als kritisch, wie in Abbildung \autoref{fig:mistral-fall} zu sehen. In der Begründung wird zwar die Verarbeitung personenbezogener Daten beim Erfassen des Standorts erkannt, dieser Zusammenhang aber nicht auf die nachfolgende Aktivität übertragen. Dieses \ac{FN} steht dem angestrebten hohen Recall entgegen und zeigt, dass einige Modelle Schwierigkeiten haben, Datenflüsse über mehrere Schritte hinweg zu verfolgen.

\begin{figure}
    \centering
    \includegraphics[width=.55\textwidth]{images/results/examples/mistral-large-run-3-small-maps-app}
    \caption{Ergebnis des Testfalls \enquote{Karten-App – Standort Erfassen} mit farblich hervorgehobenen Aktivitäten. Die Aktivität \enquote{Route berechnen} wurde fälschlicherweise nicht als kritisch markiert.}
    \label{fig:mistral-fall}
\end{figure}

Auf Basis der Erkenntnisse dieses gesamten Kapitels werden im folgenden Abschnitt die formulierten Forschungsfragen beantwortet. Dabei wird untersucht welches Modell sich insgesamt am besten für die Identifikation \ac{DSGVO}‑kritischer Aktivitäten eignet.