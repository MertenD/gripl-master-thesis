\chapter{Ergebnisse}\label{ch:ergebnisse}

Dieses Kapitel präsentiert die Resultate der Klassifizierungsexperimente und stellt sie in den Kontext der in Abschnitt~\ref{sec:zielsetzung-und-beitrage} formulierten Forschungsfragen und Qualitätsziele. Ziel der Untersuchung ist es, \acp{LLM} auf ihre Fähigkeit zur Identifikation \ac{DSGVO}-kritischer Aktivitäten in \ac{BPMN}-Prozessmodellen zu prüfen. Die folgenden Abschnitte fassen die zentralen Erkenntnisse zusammen, analysieren die Ergebnisse entlang verschiedener Modellkategorien, bewerten die Robustheit und veranschaulichen typische Fehlerbilder anhand von Fallstudien. Abschließend werden die Forschungsfragen beantwortet.

Alle Abbildungen und Tabellen in diesem Kapitel wurden auf Basis der im Evaluationsframework generierten Experimentergebnisse erstellt. Dabei wurden die Ergebnisse aus den einzelnen Experimenten zusammengeführt, sodass sämtliche Modelle gemeinsam verglichen werden können. Die im Evaluationsframework erzeugten Diagramme wurden bewusst nicht direkt übernommen, da sie bei vielen Modellen nur schwer skalieren. Die vollständigen Reports und Rohdaten der einzelnen Experimente können weiterhin dem Repository\footnote{Siehe das GitLab Repository: \hyperlink{// TODO}{// TODO}} entnommen und über das Evaluationsframework erkundet werden.

\chapter{Zusammenfassung}\label{ch:zusammenfassung}

Hier neben der allgemeinen Zusammenfassung unbedingt noch die erste Forschungsfrage explizit beantworten
\section{Analyse nach Modellkategorien}\label{sec:analyse-nach-modellkategorien}

Für ein besseres Verständnis der Leistungsunterschiede werden die Modelle im Folgenden nach verschiedenen Kriterien gruppiert und verglichen. Dabei wird jeweils diskutiert, wie sich die Gruppen in Bezug auf die Qualitätsziele unterscheiden und welche Trends sich beobachten lassen. Die Implikationen für Praxis und Forschungsfragen werden am Ende dieses Kapitels in Abschnitt~\ref{sec:antworten-auf-forschungsfragen} gebündelt beantwortet.

\subsection*{Proprietäre versus Open-Weight Modelle}

Die beiden proprietären Modelle \texttt{GPT-4o} und \texttt{Mistral Medium 3.1} erreichen F1-Scores von $0{,}822$ bzw.\ $0{,}843$. Trotz seiner exzellenten Precision von $0{,}892$ verfehlt \texttt{GPT-4o} aufgrund des niedrigen Recalls von $0{,}762$ das Mindestziel und übersieht damit relativ viele kritische Aktivitäten. \texttt{Mistral Medium 3.1} bietet mit einem Recall von $0{,}877$ eine bessere Balance und erfüllt alle Qualitätsziele.

Die offene Kategorie zeigt ein heterogenes Bild. Mehrere Modelle wie \texttt{Qwen3-235B-\linebreak~A22B-Thinking-2507} mit einem F1-Score $= 0{,}874$, \texttt{GPT-OSS-20B} mit F1-Score $= 0{,}866$ und \texttt{DeepSeek-R1-Distill-Qwen-14B} mit F1-Score $= 0{,}848$ übertreffen die proprietären Modelle. Sie erkennen kritische Aktivitäten sehr zuverlässig und klassifizieren nur wenige unkritische Aktivitäten fälschlich als kritisch. Gleichzeitig gibt es mit \texttt{Mixtral-8x7B-Instruct-v0.1}, das F1-Score $= 0{,}596$ erzielte, auch klare Ausreißer nach unten, die weder genug kritische Aktivitäten erkennen noch eine akzeptable Precision bieten.

Insgesamt zeigt sich, dass hochwertige offene Modelle ein besseres Verhältnis von Recall und Precision aufweisen und die Qualitätsziele häufig klar erfüllen. Für die Praxis bedeutet dies, dass offene Modelle eine attraktive Alternative zu proprietären Lösungen darstellen, jedoch ist die Auswahl des Modells entscheidend, da die Leistungsunterschiede innerhalb der offenen Kategorie erheblich sind.

\subsection*{Kleine versus Große Modelle}

Tabelle \ref{tab:small-vs-large} vergleicht die Mittelwerte der Metriken für kleine Modelle ($\leq 25$\,B Parameter) und große Modelle ($>25$\,B Parameter). Im Durchschnitt unterscheiden sich die Gruppen nur geringfügig. Die kleinen Modelle erreichen einen mittleren F1-Score von $0{,}805$ und die großen Modelle $0{,}806$, wobei die großen Modelle ohne den Ausreißer \texttt{Mixtral-8x7B-Instruct-v0.1} einen leicht höheren Durchschnitt von $0{,}836$ erreichen. Der beste F1-Score unter den kleinen Modellen stammt von \texttt{GPT-OSS-20B} mit $0{,}866$, bei den großen Modellen führt \texttt{Qwen3-235B-A22B-\linebreak~Thinking-2507} mit $0{,}874$. Bemerkenswert ist der leicht höhere durchschnittliche Recall der kleinen Modelle von $0{,}843$ gegenüber den großen mit $0{,}839$, wohingegen Precision und Accuracy annähernd identisch sind.

\begin{table}[htbp]
 \centering
 \caption{Kleine vs. große Modelle: Durchschnittswerte pro Gruppe und jeweils bestes Modell.}
 \label{tab:small-vs-large}
 \begin{adjustbox}{width=\textwidth}
  \begin{threeparttable}[width=\textwidth]
   \begin{tabular}[width=\textwidth]{l r r}
    \toprule
    \textbf{Metrik} & \textbf{Klein} ($\leq 25B$) & \textbf{Groß} ($> 25B$) \\
    \midrule
    Anzahl Modelle\tnote{1}             & 5                         & 8 \\
    Ø F1-Score $\pm$ SD\tnote{2}      & 0,805 $\pm$ 0,057                     & 0,806 $\pm$ 0,089 \\
    Ø Precision $\pm$ SD    & 0,774 $\pm$ 0,050                     & 0,779 $\pm$ 0,085 \\
    Ø Recall $\pm$ SD       & 0,843 $\pm$ 0,086                     & 0,839 $\pm$ 0,128 \\
    Ø Accuracy $\pm$ SD     & 0,744 $\pm$ 0,067                     & 0,749 $\pm$ 0,099 \\
    Bester F1-Score & 0,866                     & 0,874 \\
    Bestes Modell (F1-Score)   & GPT-OSS-20B               & Qwen3-235B-A22B-Thinking-2507 \\
    Bester Precision & 0,829                     & 0,892 \\
    Bestes Modell (Precision) & DeepSeek-R1-Distill-Qwen-14B        & GPT-4o \\
    Bester Recall & 0,918                     & 0,932 \\
    Bestes Modell (Recall)      & GPT-OSS-20B      & Qwen3-235B-A22B-Thinking-2507 \\
    Beste Accuracy & 0,821                     & 0,830 \\
    Bestes Modell (Accuracy)     & GPT-OSS-20B               & Qwen3-235B-A22B-Thinking-2507 \\
    \bottomrule
   \end{tabular}
   \begin{tablenotes}
    \footnotesize
    \item[1] Einteilung nach gesamten Milliarden Parametern bei \ac{MoE}. Die Proprietären Modelle \texttt{GPT-4o} und \texttt{Mistral Medium 3.1} wurden trotz fehlender Parameterangabe als große Modelle eingeordnet.
    \item[2] Ohne \texttt{Mixtral-8x7B-Instruct-v0.1} beträgt der Durchschnitt der großen Modelle $\pm$ SD $0.836 \pm 0.029$.
   \end{tablenotes}
  \end{threeparttable}
 \end{adjustbox}
\end{table}

Diese Ergebnisse bestätigen, dass die Modellgröße allein kein Garant für eine bessere Klassifikationsleistung ist. Kleinere Modelle wie \texttt{GPT-OSS-20B} liefern sehr starke Screening-Leistung bei geringeren Kosten und lassen sich leichter On-\linebreak~Premises\footnote{
On-Premises bezeichnet den Betrieb von IT-Systemen im eigenen Rechenzentrum statt in der Cloud.
} betreiben. In der Praxis sollte daher das Auswahlkriterium für ein Modell die \emph{Balance aus Recall, Precision und Betriebskosten} sein, nicht die Parameterzahl.

\subsection*{Europäische vs. internationale Modelle}

Tabelle \ref{tab:eu-vs-international} stellt die Mittelwerte der europäischen Mistral-Modelle den übrigen internationalen Modellen gegenüber. Die europäischen Modelle zeigen eine größere Streuung: das kommerzielle \texttt{Mistral Medium 3.1} erfüllt mit einem F1-Score von $0{,}843$ und Recall von $0{,}877$ alle Zielkriterien und liegt knapp vor dem Referenzmodell \texttt{GPT-4o}. Ähnlich sieht es bei dem Open-Weight-Modell \texttt{Mistral-Large-\linebreak~Instruct-2411} aus. Dagegen verfehlen \texttt{Mistral-7B-Instruct-v0.3} und insbesondere \texttt{Mixtral-8x7B-Instruct-v0.1} die Qualitätsziele deutlich. Im Durchschnitt bleiben die europäischen Modelle hinter den internationalen Spitzenreitern zurück. Letztere – allen voran \texttt{Qwen3-235B-A22B-Thinking-2507} mit einem F1-Score von $0{,}874$ und \texttt{GPT-OSS-20B} mit $0{,}866$ – erreichen im Mittel einen höheren F1-Score sowie Recall und weisen eine geringere Varianz auf.

\begin{table}[htbp]
 \centering
 \caption{Europäische vs. internationale Modelle: Durchschnittswerte pro Gruppe und jeweils bestes Modell.}
 \label{tab:eu-vs-international}
 \begin{adjustbox}{width=\textwidth}
  \begin{threeparttable}[width=\textwidth]
   \begin{tabular}[width=\textwidth]{l r r}
    \toprule
    \textbf{Metrik} & \textbf{\ac{EU}-Modelle} & \textbf{Internationale Modelle} \\
    \midrule
    Anzahl Modelle              & 4                           & 9 \\
    Ø F1-Score $\pm$ SD         & 0{,}760 $\pm$ 0{,}098       & 0{,}826 $\pm$ 0{,}045 \\
    Ø Precision $\pm$ SD        & 0{,}738 $\pm$ 0{,}061       & 0{,}797 $\pm$ 0{,}056 \\
    Ø Recall $\pm$ SD           & 0{,}789 $\pm$ 0{,}138       & 0{,}864 $\pm$ 0{,}076 \\
    Ø Accuracy $\pm$ SD         & 0{,}694 $\pm$ 0{,}101       & 0{,}771 $\pm$ 0{,}057 \\
    Bester F1-Score             & 0{,}843                     & 0{,}874 \\
    Bestes Modell (F1-Score)    & Mistral Medium 3.1          & Qwen3-235B-A22B-Thinking-2507 \\
    Bester Precision            & 0{,}811                     & 0{,}892 \\
    Bestes Modell (Precision)   & Mistral Medium 3.1          & GPT-4o \\
    Bester Recall               & 0{,}877                     & 0{,}932 \\
    Bestes Modell (Recall)      & Mistral Medium 3.1          & Qwen3-235B-A22B-Thinking-2507 \\
    Beste Accuracy              & 0{,}794                     & 0{,}830 \\
    Bestes Modell (Accuracy)    & Mistral Medium 3.1          & Qwen3-235B-A22B-Thinking-2507 \\
    \bottomrule
   \end{tabular}
   \begin{tablenotes}
    \footnotesize
    \item Die \ac{EU}‑Modelle umfassen \texttt{Mistral‑7B‑Instruct‑v0.3}, \texttt{Mixtral‑8x7B‑Instruct‑v0.1}, \texttt{Mistral‑Large‑Instruct‑2411} und \texttt{Mistral Medium 3.1}. Die internationalen Modelle sind die übrigen in Kapitel \ref{sec:ueberblick} betrachteten Modelle.
   \end{tablenotes}
  \end{threeparttable}
 \end{adjustbox}
\end{table}

Diese Vergleiche belegen, dass ein europäischer Ursprung nicht zwangsläufig mit einer geringeren Leistung einhergeht – \texttt{Mistral Medium 3.1} erreicht gute Werte, dicht gefolgt von \texttt{Mistral-Large-Instruct-2411}. Allerdings zeigen die Ergebnisse auch, dass einige europäische Modelle hinter den internationalen Konkurrenten zurückbleiben. Insgesamt sind die internationalen Modelle im Durchschnitt leistungsfähiger und stabiler, da sie die europäischen Modelle in jeder Metrik im Durchschnitt übertreffen und eine geringere Varianz aufweisen. Zudem ist in jeder Metrik ein internationales Modell führend.
\section{Robustheit}\label{sec:robustheit2}

Die Robustheitsanalyse über mehrere Seeds unterstreicht die Praxistauglichkeit der meisten Modelle in Kombination mit der entwickelten Klassifizierungspipeline. Für die Mehrzahl der \acp{LLM} liegt die Standardabweichung des F1-Scores über fünf Wiederholungen bei $\leq 0{,}02$, was auf eine geringe Varianz und konsistente Leistung hinweist. Vereinzelt zeigen Modelle eine höhere Varianz oder benötigen mehr Wiederholungen zur Korrektur von Parsing-Fehlern. Solche Unterschiede sind für den operativen Einsatz relevant, da sie sich direkt in Durchsatz, Latenz und Stabilität der Gesamtpipeline niederschlagen. Modelle mit erhöhter Varianz sollten daher im produktiven Betrieb sorgfältig überwacht und validiert werden, um unerwartete Leistungseinbußen zu vermeiden.

Wesentlich zur Zuverlässigkeit trägt die entwickelte Klassifizierungspipeline bei. \emph{Structured Output} via Langchain4j mit explizitem JSON-Schema (und, wo verfügbar, API-seitig erzwungenem \texttt{response\_format}) erhöht die Format-Treue, ein explizites \texttt{isRelevant}-Flag mit nachgelagertem Relevanz-Filter entschärft Widersprüche zwischen Begründung und Klassifikation, die \emph{id-Validierung/-Vervoll-\linebreak~ständigung} reduziert typische Ausgabefehler und der \emph{Retry-Mechanismus} behebt Parsing-Fehler automatisiert. Diese Maßnahmen tragen wesentlich zur Ergebnisstabilität bei und sollten in produktiven Systemen implementiert werden, um die Zuverlässigkeit der Modellausgaben zu gewährleisten.

Ob das Preprocessing der Klassifizierungspipeline zur Leistung beiträgt, lässt sich nicht abschließend beurteilen. Die im nächsten Abschnitt beschriebenen Fallstudien legen nahe, dass trotz des im Preprocessing bereitgestellten Kontexts Datenflüsse und Prozesszusammenhänge durch \ac{LLM} weiterhin unberücksichtigt bleiben oder falsch interpretiert werden. Hier könnten künftige Anpassungen der Pipeline ansetzen, um den Kontext für die Modelle weiter zu verbessern.
\section{Fallstudien}\label{sec:fallstudien}

Neben den aggregierten Metriken aus den Ergebnissen bieten einzelne Testfälle wichtige Einblicke in die Stärken und Schwächen der Modelle. Im Folgenden werden drei exemplarische Szenarien vorgestellt, die jeweils typische Fehlklassifikationen illustrieren: ...

\subsection*{Sales Warehouse}

Bei dem Testfall \enquote{Sales Warehouse} handelt es sich um einen englischen Prozess aus dem Testdatensatz \enquote{Universität}. Der Prozess ist in Abbildung \ref{fig:qwen3-fall} zu sehen. Im Testfall sind vier Aktivitäten als kritisch markiert. Das Modell \texttt{Qwen3-235B-A22B-\linebreak~Thinking-2507} erkennt alle vier korrekt, markiert jedoch zusätzlich die Aktivität \enquote{Ship product} als kritisch. Die manuell festgelegten Labels ordnen das Versenden eines Produkts als unkritisch ein, da Logistikvorgänge in der Regel ohne Verarbeitung personenbezogener Daten erfolgen (vgl. Tabelle \ref{tab:labeling-examples}). \texttt{Qwen3-235B-A22B-\linebreak~Thinking-2507} begründet die Entscheidung mit der Nutzung der Kundenadresse zum Versand und zur Zustellung. Diese Begründung zeigt, dass das Modell mögliche Datenflüsse im Hintergrund berücksichtigt und daher zu einer vorsichtigeren Klassifikation gelangt. Angesichts der hohen Strafen bei übersehenen Datenschutzverstößen und des angestrebten hohen Recalls kann dieses \ac{FP} als vertretbar gelten.

\begin{figure}
    \centering
    \includegraphics[height=.41\textheight]{images/results/examples/qwen3-235B-run-3-uni-sales-warehouse}
    \caption{Ergebnis des Testfalls \enquote{Sales Warehouse} mit farblich hervorgehobenen Aktivitäten. Grün markierte Aktivitäten sind korrekt als kritisch erkannt, rot markierte stellen \acp{FP} dar.}
    \label{fig:qwen3-fall}
\end{figure}

Das Beispiel verdeutlicht eine grundsätzliche Limitierung der Klassifizierung: Fehlen in einem \ac{BPMN}-Modell explizite Informationen über Verarbeitungsschritte, ist es für das System schwierig, eine eindeutige Klassifikation vorzunehmen.

\subsection*{Marketing-Kampagne}

Im deutschen Testfall \enquote{Marketing-Kampagne}, aus dem Testdatensatz \enquote{Kleine Szenarien}, sind drei Aktivitäten als kritisch gelabelt: \enquote{Leads sammeln}, \enquote{Newsletter versenden} und \enquote{CRM aktualisieren}. \texttt{GPT‑OSS‑20B} identifiziert diese korrekt, markiert aber zusätzlich die Aktivität \enquote{Klickraten auswerten} als kritisch. Die Prozessmodellierung sah vor, dass die Klickdaten komplett anonymisiert werden und daher keine personenbezogenen Daten verarbeitet werden. Da diese Information im \ac{BPMN}-Diagramm jedoch nicht explizit hinterlegt ist, stuft das Modell die Analyse der Klickraten als potenziell personenbezogen ein und führt als Begründung die Nutzung der E‑Mail‑Adresse an. \texttt{Qwen3-235B-A22B-Thinking-2507} und einige weitere Modelle bewerteten diesen Schritt ebenfalls als kritisch, während \texttt{Mistral‑7B‑\linebreak~Instruct‑v0.3} in zwei von fünf Wiederholungen und die Gemma‑Modelle in keiner der Wiederholungen eine kritische Klassifikation vornahmen. Der Prozess inklusive farblich hervorgehobener Aktivitäten ist in Abbildung \ref{fig:gptoss-fall} zu sehen.

Dieses Beispiel zeigt, dass ohne genaue Kontextangaben zur Anonymisierung selbst scheinbar unbedenkliche Auswertungen als datenschutzrelevant erscheinen können. Es unterstreicht, dass die \acp{LLM} im Zweifel eher ein kritisches Label vergeben, um \acp{FN} zu vermeiden, wie es das Hauptziel der Klassifikation aus Abschnitt \ref{sec:qualitatsziele} vorsieht.

\begin{figure}
    \centering
    \includegraphics[width=\textwidth]{images/results/examples/oss-20b-run-1-small-marketing}
    \caption{Ergebnis des Testfalls \enquote{Marketing-Kampagne} mit farblich hervorgehobenen Aktivitäten. Die Aktivität \enquote{Klickraten auswerten} wurde als zusätzliches kritisches Element markiert.}
\label{fig:gptoss-fall}
\end{figure}

\subsection*{Karten App - Standort Erfassen}

Im Fall \enquote{Karten-App – Standort Erfassen}, ebenfalls aus dem Testdatensatz \enquote{Kleine Szenarien}, treten zwei Aktivitäten auf: \enquote{Standort erfassen} und \enquote{Route berechnen}. Beide sollten als kritisch gekennzeichnet werden, da im zweiten Schritt der zuvor erfasste Benutzerstandort zur Berechnung der Route verwendet wird. \texttt{Mistral-Large} erkennt jedoch in drei von fünf Läufen nur die erste Aktivität als kritisch und die Aktivität \enquote{Route berechnen} wird trotz der Datenassoziation nicht als kritisch eingestuft. Die Begründung des Modells erklärt zwar, dass \enquote{Standort erfassen} personenbezogene Daten verarbeitet, überträgt diese Argumentation aber nicht auf den unmittelbar folgenden Schritt. Dieses \ac{FN} ist problematisch, da es dem gewünschten hohen Recall entgegensteht und dieser Testfall zeigt, dass selbst mit vorhandenen Datenobjekten manche Modelle Schwierigkeiten haben, Datenflüsse über mehrere Aktivitäten hinweg zu erfassen. Es verdeutlicht auch, dass unterschiedliche Seeds zu unterschiedlichen Klassifikationen führen können. Der Prozess inklusive farblich hervorgehobener Aktivitäten ist in Abbildung \ref{fig:mistral-fall} zu sehen.

\begin{figure}
    \centering
    \includegraphics[width=.55\textwidth]{images/results/examples/mistral-large-run-3-small-maps-app}
    \caption{Ergebnis des Testfalls \enquote{Karten-App – Standort Erfassen} mit farblich hervorgehobenen Aktivitäten. Die Aktivität \enquote{Route berechnen} wurde fälschlicherweise nicht als kritisch markiert.}
    \label{fig:mistral-fall}
\end{figure}
\section{Beantwortung der Forschungsfragen}\label{sec:antworten-auf-forschungsfragen}

Auf Basis der vorhergehenden Auswertungen lassen sich die in Abschnitt~\ref{sec:zielsetzung-und-beitrage} formulierten Forschungsfragen beantworten. Die folgenden Antworten berücksichtigen sowohl die quantitativen Ergebnisse als auch die qualitative Beobachtungen aus den Fallstudien und ordnen sie unter Berücksichtigung der Qualitätsziele ein.

\paragraph{UF1: Wie gut schneiden europäische Modelle im Vergleich zu internationalen Modellen ab?}

Die europäischen Modelle zeigen eine große Bandbreite in ihrer Leistungsfähigkeit. \texttt{Mistral Medium 3.1} erfüllt mit einem F1-Score $= 0{,}843$, einem Recall $= 0{,}877$ und einer Precision $= 0{,}811$ sämtliche Qualitätsziele und übertrifft das Referenzmodell \texttt{GPT-4o}. \texttt{Mistral-Large-Instruct-2411} erreicht mit einem F1-Score $= 0{,}823$ ebenfalls alle Zielwerte. Dagegen schneiden \texttt{Mistral-\linebreak~7B-Instruct-v0.3} mit F1-Score $= 0{,}777$ und insbesondere \texttt{Mixtral-8x7B-\linebreak~Instruct-v0.1} mit F1-Score $= 0{,}596$ deutlich schlechter ab. Im Durchschnitt liegen die internationalen Modelle - insbesondere die Qwen- und GPT-OSS-Varianten - vor den europäischen und bieten eine robustere Balance aus Recall und Precision. Dennoch zeigen \texttt{Mistral Medium 3.1} und \texttt{Mistral-Large-Instruct-\linebreak~2411}, dass leistungsfähige europäische Alternativen existieren.

\paragraph{UF2: Wie unterscheiden sich große und kleine Modelle in ihrer Leistungsfähigkeit?}

Der direkte Vergleich zeigt, dass sich kleine ($\leq 25$B Parameter) und große Modelle kaum im Durchschnitt ihrer Metriken unterscheiden. Beide Größenklassen erreichen praktisch identische mittlere F1-Scores von etwa $0{,}80$. Ohne das Ausreißermodell \texttt{Mixtral-8x7B-Instruct-v0.1} liegt der Durchschnitt der großen Modelle mit $0{,}836$ zwar etwas höher, doch belegen Modelle wie \texttt{GPT-OSS-\linebreak~20B}, dass kleinere Modelle mit den großen mithalten können. Entscheidend sind Trainingsdaten, Feinabstimmung und Architektur, nicht allein die Parameteranzahl.

\paragraph{UF3: Welche Open-Source-Modelle (insbesondere aus der EU) erzielen die besten Ergebnisse?}

Unter den offenen Modellen dominieren die chinesischen Qwen-Varianten und die GPT-OSS-Modelle. \texttt{Qwen3-235B-A22B-Thinking-2507} erreicht mit einem F1-Score $= 0{,}874$ und einem Recall $= 0{,}932$ die Spitzenposition, gefolgt von \texttt{GPT-OSS-20B} mit F1-Score $= 0{,}866$ und Recall $= 0{,}918$ und \texttt{DeepSeek-R1-Distill-Qwen-14B} mit F1-Score $= 0{,}848$ und Precision $= 0{,}829$. Diese Modelle übertreffen die proprietären Benchmarks deutlich. Das leistungsstärkste offene \ac{EU}-Modell ist \texttt{Mistral-Large-Instruct-2411} mit F1-Score $= 0{,}823$, während \texttt{Mistral-7B-Instruct-v0.3} und \texttt{Mixtral-8x7B-Instruct-\linebreak~v0.1} die Zielwerte verfehlen.

\paragraph{UF4: Wie gut schneiden Open-Source-Modelle gegenüber kommerziellen Modellen wie GPT-4o ab?}

Mehrere Open-Source-Modelle übertreffen die kommerziellen Vertreter. \texttt{Qwen3-235B-A22B-Thinking-2507}, \texttt{GPT-OSS-20B} und\linebreak\texttt{DeepSeek-R1-Distill-Qwen-14B} erreichen höhere F1- und Recall-Werte als sowohl \texttt{GPT-4o} als auch \texttt{Mistral Medium 3.1}. \texttt{GPT-4o} überzeugt mit einer außergewöhnlich hohen Precision von $0{,}892$, verfehlt aber das Recall-Mindestziel. \texttt{Mistral Medium 3.1} bietet einen ausgewogenen Kompromiss und erfüllt alle Zielwerte, liegt aber hinter den besten Open-Source-Modellen. Insgesamt zeigen hochwertige Open-Source-Modelle die beste Balance zwischen hohem Recall und akzeptabler Precision.

Auf Basis der durchgeführten Experimente, Analysen und Antworten auf die Unterfragen lässt sich die Hauptforschungsfrage im Folgenden beantworten.

\paragraph{FF1: Wie zuverlässig identifizieren \acp{LLM} DSGVO-kritische Aktivitäten in\linebreak~BPMN-Prozessmodellen?}

Die überwiegende Mehrheit der Modelle kann kritische Aktivitäten mit hoher Zuverlässigkeit erkennen. Neun von dreizehn Modellen erreichen einen F1-Score von mindestens $0{,}80$ und erfüllen damit den Zielwert. Die Spitzenmodelle \texttt{Qwen3-235B-A22B-Thinking-2507}, \texttt{GPT-OSS-20B}, \texttt{DeepSeek-\linebreak~R1-Distill-Qwen-14B} und \texttt{Mistral Medium 3.1} erzielen F1-Scores zwischen\linebreak$0{,}843$ und $0{,}874$ bei Recall-Werten von $0{,}868$ bis $0{,}932$. Gleichzeitig gibt es Modelle wie \texttt{Mixtral-8x7B-Instruct-v0.1} und \texttt{Qwen2.5-7B-Instruct}, die deutlich abfallen.

Die Robustheitsanalyse zeigt, dass die meisten Modelle, mit einer Standardabweichung der F1-Scores von $\leq 0{,}02$, eine geringe Varianz über verschiedene Seeds aufweisen und häufig keine Retries benötigen, um eine korrekte JSON-Ausgabe zu produzieren. Ausreißer wie \texttt{Mistral Medium 3.1} (höhere Varianz) oder \texttt{Mistral-\linebreak~7B-Instruct-v0.3} (viele Retries) sollten im praktischen Einsatz sorgfältig überprüft werden.

Die Fallstudien unterstreichen, dass \ac{FP} vor allem dann entstehen, wenn im Prozessmodell wichtige Kontextinformationen fehlen, wie z.\,B.\ Anonymisierung von\linebreak~Klickraten, und Modelle daher konservativ entscheiden. \ac{FN} treten auf, wenn Datenflüsse über mehrere Aktivitäten nicht korrekt erkannt werden. Trotz dieser Fehlerbilder zeigen die Experimente, dass \acp{LLM} für ein automatisiertes Screening von \ac{BPMN}-Prozessen sehr gut geeignet sind. Eine nachgelagerte menschliche Prüfung bleibt jedoch sinnvoll, um verbleibende \ac{FP} und \ac{FN} zu adressieren und die Ergebnisse kontextsensitiv zu bewerten.
