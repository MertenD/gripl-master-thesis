\section{Motivation}\label{sec:motivation}

\begin{itemize}
    \item Geschäftsprozesse fast in allen Organisationen allgegenwärtig und bilden die Grundlage für effiziente Abläufe
    \item Außerdem ist durch die \ac{DSGVO} in Europa Datenschutz ein zentraler regulatorischer Aspekt \cite{Capodieci2023BPMNEnabledDP, GDPR2016}. Unternehmen müssen sicherstellen, dass in Prozessen personenbezogene Daten rechtskonform verarbeitet werden. Ansonsten drohen Strafen von bis zu 20 Millionen Euro oder 4\% des weltweiten gesamten erzielten Jahresumsatzes \cite{GDPR2016}.
    \item Die Überprüfung von Prozessen auf konformität in Bezug auf Datenschutz ist jedoch zeit- und kostenintensiv \cite{nake2023towards, varela2025business}. Besonders in großen Organisationen mit hunderten von Prozessen parallel ist eine manuelle Analyse nicht wirklich praktikabel und fehleranfällig. Fehlerhafte Unterekennung datenschutzkritischer Aktivitäten, also False Negatives, kann weitreichende Folgen haben - von Reputationsschäden bis hin zu hohen Bußgeldern \cite{nake2023towards}.
    \item \acp{LLM} versprechen genau an dieser Stelle eine Lösung, indem sie ein schnelles automatisiertes Screening von Prozessmodellen ermöglichen. Erste Arbeiten zeigen das Potenzial von \acp{LLM}, etwa bei der Identifikation datenschutzrelevanter Verarbeitungstätigkeiten oder in der Analyse von Datenschutzerklärungen \cite{ciaramella2022leveraging, pragyan2024toward}. Besonders interessant sind dabei europäische Open-Source Modelle wie die von Mistral \cite{mistralai}. Sie sind zum einen frei verfügbar und transparent und zum anderen sind sie bislang noch kaum in Hinblick auf \ac{DSGVO}-bezogene Aufgaben evaluiert worden. Bisher fehlen belastbare empirische Vergleiche und reproduzierbare Evidenz, die eine fundierte Bewertung dieser Modelle erlauben würden \cite{schwerin2024systematic}.
\end{itemize}