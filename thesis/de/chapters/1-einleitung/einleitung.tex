\chapter{Einleitung}\label{ch:einleitung}

Geschäftsprozesse sind in nahezu allen Organisationen allgegenwärtig und bilden die Grundlage für effiziente Abläufe. Zugleich ist in Europa durch die \ac{DSGVO} der Datenschutz zu einem zentralen regulatorischen Aspekt geworden \cite{Capodieci2023BPMNEnabledDP, GDPR2016}. Unternehmen müssen sicherstellen, dass in ihren Prozessen personenbezogene Daten rechtskonform verarbeitet werden; andernfalls drohen Strafen von bis zu 20 Millionen Euro oder 4\% des weltweit gesamt erzielten Jahresumsatzes \cite{GDPR2016}.


Die Überprüfung von Prozessen auf Konformität in Bezug auf Datenschutz ist jedoch zeit- und kostenintensiv \cite{nake2023towards, varela2025business}. Besonders in großen Organisationen mit hunderten parallel laufenden Prozessen ist eine manuelle Analyse kaum praktikabel und zudem fehleranfällig. Fehlerhafte Untererkennungen datenschutzkritischer Aktivitäten, sogenannte \ac{FN}, können weitreichende Folgen haben - von Reputationsschäden bis hin zu hohen Bußgeldern \cite{nake2023towards}.


Vor diesem Hintergrund rücken \acp{LLM} als aufstrebende Technologie im Bereich \ac{KI} in den Fokus. Sie sind darauf trainiert, natürliche Sprache auch in langen und komplexen Texten zu verstehen, Zusammenhänge über große Kontexte hinweg zu erkennen und Anweisungen zu befolgen. Damit erscheinen \acp{LLM} als vielversprechender Ansatz für das automatisierte Screening von Prozessmodellen. Erste Arbeiten belegen dieses Potenzial, etwa bei der Identifikation datenschutzrelevanter Verarbeitungstätigkeiten oder in der Analyse von Datenschutzerklärungen \cite{ciaramella2022leveraging, pragyan2024toward}.


Besonders interessant sind in diesem Kontext europäische Open-Source-Modelle wie die von Mistral \cite{mistralai}. Sie sind zum einen frei verfügbar und transparent, zum anderen wurden sie bislang kaum im Hinblick auf \ac{DSGVO}-bezogene Aufgaben evaluiert. Es fehlen belastbare, reproduzierbare empirische Vergleiche, die eine fundierte Bewertung dieser Modelle erlauben würden \cite{schwerin2024systematic}.

\section{Problemstellung}\label{sec:problemstellung}

Trotz der genannten Potenziale fehlt es bisher an standardisierten, reproduzierbaren Vergleichen verschiedener Modelle für die konkrete Aufgabe Aktivitäten in Geschäftsprozessen nach \enquote{kritisch} und \enquote{unkritisch} zu klassifizieren. Erste Ansätze, wie z.\,B. der von Nake et al. \cite{nake2023towards}, zeigen dass ML Ansätze grundsätzlich in der Lage sind \ac{DSGVO}-kritische Aktivitäten in textuellen Prozessbeschreibungen zu erkennen; dennoch existieren keine einheitlichen Benchmarks, die einen systematischen Vergleich unterschiedlicher \acp{LLM} erlauben.

Auch von Schwerin et al. \cite{schwerin2024systematic} heben hervor, dass trotz großer Fortschritte im Einsatz von \acp{LLM} für juristische Aufgaben bislang erhebliche Lücken in der Evaluation für compliance-spezifische Anwendungen bestehen und geeignete \ac{DSGVO}-spezifische Benchmarks fehlen. Somit mangelt es derzeit an einer belastbaren empirischen Grundlage, um Modelle zuverlässig und vergleichbar zu bewerten.

Besonders interessant ist die Frage, wie sich Open-Source-Modelle - insbesondere mit Ursprung aus der \ac{EU} - im Vergleich zu internationalen außerhalb der \ac{EU} entwickelten Modellen schlagen und welche Trade-offs dabei entstehen \cite{schwerin2024systematic}. Diese Perspektive ist nicht nur aus Leistungs-, sondern auch aus Transparenz- und Regulierungsgründen relevant.

Eine zusätzliche Herausforderung ergibt sich aus der Natur von \ac{BPMN}-Modellen: Typischerweise konzentrieren sie sich auf den Kontrollfluss und vernachlässigen die Datenebene. Datenobjekte werden oftmals gar nicht explizit modelliert oder nur implizit in den Aktivitäten referenziert. Dadurch ist die Datennutzung von Aktivitäten nicht direkt erkennbar und muss aus textuellen Beschreibungen und dem Kontext erschlossen werden \cite{schneid2021uncovering}. Das erschwert die automatische Identifikation von \ac{DSGVO}-kritischen Aktivitäten, da Algorithmen personenbezogene Datenflüsse zunächst indirekt und über den Kontext ableiten müssen.
\section{Zielsetzung und Beiträge}\label{sec:zielsetzung-und-beitrage}

Ziel der Arbeit ist es, einen methodischen Beitrag zur automatisierten Identifikation von \ac{DSGVO}-kritischen Aktivitäten in Geschäftsprozessen zu leisten. Hierfür werden folgende Beiträge angestrebt:

\begin{itemize}
    \item Entwicklung einer Klassifizierungspipeline für Geschäftsprozesse, die Aktivitäten binär in datenschutzkritisch oder unkritisch einordnet.
    \item Konzeption und Umsetzung eines Evaluationsframeworks, das reproduzierbare Vergleiche verschiedener \acp{LLM} und Algorithmen über eine einheitliche Schnittstelle ermöglicht.
    \item Entwicklung einer Labelingsoftware zur Erstellung und Annotation von Datensätzen für das Evaluationsframework.
    \item Aufbau eines repräsentativen Datensatzes aus gelabelten \ac{BPMN}-Prozessen, inklusive klar definierter Labeling-Kriterien.
    \item Bereitstellung überprüfbarer empirischer Befunde, inklusive Code, Konfigurationen der Experimente und Seeds, um Nachvollziehbarkeit und Reproduzierbarkeit zu gewährleisten.
\end{itemize}
\section{Aufbau der Arbeit}\label{sec:aufbau-der-arbeit}

Die Arbeit ist wie folgt gegliedert: Kapitel 2 gibt einen Überblick über den theoretischen Hintergrund, die \ac{DSGVO} und \ac{BPMN} sowie eine Einführung in \acp{LLM} und verwandte Arbeiten. Kapitel 3 beschreibt den Rahmen der Entwicklung der Klassifizierungspipeline, des Evaluationsframeworks und der Experimente. Kapitel 4 stellt den entwickelten Algorithmus zur Klassifikation von \ac{BPMN}-Modellen und dessen einheitliche Schnittstelle vor. Kapitel 5 präsentiert die Labeling-Software und erläutert die Erstellung der Datensätze. Anschließend werden in Kapitel 6 die Architektur und der Funktionsumfang der Evaluationspipeline beschrieben. Kapitel 7 zeigt, wie die Auswahl der \acp{LLM} erfolgte. Kapitel 8 erläutert den Versuchsaufbau und die Durchführung der Experimente, Kapitel 9 stellt die Ergebnisse vor und Kapitel 10 diskutiert diese im Kontext der Forschungsfragen. Zum Schluss fasst Kapitel 11 die Arbeit zusammen und gibt einen Ausblick auf mögliche zukünftige Forschungsthemen.