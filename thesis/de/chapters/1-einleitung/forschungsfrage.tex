\section{Forschungsfrage und Unterfragen}\label{sec:forschungsfrage-und-unterfragen}

Die zentrale Forschungsfrage dieser Arbeit lautet:

\begin{description}
    \item[\textbf{FF1}] Wie zuverlässig identifizieren \acp{LLM} \ac{DSGVO}-kritische Aktivitäten in \ac{BPMN}-Prozessmodellen?
\end{description}

Um diese Frage differenziert beantworten zu können werden außerdem folgende Unterfragen betrachtet:

\begin{description}
    \item[\textbf{UF1}] Wie gut schneiden europäische Open-Source-Modelle im Vergleich zu internationalen Modellen ab?
    \item[\textbf{UF2}] Wie unterscheiden sich große und kleine Modelle in ihrer Leistungsfähigkeit?
    \item[\textbf{UF3}] Welche Open-Source-Modelle (insbesondere aus der EU) erzielen die besten Ergebnisse?
    \item[\textbf{UF4}] Wie gut schneiden Open-Source-Modelle gegenüber kommerziellen Modellen wie GPT-4o ab?
\end{description}

Für ein initiales Screening reicht, wie in \cite{nake2023towards}, eine binäre Klassifikation (kritisch vs. unkritisch). Eine tiefergehende rechtliche Prüfung kann in einem nachfolgendem Schritt durchgeführt werden und ist nicht Bestandteil dieser Arbeit.