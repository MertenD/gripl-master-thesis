\section{Zielsetzung und Beiträge}\label{sec:zielsetzung-und-beitrage}

Ziel der Arbeit ist es, einen methodischen Beitrag zur automatisierten Identifikation von \ac{DSGVO}-kritischen Aktivitäten in Geschäftsprozessen zu leisten. Hierfür werden folgende Beiträge angestrebt:

\begin{itemize}
    \item Entwicklung einer Klassifizierungspipeline für Geschäftsprozesse, die Aktivitäten binär in datenschutzkritisch oder unkritisch einordnet.
    \item Konzeption und Umsetzung eines Evaluationsframeworks, das reproduzierbare Vergleiche verschiedener \acp{LLM} und Algorithmen über eine einheitliche Schnittstelle ermöglicht.
    \item Entwicklung einer Labeling-Software zur Erstellung und Annotation von Datensätzen für das Evaluationsframework.
    \item Aufbau eines repräsentativen Datensatzes aus gelabelten \ac{BPMN}-Prozessen, inklusive klar definierter Labeling-Kriterien.
    \item Bereitstellung überprüfbarer empirischer Befunde, inklusive Code, Konfigurationen der Experimente und Seeds, um Nachvollziehbarkeit und Reproduzierbarkeit zu gewährleisten.
\end{itemize}

Auf dieser Grundlage ergibt sich die zentrale Forschungsfrage dieser Arbeit:

\begin{description}
    \item[\textbf{FF1}] Wie zuverlässig identifizieren \acp{LLM} \ac{DSGVO}-kritische Aktivitäten in \ac{BPMN}-Prozessmodellen?
\end{description}

Um diese Frage differenziert beantworten zu können, werden außerdem folgende Unterfragen betrachtet:

\begin{description}
    \item[\textbf{UF1}] Wie gut schneiden europäische Modelle im Vergleich zu internationalen Modellen ab?
    \item[\textbf{UF2}] Wie unterscheiden sich große und kleine Modelle in ihrer Leistungsfähigkeit?
    \item[\textbf{UF3}] Welche Open-Source-Modelle (insbesondere aus der \ac{EU}) erzielen die besten Ergebnisse?
    \item[\textbf{UF4}] Wie gut schneiden Open-Source-Modelle gegenüber kommerziellen Modellen wie \texttt{GPT-4o} ab?
\end{description}

Für ein initiales Screening reicht, wie in \cite{nake2023towards}, eine binäre Klassifikation (kritisch vs. unkritisch). Eine tiefergehende rechtliche Prüfung kann in einem nachfolgenden Schritt durchgeführt werden und ist nicht Bestandteil dieser Arbeit.