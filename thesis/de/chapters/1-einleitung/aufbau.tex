\section{Aufbau der Arbeit}\label{sec:aufbau-der-arbeit}

Die Arbeit ist wie folgt gegliedert: Kapitel 2 gibt einen Überblick über den theoretischen Hintergrund, die \ac{DSGVO} und \ac{BPMN} sowie eine Einführung in \acp{LLM} und verwandte Arbeiten. Kapitel 3 beschreibt den Rahmen der Entwicklung der Klassifizierungspipeline, des Evaluationsframeworks und der Experimente. Kapitel 4 stellt den entwickelten Algorithmus zur Klassifikation von \ac{BPMN}-Modellen und dessen einheitliche Schnittstelle vor. Kapitel 5 präsentiert die Labeling-Software und erläutert die Erstellung der Datensätze. Anschließend werden in Kapitel 6 die Architektur und der Funktionsumfang der Evaluationspipeline beschrieben. Kapitel 7 zeigt, wie die Auswahl der \acp{LLM} erfolgte. Kapitel 8 erläutert den Versuchsaufbau und die Durchführung der Experimente, Kapitel 9 stellt die Ergebnisse vor und Kapitel 10 diskutiert diese im Kontext der Forschungsfragen. Zum Schluss fasst Kapitel 11 die Arbeit zusammen und gibt einen Ausblick auf mögliche zukünftige Forschungsthemen.