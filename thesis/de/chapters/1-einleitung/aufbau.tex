\section{Aufbau der Arbeit}\label{sec:aufbau-der-arbeit}

Die Arbeit ist wie folgt gegliedert: Kapitel 2 gibt einen Überblick über den theoretischen Hintergrund, die \ac{DSGVO} und \ac{BPMN} sowie eine Einführung in \acp{LLM} und verwandte Arbeiten. Kapitel 3 beschreibt den Rahmen der Entwicklung der Klassifizierungspipeline, des Evaluationsframeworks und der Experimente. Kapitel 4 stellt den entwickelten Algorithmus zur Klassifikation von \ac{BPMN}-Modellen und dessen einheitliche Schnittstelle vor. Kapitel 5 präsentiert die Architektur und den Funktionsumfang der Evaluationspipeline. Anschließend werden in Kapitel 6 die Labeling-Software und die Erstellung der Datensätze erläutert. Kapitel 7 zeigt, wie die Auswahl der \acp{LLM} erfolgte. Kapitel 8 erläutert den Versuchsaufbau, Kapitel 9 die Durchführung der Experimente und Kapitel 10 stellt die Ergebnisse vor. In Kapitel 11 werden die Ergebnisse im Kontext der Forschungsfragen diskutiert. Zum Schluss fasst Kapitel 12 die Arbeit zusammen und Kapitel 13 gibt einen Ausblick auf mögliche zukünftige Forschungsthemen.