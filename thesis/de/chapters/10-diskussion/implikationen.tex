\section{Implikationen für Praxis und Forschung}\label{sec:implikationen-fur-praxis-und-forschung}

Für die Praxis bieten die Ergebnisse dieser Arbeit eine fundierte Grundlage für den Einsatz von \acp{LLM} zur automatisierten Identifikation \ac{DSGVO}-kritischer Aktivitäten in \ac{BPMN}-Prozessen. Unternehmen können auf Basis der aggregierten Ergebnisse und identifizierten Spitzenmodelle wie \texttt{Qwen3-235B-A22B-Thinking-2507} und \texttt{GPT-OSS-20B} fundierte Entscheidungen darüber treffen, welche Modelle für ihr Risikomanagement am besten geeignet sind. Dabei sollten sie die spezifischen Anforderungen an Recall und Precision sowie betriebliche Rahmenbedingungen wie Hosting und Kosten berücksichtigen. Ebenfalls relevant ist der Betriebsstandort der Modelle im Hinblick auf die europäische Datenschutzgesetzgebung, falls personenbezogene Daten verarbeitet werden. Kleinere Modelle, die On-Premises betrieben werden können, oder Anbieter wie Mistral AI mit Sitz in der \ac{EU} bieten hier Vorteile.

Die entwickelte Klassifizierungspipeline stellt eine bewährte Methodik dar, die in produktiven Systemen implementiert werden kann, um die Zuverlässigkeit der Modellausgaben zu gewährleisten. Modellierungsseitig helfen explizite Datenflüsse und dokumentierte Annahmen, die in Abschnitt~\ref{sec:fallstudien} beobachteten \ac{FP}- und \ac{FN}-Muster zu verringern.

Für die Forschung lassen sich drei Schwerpunkte ableiten: Erstens sollte auf der in Abschnitt~\ref{sec:bpmn-preprocessing} angelegten Vorverarbeitung aufgebaut und die Extraktion von Graph-Features systematisch erweitert werden, um die Kontextverfolgung über Aktivitätsketten zu verbessern. Zweitens versprechen wissensgestützte Komponenten - etwa \ac{DSGVO}-\ac{RAG} sowie nachgelagerte Entailment- und Konsistenzprüfungen – präzisere Entscheidungen in unklaren Fällen. Drittens ist eine Erweiterung des Benchmarks über Domänen, Sprachen und Modellierungsstile sowie ein Mehrannotator-Labeling interessant, um die Aussagekraft und Generalisierbarkeit der Ergebnisse zu erhöhen. Besonders Annotationen von mehreren Experten würden subjektive Interpretationen minimieren und die Validität der Testdaten stärken.