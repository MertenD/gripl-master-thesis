\chapter{Versuchsaufbau und Durchführung}\label{ch:versuchsaufbau-und-durchfuhrung}

Wie in Abschnitt \ref{sec:experimentdesign} beschrieben, soll eine fairer Vergleich verschiedener \acp{LLM} erreicht werden. Dazu werden alle, der im vorherigen Kapitel beschriebenen, Modelle durch dieselbe Klassifikationspipeline geschickt und anhand der im Kapitel \ref{sec:qualitatsziele} definierten Metriken (Accuracy, Precision, Recall, F1) bewertet. Außerdem wird betrachtet wie viele Testfälle erfolgreich klassifiziert wurden und wie robust die Modelle sind. Dieses Kapitel beschreibt den konkreten Versuchsaufbau und die Durchführung der Experimente. Die hier dokumentierten Parameter und Konfigurationen sind wesentlich, um die Ergebnisse nachvollziehbar und reproduzierbar zu machen.

Um sowohl kleine als auch große Modelle testen zu können, wurde \emph{OpenRouter} \cite{openrouter} als API-Anbieter genutzt. Über diese Cloud-basierte Schnittstelle lassen sich auch Modelle ausführen, die lokal aufgrund begrenzter Hardware nicht betrieben werden können. Der API-Schlüssel wird über eine Umgebungsvariable in die Konfigurationsdatei eingebunden, um sensible Daten aus den Konfigurationen fernzuhalten.

In den Experimenten wurden mehrere Modelle aus unterschiedlichen Anbieterfamilien getestet. Für jeden Anbieter gibt es ein eigenes Experiment, in dem mehrere Modellgrößen (z.\,B. 7B, 8x7B, Large) gegeneinander verglichen werden. Da alle Experimente die gleiche Pipeline und die gleichen Datensätze verwenden, können auch die Ergebnisse verschiedener Anbieter untereinander verglichen werden. Diese Aufteilung in verschiedene Experimente dient lediglich der Übersichtlichkeit in der Benutzeroberfläche des Evaluationsframeworks.

\section{Vergleichbarkeit}\label{sec:einheitliche-klassifizierungspipeline-und-datensatze}

Um die Vergleichbarkeit der Experimente zu gewährleisten, werden alle Modelle durch dieselbe Klassifikationspipeline geschickt. Die technische Implementierung dieser Pipeline wurde in Kapitel \ref{ch:klassifizierungsalgorithmus-(design-und-implementierung)} beschrieben und kann im Evaluationsframework genutzt werden. Jeder Testfall besteht aus einem \ac{BPMN}-Prozess mit Labeln für \ac{DSGVO}-kritische Aktivitäten. Ein Testfall gilt als korrekt klassifiziert, wenn genau die als kritisch gelabelten Aktivitäten auch als kritisch erkannt werden - bereits ein \ac{FP} oder \ac{FN} führt zu einem nicht bestandenen Testfall.

Als Datenbasis kommen drei im Labeling-Tool erzeugte Testdatensätze zum Einsatz. Diese decken unterschiedliche Prozesskontexte ab und werden in den Experimenten mit den \texttt{ids} \emph{1 (kleine Prozesse)}, \emph{2 (Universität)} und \emph{7 (mittelgroße Praxisbeispiele} referenziert. Für jedes Experiment werden alle verfügbaren Datensätze verwendet. Auf diese Weise können Unterschiede zwischen den Modellen nicht auf unterschiedliche Datenquellen zurückgeführt werden.
\section{Konfigurationen}\label{sec:konfigurationen}

Die Konfigurationen der Experimente sind im YAML-Format in Listings \ref{lst:mistral-experiment-config}, \ref{lst:gemma-experiment-config}, \ref{lst:deepseek-experiment-config}, \ref{lst:qwen-experiment-config} und \ref{lst:gpt-experiment-config} dargestellt. Sie enthalten die zu evaluierenden Modelle, die zu verwendenden Datensätze, den Basis-Seed sowie die Anzahl der Wiederholungen und weitere Rahmenparameter. In Listing \ref{lst:mistral-experiment-config} wird ein Experiment dargestellt, in dem vier verschiedene Mistral-Modelle über OpenRouter evaluiert werden. Die Datensätze werden jeweils fünf Mal durchlaufen. Der Basis-Seed ist auf \texttt{24523833} gesetzt.

\begin{lstlisting}[float, caption={Konfigurationsdatei des Experiments mit Mistral Modellen}, label={lst:mistral-experiment-config}]
defaultEvaluationEndpoint: /gdpr/analysis/prompt-engineering
seed: 24523833
maxConcurrent: 10
repetitions: 5
models:
  - label: Mistral-7B-Instruct-v0.3
    llmProps:
      baseUrl: https://openrouter.ai/api/v1
      modelName: mistralai/mistral-7b-instruct-v0.3
      apiKey: ${OPEN_ROUTER_API_KEY}
      temperature: 0.1
      topP: 1
  - label: Mistral-8x7B-Instruct-v0.1
    llmProps:
      baseUrl: https://openrouter.ai/api/v1
      modelName: mistralai/mixtral-8x7b-instruct
      apiKey: ${OPEN_ROUTER_API_KEY}
      temperature: 0.1
      topP: 1
  - label: Mistral-Large-Instruct-2411
    llmProps:
      baseUrl: https://openrouter.ai/api/v1
      modelName: mistralai/mistral-large-2411
      apiKey: ${OPEN_ROUTER_API_KEY}
      temperature: 0.1
      topP: 1
  - label: Mistral Medium 3.1
    llmProps:
      baseUrl: https://openrouter.ai/api/v1
      modelName: mistralai/mistral-medium-3.1
      apiKey: ${OPEN_ROUTER_API_KEY}
      temperature: 0.1
      topP: 1
datasets:
  - 2
  - 7
  - 1
\end{lstlisting}

Auf Basis des Seeds aus der Konfiguration und der aktuellen Wiederholungsnummer wird in dem Evaluationsframework für jede Wiederholung deterministisch ein neuer Seed generiert. Dadurch sind die Ergebnisse reproduzierbar und dennoch wird die Stabilität der Modelle über mehrere Wiederholungen mit unterschiedlichen Seeds abgebildet. Alle Datensätze, Konfigurationen und die daraus resultierenden Ergebnisse sind außerdem im GitLab‑Repository verfügbar\footnote{Siehe das GitLab Repository: \hyperlink{// TODO}{// TODO}}.

Um bei der Zero‑Shot‑Klassifikation deterministische und formatkonsistente Ergebnisse zu erzielen, wurden die Inferenz‑Hyperparameter \texttt{temperature} und \texttt{topP} bewusst konservativ gewählt. Der Parameter \texttt{temperature} steuert die Zufälligkeit der Modellausgabe: niedrige Werte priorisieren die wahrscheinlichsten Tokens und machen die Ausgabe deterministischer. Für Aufgaben, die faktische Genauigkeit und Precision erfordern, empfehlen aktuelle Arbeiten daher sehr niedrige Temperaturen; höhere Werte (z.\,B.\ \texttt{temperature}=0{,}8 oder 2) verschlechtern hingegen die Klassifikationsleistung und führen zu nicht‑reproduzierbaren Ausgaben \cite{renze2024effect,mu2024navigating}. In dieser Arbeit wird durchgängig \texttt{temperature}=0{,}1 verwendet. Dieser Wert reduziert Zufallseffekte deutlich, ohne den Output übermäßig einzuschränken, und folgt den Empfehlungen vergleichbarer Studien zur Zero-Shot-Klassifikation \cite{mu2024navigating}. Auf \texttt{temperature}=0 wurde bewusst verzichtet: Zwar wäre die Ausgabe damit theoretisch noch deterministischer, in eigenen Tests traten jedoch vermehrt Formatfehler auf (z.\,B. fehlende oder zusätzliche Zeichen im JSON-Output). \texttt{temperature}=0{,}1 stellt daher einen guten Kompromiss zwischen Determinismus und Formatstabilität dar.

Der Parameter \texttt{topP} steuert, wie \enquote{eng} die Auswahl für das nächste Token gefasst wird. Dafür werden die möglichen Fortsetzungen nach ihrer Wahrscheinlichkeit sortiert und nur so viele der wahrscheinlichsten genommen, bis zusammen etwa \texttt{topP} erreicht ist (z.\,B. 0{,}9 $\widehat{=}$ 90\%). Aus diesen wird dann gewählt. Beispiel: Bei \enquote{Der Himmel ist \dots} stehen \enquote{blau}, \enquote{bewölkt} oder \enquote{klar} meist weit oben. Mit \texttt{topP}=0{,}9 bleiben diese Kandidaten im Rennen, während seltene oder unpassende Fortsetzungen (\enquote{gestern}, \enquote{eine}) ignoriert werden. Bei \texttt{topP}=0{,}2 bleibt oft fast nur \enquote{blau} übrig, weil es das Wahrscheinlichste ist. Mit \texttt{topP}=1 wird nichts vorab ausgeschlossen - dann bestimmt allein die \texttt{temperature} den Zufallsanteil \cite{renze2024effect}. In Kombination mit sehr niedriger \texttt{temperature} liefert \texttt{topP}=1 fokussierte, weitgehend deterministische Ausgaben bei gleichzeitig maximalem Stichprobenraum \cite{mu2024navigating}.
\section{Durchführung}\label{sec:durchfuhrung}

Die Durchführung der Experimente erfolgt automatisiert über das Evaluationsframework.  Für jede in der Konfigurationsdatei angegebene Modellvariante werden alle Testfälle aus den ausgewählten Datensätzen an die Klassifikationspipeline übergeben. Während der Ausführung werden für jeden Testfall die Einzelergebnisse der Konfusionsmatrix sowie der Status \enquote{bestanden} oder \enquote{nicht bestanden} bestimmt. Diese Kennzahlen werden pro Modell aggregiert und anschließend genutzt, um die aus Kapitel \ref{sec:qualitatsziele} bekannten Metriken zu berechnet.

Für jedes Modell werden außerdem über alle Wiederholungen hinweg sowohl die Durchschnittswerte als auch dessen Standardabweichung für die Metriken berechnet. Die Standardabweichung gibt an, wie stark die Ergebnisse der einzelnen Läufe um den Mittelwert streuen. Ein niedriger Wert deutet auf eine hohe Stabilität des Modells hin, während ein hoher Wert auf eine größere Variabilität in den Ergebnissen hinweist. Diese Information ist besonders wichtig, um die Zuverlässigkeit der Modelle zu bewerten, da einige \acp{LLM} aufgrund ihrer nicht-deterministischen Natur unterschiedliche Ergebnisse bei wiederholten Ausführungen desselben Testfalls liefern können.

Im nächsten Kapitel werden die erzielten Ergebnisse dieser Exerimente detailliert vorgestellt und analysiert.
