\section{Durchführung}\label{sec:durchfuhrung}

Die Durchführung der Experimente erfolgt automatisiert über das Evaluationsframework.  Für jede in der Konfigurationsdatei angegebene Modellvariante werden alle Testfälle aus den ausgewählten Datensätzen an die Klassifikationspipeline übergeben. Während der Ausführung werden für jeden Testfall die Einzelergebnisse der Konfusionsmatrix sowie der Status \enquote{bestanden} oder \enquote{nicht bestanden} bestimmt. Diese Kennzahlen werden pro Modell aggregiert und anschließend genutzt, um die aus Kapitel \ref{sec:qualitatsziele} bekannten Metriken zu berechnen.

Für jedes Modell werden außerdem über alle Wiederholungen hinweg sowohl die Durchschnittswerte als auch dessen Standardabweichung für die Metriken berechnet. Die Standardabweichung gibt an, wie stark die Ergebnisse der einzelnen Läufe um den Mittelwert streuen. Ein niedriger Wert deutet auf eine hohe Stabilität des Modells hin, während ein hoher Wert auf eine größere Variabilität in den Ergebnissen hinweist. Diese Information ist besonders wichtig, um die Zuverlässigkeit der Modelle zu bewerten, da einige \acp{LLM} aufgrund ihrer nicht-deterministischen Natur unterschiedliche Ergebnisse bei wiederholten Ausführungen desselben Testfalls liefern können.

Im nächsten Kapitel werden die erzielten Ergebnisse dieser Experimente detailliert vorgestellt und analysiert.