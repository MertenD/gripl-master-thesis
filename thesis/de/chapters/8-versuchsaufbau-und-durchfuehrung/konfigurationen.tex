\section{Konfigurationen}\label{sec:konfigurationen}

Die Konfigurationen der Experimente sind im YAML-Format in Listings \ref{lst:mistral-experiment-config}, \ref{lst:gemma-experiment-config}, \ref{lst:deepseek-experiment-config}, \ref{lst:qwen-experiment-config} und \ref{lst:gpt-experiment-config} dargestellt. Sie enthalten die zu evaluierenden Modelle, die zu verwendenden Datensätze, den Basis-Seed sowie die Anzahl der Wiederholungen und weitere Rahmenparameter. In Listing \ref{lst:mistral-experiment-config} wird ein Experiment dargestellt, in dem vier verschiedene Mistral-Modelle über OpenRouter evaluiert werden. Die Datensätze werden jeweils fünf Mal durchlaufen. Der Basis-Seed ist auf \texttt{24523833} gesetzt.

\begin{lstlisting}[float, caption={Konfigurationsdatei des Experiments mit Mistral Modellen}, label={lst:mistral-experiment-config}]
defaultEvaluationEndpoint: /gdpr/analysis/prompt-engineering
seed: 24523833
maxConcurrent: 10
repetitions: 5
models:
  - label: Mistral-7B-Instruct-v0.3
    llmProps:
      baseUrl: https://openrouter.ai/api/v1
      modelName: mistralai/mistral-7b-instruct-v0.3
      apiKey: ${OPEN_ROUTER_API_KEY}
      temperature: 0.1
      topP: 1
  - label: Mistral-8x7B-Instruct-v0.1
    llmProps:
      baseUrl: https://openrouter.ai/api/v1
      modelName: mistralai/mixtral-8x7b-instruct
      apiKey: ${OPEN_ROUTER_API_KEY}
      temperature: 0.1
      topP: 1
  - label: Mistral-Large-Instruct-2411
    llmProps:
      baseUrl: https://openrouter.ai/api/v1
      modelName: mistralai/mistral-large-2411
      apiKey: ${OPEN_ROUTER_API_KEY}
      temperature: 0.1
      topP: 1
  - label: Mistral Medium 3.1
    llmProps:
      baseUrl: https://openrouter.ai/api/v1
      modelName: mistralai/mistral-medium-3.1
      apiKey: ${OPEN_ROUTER_API_KEY}
      temperature: 0.1
      topP: 1
datasets:
  - 2
  - 7
  - 1
\end{lstlisting}

Auf Basis des Seeds aus der Konfiguration und der aktuellen Wiederholungsnummer wird in dem Evaluationsframework für jede Wiederholung deterministisch ein neuer Seed generiert. Dadurch sind die Ergebnisse reproduzierbar und dennoch wird die Stabilität der Modelle über mehrere Wiederholungen mit unterschiedlichen Seeds abgebildet. Alle Datensätze, Konfigurationen und die daraus resultierenden Ergebnisse sind außerdem im GitLab‑Repository verfügbar\footnote{Siehe das GitLab Repository: \hyperlink{// TODO}{// TODO}}.

Um bei der Zero‑Shot‑Klassifikation deterministische und formatkonsistente Ergebnisse zu erzielen, wurden die Inferenz‑Hyperparameter \texttt{temperature} und \texttt{topP} bewusst konservativ gewählt. Der Parameter \texttt{temperature} steuert die Zufälligkeit der Modellausgabe: niedrige Werte priorisieren die wahrscheinlichsten Tokens und machen die Ausgabe deterministischer. Für Aufgaben, die faktische Genauigkeit und Precision erfordern, empfehlen aktuelle Arbeiten daher sehr niedrige Temperaturen; höhere Werte (z.\,B.\ \texttt{temperature}=0{,}8 oder 2) verschlechtern hingegen die Klassifikationsleistung und führen zu nicht‑reproduzierbaren Ausgaben \cite{renze2024effect,mu2024navigating}. In dieser Arbeit wird durchgängig \texttt{temperature}=0{,}1 verwendet. Dieser Wert reduziert Zufallseffekte deutlich, ohne den Output übermäßig einzuschränken, und folgt den Empfehlungen vergleichbarer Studien zur Zero-Shot-Klassifikation \cite{mu2024navigating}. Auf \texttt{temperature}=0 wurde bewusst verzichtet: Zwar wäre die Ausgabe damit theoretisch noch deterministischer, in eigenen Tests traten jedoch vermehrt Formatfehler auf (z.\,B. fehlende oder zusätzliche Zeichen im JSON-Output). \texttt{temperature}=0{,}1 stellt daher einen guten Kompromiss zwischen Determinismus und Formatstabilität dar.

Der Parameter \texttt{topP} steuert, wie \enquote{eng} die Auswahl für das nächste Token gefasst wird. Dafür werden die möglichen Fortsetzungen nach ihrer Wahrscheinlichkeit sortiert und nur so viele der wahrscheinlichsten genommen, bis zusammen etwa \texttt{topP} erreicht ist (z.\,B. 0{,}9 $\widehat{=}$ 90\%). Aus diesen wird dann gewählt. Beispiel: Bei \enquote{Der Himmel ist \dots} stehen \enquote{blau}, \enquote{bewölkt} oder \enquote{klar} meist weit oben. Mit \texttt{topP}=0{,}9 bleiben diese Kandidaten im Rennen, während seltene oder unpassende Fortsetzungen (\enquote{gestern}, \enquote{eine}) ignoriert werden. Bei \texttt{topP}=0{,}2 bleibt oft fast nur \enquote{blau} übrig, weil es das Wahrscheinlichste ist. Mit \texttt{topP}=1 wird nichts vorab ausgeschlossen - dann bestimmt allein die \texttt{temperature} den Zufallsanteil \cite{renze2024effect}. In Kombination mit sehr niedriger \texttt{temperature} liefert \texttt{topP}=1 fokussierte, weitgehend deterministische Ausgaben bei gleichzeitig maximalem Stichprobenraum \cite{mu2024navigating}.