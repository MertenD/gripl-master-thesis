\section{Vergleichbarkeit}\label{sec:einheitliche-klassifizierungspipeline-und-datensatze}

Um die Vergleichbarkeit der Experimente zu gewährleisten, werden alle Modelle durch dieselbe Klassifikationspipeline geschickt. Die technische Implementierung dieser Pipeline wurde in Kapitel \ref{ch:klassifizierungsalgorithmus-(design-und-implementierung)} beschrieben und kann im Evaluationsframework genutzt werden. Jeder Testfall besteht aus einem \ac{BPMN}-Prozess mit Labeln für \ac{DSGVO}-kritische Aktivitäten. Ein Testfall gilt als korrekt klassifiziert, wenn genau die als kritisch gelabelten Aktivitäten auch als kritisch erkannt werden - bereits ein \ac{FP} oder \ac{FN} führt zu einem nicht bestandenen Testfall.

Als Datenbasis kommen drei im Labeling-Tool erzeugte Testdatensätze zum Einsatz. Diese decken unterschiedliche Prozesskontexte ab und werden in den Experimenten mit den \texttt{id}s \emph{1 (kleine Prozesse)}, \emph{2 (Universität)} und \emph{7 (mittelgroße Praxisbeispiele)} referenziert. Für jedes Experiment werden alle verfügbaren Datensätze verwendet. Auf diese Weise können Unterschiede zwischen den Modellen nicht auf unterschiedliche Datenquellen zurückgeführt werden.