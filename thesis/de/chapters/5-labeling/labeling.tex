\chapter{Labeling und Datensätze}\label{ch:labeling-und-datensatze}

Für die Evaluation der Klassifikation ist es erforderlich, zunächst geeignete Testdatensätze mit Annotationen bereitzustellen. Ein Datensatz umfasst mehrere Testfälle, die jeweils aus einem \ac{BPMN}-Prozessmodell bestehen. Standardisierte Datensätze schaffen einheitliche Prüfbedingungen und ermöglichen objektive Leistungsvergleiche. Die in dieser Arbeit verwendeten Testfälle decken ein breites Spektrum ab - von kundenorientierten Service- und Bestellprozessen (E-Commerce) über fachliche Abläufe im Versicherungs- und Gesundheitswesen bis hin zu technischen Szenarien (Smart-Home/IoT). Ergänzend sind typische betriebliche Querschnittsprozesse (z.\,B. Finanzen, Logistik, HR) sowie lehrnahe Prozesse aus universitären Übungsaufgaben enthalten. Darüber hinaus umfasst der Datensatz bewusst kleine, gezielt reduzierte Testfälle, um unterschiedliche Modellkomplexitäten und Randfälle abzubilden. Die Modelle liegen sowohl in deutscher als auch in englischer Sprache vor.

\section{Labeling Tool}\label{sec:labeling-tool}

Hier brauche noch genaue Kapitel, aber hier soll auf jeden Fall folgendes rein:
\begin{itemize}
    \item Definition von Labeln hier, oder kommt das schon vorher? (Muss im Verlauf beim schreiben entschieden werden)
    \item Labeling Software wurde entwickelt
    \item Es können Datensätze mit Namen und Beschreibungen erstellt werden
    \item Für jeden Datensatz können beliebig viele Testcases erstellt werden
    \item Ein Testcase ist ein BPMN Diagramm, welches direkt in der App mithilfe von bpmn.io bearbeitet werden kann
    \item Zusätzlich bietet der Editor eine Labeling Funktionalität, mit welcher Aktivitäten, welche als kritisch erkannt werden sollen, gelabelt werden. Zusätzlich kann auch eine Erklärung angegeben werden
    \item Datensätze und gelabelte Testcases werden in Datenbank gespeichert und werden während der Evaluierung des Evaluationsframeworks benutzt
\end{itemize}
\section{Quellen und Eigenschaften der Datensätze}\label{sec:quellen-und-eigenschaften-der-datensatze}

Für die Evaluation wurden drei Gruppen von \ac{BPMN}-Datensätzen eingesetzt:

\begin{enumerate}
    \item Prozesse, die von der Universität Ulm bereitgestellt wurden (z.B.\ Lehrbeispiele aus Übungsaufgaben).
    \item Realistische, mittelgroße Szenarien aus verschiedenen Domänen. Diese Prozesse beinhalten Elemente wie Pools, Lanes, Datenobjekte und Gateways.
    \item Kleine, reduzierte Testfälle mit maximal fünf Aktivitäten und wenigen weiteren Elementen (z.B. einfacher Sequenzfluss ohne Pools).
\end{enumerate}

Diese heterogene Auswahl ist bewusst getroffen worden, da die Mischung aus verschiedenen Domänen und Modellkomplexitäten eine aussagekräftige Evaluation ermöglicht. In der Literatur wird betont, dass eine erhöhte Datensatzvielfalt die Robustheit der Bewertung steigert und einseitige Ergebnisse vermeidet \cite{blake2025datasetdiversity}. Tabelle \ref{tab:datensaetze-eckdaten} zeigt die Eckdaten der Datensätze.

\textcolor{orange}{// TODO Noch mehr Testfälle hinzufügen - insbesondere, um auf die Datenassoziationsthematik aus dem Kapitel \ref{sec:bpmn} einzugehen. Ich sollte also mindestens ein kleines Beipsiel jeweils mit und ohne Datenassoziation haben, um den Unterschied zu verdeutlichen. Das bedeutet aber auch, dass ich die Tabelle \ref{tab:datensaetze-eckdaten} neu berechnen muss.}

\begin{table}[htbp]
    \centering
    \begin{threeparttable}
    \caption{Eckdaten der verwendeten Datensätze}
    \label{tab:datensaetze-eckdaten}
    \begin{tabular}{l r r r}
        \toprule
        & Uni-Prozesse & Reale Szenarien & Kleine Testfälle \\
        \midrule
        Testfälle Gesamt                  & 5  & 5  & 10 \\
        Testfälle (DE)                    & 0 & 4 & 10 \\
        Testfälle (EN)                    & 5 & 1 & 0 \\
        Ø Aktivitäten $\pm$ SD\tnote{1}   & 13,4 $\pm$ 2,6 & 11,6 $\pm$ 4,2 & 5 $\pm$ 0 \\
        Ø Aktivitäten (kritisch) $\pm$ SD & 8,6 $\pm$ 3,6 & 6,6 $\pm$ 1,9 & 2,6 $\pm$ 1,5 \\
        Ø Datenobjekte $\pm$ SD           & 1,4 $\pm$ 1,9 & 3,6 $\pm$ 2,1 & 0 $\pm$ 0 \\
        Ø Datenassoziationen $\pm$ SD     & 2,4 $\pm$ 3,3 & 7 $\pm$ 4 & 0 $\pm$ 0 \\
        Ø Ereignisse $\pm$ SD             & 21 $\pm$ 13,8 & 8,2 $\pm$ 2,8 & 2 $\pm$ 0 \\
        Ø Gateways $\pm$ SD               & 13 $\pm$ 7,6 & 1,8 $\pm$ 1,5 & 0 $\pm$ 0 \\
        Ø Pools $\pm$ SD                  & 3,4 $\pm$ 1,1 & 3 $\pm$ 1 & 0 $\pm$ 0 \\
        Ø Lanes $\pm$ SD\tnote{2}         & 3 $\pm$ 1 & 4 $\pm$ 0,7 & 0 $\pm$ 0 \\
        Ø Nachrichtenflüsse $\pm$ SD      & 9,4 $\pm$ 5,3 & 5,2 $\pm$ 0,8 & 0 $\pm$ 0 \\
        Ø Annotationen $\pm$ SD           & 1 $\pm$ 1,7 & 0 $\pm$ 0 & 0 $\pm$ 0 \\
        \bottomrule
    \end{tabular}
    \begin{tablenotes}
        \item[1] SD = Standardabweichung $s$ der jeweiligen Anzahl pro Testfall.
        \item[2] Blackbox-Pools ohne Lanes wurden nicht mitgezählt, daher kann der Durchschnittswert der Lanes geringer ausfallen als der, der Pools.
    \end{tablenotes}
    \end{threeparttable}
\end{table}

\textcolor{orange}{// TODO Hier noch einen Abschnitt wo ich ein paar besondere Testfälle hervorhebe, z.B. mit speziellen Datenassoziationen, oder reicht der Überblick mit den Eckdaten? Spätestens in den Fallstudien der Ergebnisse werde ich ja nochmal auf einzelne Testfälle eingehen.}
\section{Labeling-Guide}\label{sec:labeling-guide}

Nachfolgend wird beschrieben, nach welchen Richtlinien die Daten für die Klassifizierung \ac{DSGVO}-kritischer Aktivitäten gelabelt wurden.

Die Aktivitäten in den Testfällen sollen mit dem Label \enquote{kritisch} versehen werden, wenn sie potenziell personenbezogene Daten verarbeiten und somit im Sinne der \ac{DSGVO} relevant sein könnten. Die wichtigsten Begriffe der \ac{DSGVO} wurden bereits in Abschnitt \ref{sec:dsgvo} definiert.

Beim Labeln einer Aktivität können Grenzfälle auftreten - etwa, wenn kein Datenobjekt vorhanden ist, der Name aber auf Datenverarbeitung hindeutet (z.\,B. \enquote{Verträge archivieren}). Solche Verträge können personenbezogen sein (z.\,B. Arbeitsverträge) oder rein geschäftlich zwischen Unternehmen. In diesen Fällen wird zunächst der Kontext geprüft: Gibt es Hinweise auf personenbezogene Daten, z.\,B. über Pools/Lanes oder angrenzende Aktivitäten im Prozess? Fehlen eindeutige Hinweise, wird die Aktivität als unkritisch gelabelt. Deutet der Kontext hingegen auf die Verarbeitung personenbezogener Daten hin, z.\,B. durch einen Prozessnamen wie \enquote{Mitarbeiterverwaltung} oder vorangehende Aktivitäten wie \enquote{Mitarbeiterdaten erfassen}, erhält die Aktivität das Label kritisch. Im Zweifel wird kritisch gelabelt, um eine höhere Sensitivität zu gewährleisten.

Tabelle \ref{tab:labeling-examples} listet beispielhaft einige Aktivitäten mit ihrer Klassifikation und der zugehörigen Begründung auf.

\begin{table}[htbp]
    \centering
    \caption{Beispielhafte Aktivitäten und Label.}
    \begin{tabularx}{\textwidth}{p{0.4\textwidth} c p{0.4\textwidth}}
        \toprule
        Aktivität & Kritisch? & Kommentar \\
        \midrule
        Lieferadresse eingeben & Ja & Name, Anschrift des Kunden werden aufgenommen. \\
        Rückfrage an Kunden senden & Ja & Kontaktinformationen werden verwendet. \\
        Fall anlegen & Ja & Aktivität befindet sich im Kundenservice-Kontext, personenbezogene Daten wahrscheinlich. \\
        Sprache zu Text verarbeiten & Ja & Im Kontext eines Sprachassistenten werden biometrische Daten des Nutzers verarbeitet. \\
        Produkt versenden & Nein\textsuperscript{*} & Logistik und Sachvorgänge sind nicht per se datenschutzkritisch, solange keine neue Datenverarbeitung, wie ein Labeldruck stattfindet. \\
        Systemprotokoll auslesen & Ja & Im Kontext einer technischen Wartung können personenbezogene Daten (z.\,B. Nutzer-\texttt{id}s) enthalten sein. \\
        Logdaten archivieren (anonym) & Nein & Keine personenbezogenen Daten enthalten. \\
        Gerät kalibrieren & Nein & Im Kontext einer technischen Wartung werden keine personenbezogenen Daten verarbeitet. \\
        \bottomrule
    \end{tabularx}
    \label{tab:labeling-examples}
\end{table}