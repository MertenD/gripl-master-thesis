\section{Labeling Tool}\label{sec:labeling-tool}

Hier sollte ich auch darauf eingehen, dass die fertigen gelabelten Testdaten in einer Datenbank abgelegt werden, damit ich mich im Kapitel \ref{sec:testdaten} des Evaluationsframeworks darauf beziehen kann, dass die Datensätze über ihre IDs, die in der Konfig angegeben werden, aus der Datenbank gelesen werden.

Hier brauche noch genaue Kapitel, aber hier soll auf jeden Fall folgendes rein:
\begin{itemize}
    \item Definition von Labeln hier, oder kommt das schon vorher? (Muss im Verlauf beim schreiben entschieden werden)
    \item Labeling Software wurde entwickelt
    \item Es können Datensätze mit Namen und Beschreibungen erstellt werden
    \item Für jeden Datensatz können beliebig viele Testcases erstellt werden
    \item Ein Testcase ist ein BPMN Diagramm, welches direkt in der App mithilfe von bpmn.io bearbeitet werden kann
    \item Zusätzlich bietet der Editor eine Labeling Funktionalität, mit welcher Aktivitäten, welche als kritisch erkannt werden sollen, gelabelt werden. Zusätzlich kann auch eine Erklärung angegeben werden
    \item Datensätze und gelabelte Testcases werden in Datenbank gespeichert und werden während der Evaluierung des Evaluationsframeworks benutzt
\end{itemize}