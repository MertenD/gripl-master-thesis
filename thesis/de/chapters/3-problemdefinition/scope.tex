\section{Scope und Annahmen}\label{sec:scope-und-annahmen}

Dieser Abschnitt definiert Geltungsbereich, Annahmen und Risiken des Ansatzes. Dadurch wird eine klare Einordnung der Ergebnisse und ihrer Reproduzierbarkeit ermöglicht.

\subsection*{Geltungsbereich}

Die folgenden Punkte definieren den Geltungsbereich der Arbeit:

\begin{itemize}
    \item Klassifiziert werden ausschließlich \emph{Aktivitäten}. Dafür wird sinnvoller Kontext berücksichtigt, wie Labels, Pools/Lanes (z.\,B.\ \enquote{Onlineshop}, \enquote{Kunde}), Message Flows sowie vorhandene Datenobjekte (z.\,B.\ \enquote{Paket Tracking-id}).
    \item Labels und Artefakte liegen in Deutsch und Englisch vor.
    \item Es handelt sich um ein \emph{Screening}, nicht um eine Rechtsprüfung. Kritisch klassifizierte Aktivitäten sind anschließend juristisch zu prüfen.
\end{itemize}

\subsection*{Annahmen und Risiken}

Die folgenden Annahmen und potenziellen Risiken sind für die Interpretation der Ergebnisse relevant:

\begin{itemize}
    \item Bei fehlenden Datenobjekten oder mehrdeutigen Labels kann sich die Einschätzung verschlechtern. Das ist ein bekanntes Problem in ähnlichen Studien \cite{nake2023towards}.
    \item Optional generierte \ac{LLM}-Begründungen sind als \emph{Hilfetexte} zu verstehen, um die Entscheidung des \acp{LLM} besser einordnen zu können, bilden aber nicht zwingend die tatsächlichen Entscheidungsgründe des Modells ab.
    \item Ungültiges \ac{BPMN}-XML oder Laufzeitfehler werden als \enquote{technischer Fehler} erfasst und nicht in die Metrikzählung eingerechnet. Sie werden separat berichtet.
\end{itemize}