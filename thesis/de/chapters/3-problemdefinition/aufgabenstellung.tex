\section{Aufgabenstellung}\label{sec:aufgabenstellung}

Ziel der Arbeit ist eine \emph{binäre Klassifikation} auf Ebene einzelner \ac{BPMN}-Aktivitäten: Für jede Aktivität eines Eingabemodells im \ac{BPMN}-XML-Format (Version 2.0.2) \cite{omgbpmn} soll entschieden werden, ob sie \emph{kritisch} im Sinne des Datenschutzrechts ist oder nicht.

\begin{itemize}
    \item \textbf{Eingabe} ist ein valides \ac{BPMN}-XML mit stabilen \texttt{id}-Attributen je Aktivität \cite{omgbpmn}.
    \item \textbf{Ausgabe} ist eine Menge von Aktivitäts-\texttt{ids}, die als \emph{kritisch} klassifiziert wurden. Optional werden zusätzlich eine natürlichsprachige Begründung für einzelne Entscheidungen ausgegeben. Im Fall der Klassifizierungspipeline dieser Arbeit werden die Begründungen vom \ac{LLM} generiert. Die Erklärungen dienen ausschließlich der Nachvollziehbarkeit der gewählten Klassifizierungen, werden allerdings nicht in der Evaluation berücksichtigt.
\end{itemize}

\subsection*{Begriffsbestimmung \enquote{kritisch}}

Eine Aktivität gilt in dieser Arbeit als \emph{kritisch}, wenn sie \emph{personenbezogene Daten} verarbeitet. Personenbezogene Daten sind, nach Art.~4~Abs.~1 \ac{DSGVO} \cite{GDPR2016}, alle Informationen, die sich auf eine identifizierte oder identifizierbare natürliche Person beziehen. Gemäß Art.~4~Abs.~2 \ac{DSGVO} \cite{GDPR2016} umfasst Verarbeitung jede mit personenbezogenen Daten vorgenommene Operation, wie z.\,B. Erheben, Speichern, Abrufen, Verwenden, Übermitteln und Löschen. Dies schließt auch die \emph{Nutzung bereits vorhandener Daten} (z.\,B. Lesen/Abgleichen) ein.

Diese Aufgabenstellung reiht sich in Arbeiten zur Kennzeichnung kritischer/unkritischer Tätigkeiten in Prozessartefakten ein und bildet die Referenz für die Qualitätsziele im nächsten Abschnitt. \cite{nake2023towards}