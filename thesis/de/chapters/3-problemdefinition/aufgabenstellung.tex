\section{Aufgabenstellung}\label{sec:aufgabenstellung}

Ziel der Arbeit ist eine \emph{binäre Klassifikation} auf Ebene einzelner \ac{BPMN}-Aktivitäten: Für jede Aktivität eines Eingabemodells im \ac{BPMN}-XML-Format (Version 2.0.2) \cite{omgbpmn} soll entschieden werden, ob sie \emph{kritisch} im Sinne des Datenschutzrechts ist oder nicht.

\begin{itemize}
    \item \textbf{Eingabe} ist ein valides \ac{BPMN}-XML mit stabilen \texttt{id}-Attributen je Aktivität \cite{omgbpmn}.
    \item \textbf{Ausgabe} ist eine Menge von Aktivitäts-\texttt{ids}, die als \emph{kritisch} klassifiziert worden sind. Optional kann zusätzlich eine natürlichsprachige Begründung für einzelne Entscheidungen ausgegeben werden. Im Fall der Klassifizierungspipeline dieser Arbeit werden die Begründungen vom \ac{LLM} generiert. Die Erklärungen dienen ausschlißelich der Nachvollziehbarkeit der gewählten Klassifizierungen, werden allerdings nicht in der Evaluation berücksichtigt.
\end{itemize}

\subsection*{Begriffsbestimmung \enquote{kritisch}}

Eine Aktivität gilt in dieser Arbeit als \emph{kritisch}, wenn sie \emph{personenbezogene Daten} verarbeitet.Nach Art.~4~Abs.~1 \ac{DSGVO} sind personenbezogene Daten alle Informationen, die sich auf eine identifizierte oder identifizierbare natürliche Person beziehen, und \emph{Verarbeitung} umfasst gemäß Art-~4~Abs.~2 jede mit personenbezogenen Daten vorgenommene Operation (u.\,a. Erheben, Speichern, Abrufen, Verwenden, Übermitteln, Löschen) \cite{GDPR2016}. Dies schließt auch die \emph{Nutzung bereits vorhandener Daten} (z.\,B. Lesen/Abgleichen) ein.
Die Aufgabenstellung reiht sich damit in verwandte Arbeiten ein, die kritische/unkritische Tätigkeiten in Prozessartefakten kennzeichnen (z.\,B. für textuelle Prozessbeschreibungen) \cite{nake2023towards}.