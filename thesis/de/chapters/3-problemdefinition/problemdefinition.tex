\chapter{Problemdefinition und Zielkriterien}\label{ch:problemdefinition-und-zielkriterien}

Dieses Kapitel präzisiert die Aufgabe der Arbeit, die Qualitätsziele der Klassifikation und steckt den fachlichen Geltungsbereich ab. Außerdem wird das Experimentdesign beschrieben, um die Forschungsfragen systematisch zu beantworten. Damit schafft es die Grundlage für die in Kapitel~\ref{ch:klassifizierungsalgorithmus-(design-und-implementierung)} beschriebene Klassifizierungspipeline sowie für die späteren Experimente und deren Auswertung.

Abbildung~\ref{fig:running_example} zeigt einen Beispielprozess, der in den folgenden Abschnitten als durchgehende Referenz dient. Er modelliert den Versand eines Statusberichts eines Onlineshops an Kunden. Hierfür werden personenbezogene Daten in den Aktivitäten \enquote{Tracking-id generieren} und \enquote{Status Update senden} verarbeitet.

\begin{figure}[h]
    \centering
    \label{fig:running_example}
    \includegraphics[width=0.7\textwidth]{images/running_example}
    \caption{Beispielprozess zur Veranschaulichung der Aufgabenstellung}
\end{figure}

\textcolor{orange}{// TODO Abbildung im Kapitel referenzieren wo es Sinn ergibt. Aktuell ist sie noch nirgends referenziert.}

\section{Aufgabenstellung}\label{sec:aufgabenstellung}

Ziel der Arbeit ist eine \emph{binäre Klassifikation} auf Ebene einzelner \ac{BPMN}-Aktivitäten: Für jede Aktivität eines Eingabemodells im \ac{BPMN}-XML-Format (Version 2.0.2) \cite{omgbpmn} soll entschieden werden, ob sie \emph{kritisch} im Sinne des Datenschutzrechts ist oder nicht.

\begin{itemize}
    \item \textbf{Eingabe} ist ein valides \ac{BPMN}-XML mit stabilen \texttt{id}-Attributen je Aktivität \cite{omgbpmn}.
    \item \textbf{Ausgabe} ist eine Menge von Aktivitäts-\texttt{id}s, die als \emph{kritisch} klassifiziert wurden. Optional wird zusätzlich eine natürlichsprachige Begründung für einzelne Entscheidungen ausgegeben. Im Fall der Klassifizierungspipeline dieser Arbeit werden die Begründungen vom \ac{LLM} generiert. Diese dienen ausschließlich der Nachvollziehbarkeit der gewählten Klassifizierungen, werden allerdings nicht in der Evaluation berücksichtigt.
\end{itemize}

Zur Veranschaulichung zeigt Listing~\ref{lst:running-example-bpmn} einen vereinfachten  Auszug aus dem \ac{BPMN}-XML des laufenden Beispiels. Die erwartete Ausgabe ist \{\texttt{Activity\_generate}, \texttt{Activity\_send}\}.

\begin{lstlisting}[caption={BPMN-XML-Auszug des laufenden Beispiels}, label={lst:running-example-bpmn}]
<task id="Activity_generate"  name="Generate tracking id"/>
<task id="Activity_template"  name="Load notification template"/>
<task id="Activity_send"      name="Send status update"/>
\end{lstlisting}

\subsection*{Begriffsbestimmung \enquote{kritisch}}

Eine Aktivität gilt in dieser Arbeit als \emph{kritisch}, wenn sie \emph{personenbezogene Daten} verarbeitet. Personenbezogene Daten sind, nach Art.~4~Abs.~1 \ac{DSGVO} \cite{GDPR2016}, alle Informationen, die sich auf eine identifizierte oder identifizierbare natürliche Person beziehen. Gemäß Art.~4~Abs.~2 \ac{DSGVO} \cite{GDPR2016} umfasst Verarbeitung jede mit personenbezogenen Daten vorgenommene Operation, wie z.\,B. Erheben, Speichern, Abrufen, Verwenden, Übermitteln und Löschen. Dies schließt auch die \emph{Nutzung bereits vorhandener Daten} (z.\,B. Lesen/Abgleichen) ein. Am laufenden Beispiel bedeutet dies konkret: \texttt{Activity\_send} ist kritisch, da beim Versand typischerweise personenbezogene Daten (z.\,B. E-Mail-Adresse) verarbeitet werden. \texttt{Activity\_generate} ist kritisch, weil die Tracking-\texttt{id} im Kontext des Kundenkontakts einer Person zugeordnet werden kann. \texttt{Activity\_template} ist im Grundfall nicht kritisch, sofern lediglich eine generische Vorlage geladen wird und keine personenbezogenen Daten einfließen.

Diese Aufgabenstellung reiht sich in Arbeiten zur Kennzeichnung kritischer/unkritischer Tätigkeiten in Prozessartefakten ein und bildet die Referenz für die Qualitätsziele im nächsten Abschnitt. \cite{nake2023towards}
\section{Qualitätsziele}\label{sec:qualitatsziele}

// TODO Hier prüfen, ob ich das so schon hier sagen kann oder das erst bei der Diskussion thematisiert werden soll, dass mir der hohe Recall eher positiv aufgefallen ist, weil es des öfteren plausible Erklärungen dazu gab

\begin{itemize}
    \item Hauptziel ist ein hoher Recall mit minimalen False Negatives, bei noch akzeptabler Precision und somit auch überschaubaren False Positives
\end{itemize}

// TODO Aktuell unterstützt meine Evaluierung noch keine Seeds, aber das sollte ich noch umsetzen können und dann Temperature bei den LLMs auf 0 setzen

\begin{itemize}
    \item Sekundäres Ziel ist Stabilität über mehrere Seeds hinweg
    \item Hier könnte ich noch bestimmte Wertebereiche vorschlagen, sie sinnvoll wären wie Recall $\geq 80\%$, False Positives $\leq 1{,}5/$Prozess. Unbedingt dafür noch in anderen Papern nach guten Werten schauen, wenn ich das machen sollte
\end{itemize}
\section{Scope und Annahmen}\label{sec:scope-und-annahmen}

Dieser Abschnitt definiert Geltungsbereich, Annahmen und Risiken des Ansatzes. Dadurch wird eine klare Einordnung der Ergebnisse und ihrer Reproduzierbarkeit ermöglicht.

\subsection*{Geltungsbereich}

Die folgenden Punkte definieren den Geltungsbereich der Arbeit:

\begin{itemize}
    \item Klassifiziert werden ausschließlich Aktivitäten. Dafür wird sinvoller Kontext (z.\,B. Prozessname, Pool/Lane, Datenobjekte) berücksichtigt.
    \item Labels und Artefakte liegen in Deutsch und Englisch vor.
    \item Es handelt sich um ein \emph{Screening}, nicht um eine Rechtsprüfung. Kritisch klassifizierte Aktivitäten sind anschließend juristisch zu prüfen.
\end{itemize}

\subsection*{Annahmen und Risiken}

Die folgenden Annahmen und potenziellen Risiken sind für die Interpretation der Ergebnisse relevant:

\begin{itemize}
    \item Bei fehlenden Datenobjekten oder mehrdeutigen Labels kann sich die Einschätzung verschlechtern. Das ist ein bekanntes Problem in ähnlichen Studien \cite{nake2023towards}.
    \item Optional generierte \ac{LLM}-Begründungen sind als \emph{Hilfetexte} zu verstehen, um die Entscheidung des \acp{LLM} besser einordnen zu können, bilden aber nicht zwingend die tatsächlichen Entscheidungsgründe des Modells ab.
    \item Ungültiges \ac{BPMN}-XML oder Laufzeitfehler werden als \enquote{technischer Fehler} erfasst und nicht in die Metrikzählung eingerechnet. Sie werden separat berichtet.
\end{itemize}

Dieses Kapitel definierte die binäre Klassifikation von \ac{BPMN}-Aktivitäten als kritisch/unkritisch mit Fokus auf maximalen Recall bei akzeptabler Precision als Aufgabe. Es legte Qualitätsziele, Metriken, Geltungsbereich und Annahmen fest. Darauf aufbauend geht es im nächsten Kapitel um das Design und die Implementierung der Klassifizierungspipeline, die diese Anforderungen umsetzt.
\section{Experimentdesign}\label{sec:experimentdesign}

Das gesamte Kapitel definierte die binäre Klassifikation von \ac{BPMN}-Aktivitäten als kritisch/unkritisch mit Fokus auf maximalen Recall bei akzeptabler Precision und legte Qualitätsziele, Metriken, Geltungsbereich sowie Annahmen fest. Darauf aufbauend beschreibt dieser Abschnitt das Experimentdesign, mit dem \acp{LLM} fair und reproduzierbar verglichen werden, um die Forschungsfrage \textbf{FF1} sowie die Unterfragen \textbf{UF1}–\textbf{UF4} zu beantworten. Die konkrete Ausgestaltung und Durchführung der Experimente werden in Kapitel \ref{ch:versuchsaufbau} \emph{Versuchsaufbau} erläutert. Im Folgenden werden die wesentlichen Aspekte des Experimentdesigns beschrieben:

\begin{description}
    \item[\textbf{Ziel}] Ziel ist ein transparenter Vergleich mehrerer \acp{LLM}, die alle dieselbe Klassifizierungspipeline durchlaufen. Sie wird in Kapitel \ref{ch:klassifizierungsalgorithmus-(design-und-implementierung)} daher so entworfen, dass sich das \ac{LLM} austauschen lässt. Damit das Evaluationsframework, das in Kapitel \ref{ch:evaluationsframework} beschrieben wird, verschiedene Modelle unmittelbar gegeneinander ausführen kann, erfolgt der Wechsel der \acp{LLM} zur Laufzeit über übergebene Parameter, die das Modell definieren.
    \item[\textbf{Vergleichsgegenstand}] Die Experimente werden über eine deklarative Konfiguration definiert, siehe Kapitel~\ref{sec:konfiguration-einer-evaluierung}. Sie legt fest, welche Modelle, Datensätze und weitere Parameter zum Einsatz kommen. Je nach Auswahl werden mehrere Modelle und Modellvarianten parallel im Evaluationsframework ausgeführt, darunter Open-Source- und kommerzielle Modelle. Die deklarative Konfiguration sorgt für Portabilität und Wiederholbarkeit.
    \item[\textbf{Datenbasis}] Als Datenbasis dienen die im Labeling-Tool erzeugten, gelabelten Testdatensätze, siehe Kapitel \ref{ch:labeling-und-datensatze}. Ein Testdatensatz enthält mehrere gelabelte Testfälle. Ein Testfall umfasst ein \ac{BPMN}-Prozessmodell mit Labeln, die als \ac{DSGVO}-kritischen Aktivitäten markieren. Die Auswahl der Datensätze für ein Experiment erfolgt in der Evaluierungskonfiguration, das Laden der Testfälle während der Laufzeit. Die Datensätze sollten idealerweise unterschiedliche Eigenschaften abdecken, damit die Forschungsfrage und die Unterfragen möglichst umfassend beantwortet werden.
    \item[\textbf{Metriken und Erfolgskriterium}] Ausgewertet werden die in Abschnitt~\ref{sec:qualitatsziele} beschriebenen Metriken: Accuracy, Precision, Recall und F1. Zusätzlich werden die Kennzahlen der Konfusionsmatrix betrachtet: \ac{TP}, \ac{FP}, \ac{TN}, \ac{FN}. Ein Testfall gilt als \emph{bestanden}, wenn die vom Modell als kritisch ausgegebenen Aktivitäten exakt den gelabelten kritischen Aktivitäten entsprechen. Technische Fehler werden separat ausgewiesen.
\end{description}

\subsection*{Ablauf eines Experiments}

Ein Experiment verläuft in folgenden Schritten:

\begin{enumerate}
    \item \textbf{Konfiguration laden}. Die Konfiguration mit Modellen, Datensätzen und optionalem \texttt{seed} wird geladen.
    \item \textbf{Ausführung}. Für jedes Modell werden alle ausgewählten Testfälle durch die Klassifizierungspipeline verarbeitet. Pro Testfall werden \ac{TP}, \ac{FP}, \ac{FN}, \ac{TN} sowie der Status \enquote{bestanden} oder \enquote{nicht bestanden} berechnet.
    \item \textbf{Stabilität}. Die Läufe erfolgen mit \texttt{temperature} gleich 0\footnote{Die \texttt{temperature} steuert die Zufälligkeit der Textgenerierung bei \acp{LLM}. Niedrige Werte liefern stabilere Antworten, hohe Werte vielfältigere, jedoch weniger verlässliche \cite{ibm-llm-temperature}.} und festem \texttt{seed}, sofern das jeweilige \ac{LLM} dies unterstützt. Um die Nicht-Deterministik moderner \acp{LLM} abzubilden, werden die Experimente mehrfach mit unterschiedlichen Seeds wiederholt. Die Ergebnisse werden über die Läufe gemittelt.
    \item \textbf{Bericht}. Aggregierte Kennzahlen pro Modell, wie Konfusionsmatrix, die genannten Metriken sowie die Bestehensraten werden ausgegeben. Metadaten wie verwendete Modelle, Datensätze und Seeds werden dokumentiert.
\end{enumerate}

Dieses Kapitel definiert, \emph{was} verglichen wird: Modelle, Datensätze und Metriken. Es beschreibt zudem, \emph{wie} der Vergleich erfolgt. Kapitel~\ref{ch:versuchsaufbau} dokumentiert später die praktische Umsetzung mit konkreten Modellen, exakten Parameterwerten, Seeds sowie den vollständigen genutzten Konfigurationen. Im nächsten Kapitel folgt das Design und die Implementierung der Klassifizierungspipeline, die für den Vergleich der \acp{LLM} verwendet wird.

\textcolor{orange}{// TODO Kapitel Versuchsaufbau und Durchführung können durch die Existenz von diesem Kapitel wahrscheinlich auch gut zusammengefasst werden in einem Kapitel, so wie ich es auch in diesem Abschnitt geschrieben habe.}
