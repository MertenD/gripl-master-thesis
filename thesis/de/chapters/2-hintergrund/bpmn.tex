\section{Business Process Model and Notation (BPMN)}\label{sec:bpmn}

\begin{itemize}
    \item Relevante Elemente wie Aktivitäten, Datenobjekte/-speicher, Nachrichtenflüsse, Pools/Lanes, Assoziationen
    \item Rolle von Elementen, welche zusätzliche Hinweise auf Verarbeitung von personenbezogenen Daten geben (Bspw. Aktivität holt sich Informationen aus einer Datenbank wo Kundendaten gespeichert sind; dargestellt durch Assoziation)
    \item BPMN-XML erläutern, weil das auch das Format ist in dem Businessprozesse gespeichert sind und welches ich auch als Eingabe für die Klassifizierungspipeline nutze
    \item Stabile IDs für Elemente im Businessprozess sind essenziell, damit die Ausgabe des LLM immer auf die gleichen Aktivitäten bezieht und man das Ergebnis deterministisch überprüfen kann
    \item Kleines Beispiel wie eine Datenassoziation signalisieren kann, dass eine Aktivität datenschutzkritisch ist, was nur mit dem Namen der Aktivität nicht möglich ist
\end{itemize}