\section{API-Design}\label{sec:api-design}

Dieses Kapitel beschreibt das API-Design der Klassifizierungspipeline, die zur Erkennung \ac{DSGVO}-kritischer Elemente in \ac{BPMN}-Modellen dient. Das Ziel ist es eine standardisierte Schnittstelle zu definieren, die (1) die Einbindung in bestehende Werkzeuge und das Evaluationsframework vereinfacht, (2) die Austauschbarkeit unterschiedlicher Klassifizierungsalgorithmen - insbesondere im Evaluationsframework - ermöglicht, um verschiedene Ansätze der Klassifizierung vergleichen zu können, und (3) Erweiterbarkeit fördert, sodass zukünftige Arbeiten die Schnittstelle wiederverwenden können, um ihre eigenen Klassifizierungsalgorithmen zu integrieren.

\subsection*{HTTP-Endpunkt}

Die Klassifizierungspipeline ist über einen standardisierten HTTP-Endpunkt nutzbar, dessen Struktur und die klar definierten JSON-Schemas eine einfache Integration in bestehende Werkzeuge sowie das Evaluationsframework ermöglichen. Der \texttt{POST}-Endpunkt akzeptiert \texttt{multipart/form-data} mit den folgenden Teilen:

\begin{description}
    \item[\texttt{bpmnFile} (Pflicht)] Eine BPMN-2.0-XML-Datei (\texttt{.bpmn} oder \texttt{text/xml}), die den zu analysierenden Prozess beinhaltet.
    \item[\texttt{llmProps} (Optional)] Ein JSON-Objekt zur Überschreibung von \ac{LLM}-Eigen-\linebreak~schaften zur Laufzeit. Siehe Listing \ref{lst:api-request-schema} für das JSON-Schema. Wird nichts angegeben, nutzt die Pipeline Standardwerte.
\end{description}

\begin{lstlisting}[caption={JSON-Schema der \texttt{llmProps}.},label={lst:api-request-schema}]
{
  "$schema": "https://json-schema.org/draft/2020-12/schema",
  "title": "LlmProps",
  "type": "object",
  "properties": {
    "baseUrl": {"type": "string"},
    "modelName": { "type": "string" },
    "apiKey": { "type": "string" },
    "timeoutSeconds": { "type": "number" },
    "seed": { "type": "number" },
    "temperature": { "type": "number" },
    "topP": { "type": "number" }
  },
  "required": []
}
\end{lstlisting}

Die \texttt{llmProps} erlauben das Überschreiben von \ac{LLM}-Eigenschaften zur Laufzeit. Dadurch können unterschiedliche Modelle mit demselben Klassifizierungsalgorithmus flexibel getestet und verglichen werden, ohne die Anwendung neu starten zu müssen. Dieses Design wurde gewählt, um die Experimente wie in Kapitel \ref{sec:experimentdesign} beschrieben flexibel durchführen zu können.

Die Antwort des Endpunkts hat den Medientyp \texttt{application/json}. Sie enthält eine Liste der als \ac{DSGVO}-kritisch klassifizierten Elemente. Für jedes Element können optional eine Begründung der Klassifikation sowie ein Elementname zur besseren Lesbarkeit angegeben werden. Zusätzlich kann die Anzahl der Versuche ausgegeben werden, die zur erfolgreichen Validierung der Antwort nötig waren (z.\,B. nach einem Schema-Validierungsfehler mit anschließendem Retry-Mechanismus). Diese Informationen nutzt das Evaluationsframework später, um die Robustheit der eingesetzten \acp{LLM} sichtbar zu machen. Das JSON-Schema der Antwort ist in Listing \ref{lst:api-response-schema} dargestellt.

\begin{lstlisting}[caption={JSON-Schema der API-Antwort.},label={lst:api-response-schema}]
{
  "$schema": "https://json-schema.org/draft/2020-12/schema",
  "title": "BpmnAnalysisResult",
  "type": "object",
  "properties": {
    "criticalElements": {
      "type": "array",
      "items": {
        "type": "object",
        "properties": {
          "id": { "type": "string" },
          "name": { "type": "string" },
          "reason": { "type": "string" }
        },
        "required": ["id"]
      }
    },
    "amountOfRetries": { "type": "number" }
  },
  "required": ["criticalElements"]
}
\end{lstlisting}

\newpage
Im nächsten Kapitel wird die Webapp-Sandbox beschrieben, die als Beispielanwendung dient, um die Klassifizierungspipeline intuitiv nutzen zu können. Sie verwendet das hier beschriebene API, um die Klassifizierung durchzuführen.