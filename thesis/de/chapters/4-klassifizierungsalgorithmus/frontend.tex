\section{Nutzung über Webapp}\label{sec:nutzung-uber-webapp}

Zur interaktiven Nutzung der Klassifizierung wurde eine \emph{Sandbox} in Form einer Webapp entwickelt. Sie verbindet einen vollwertigen \ac{BPMN}-Editor auf Basis von \texttt{BPMN.js} \cite{bpmn-js} mit der in Kapitel~\ref{sec:api-design} beschriebenen HTTP-Schnittstelle und macht die Analyse damit intuitiv bedienbar. In der Sandbox können \ac{BPMN}-Modelle erstellt, verändert, exportiert und importiert sowie auf Datenschutzrelevanz analysiert werden. Als kritisch klassifizierte Aktivitäten werden nach der Analyse direkt im Editor farblich hervorgehoben, wie in Abbildung \ref{fig:sandbox-frontend-analyzed-model} zu sehen ist.

\begin{figure}
    \centering
    \includegraphics[width=\linewidth]{images/sandbox/sandbox-analyzed-model}
    \caption{Sandbox im Frontend mit hervorgehobenen kritischen Aktivitäten nach Analyse.}
    \label{fig:sandbox-frontend-analyzed-model}
\end{figure}

Außerdem können die vom \ac{LLM} generierten Begründungen zu jeder als kritisch erkannten Aktivität im Editor eingesehen werden. Diese Erläuterungen werden gesammelt in einer aufklappbaren Karte im unteren Bereich des Editors angezeigt, siehe Abbildung \ref{fig:sandbox-frontend-ai-reasoning}.

\textcolor{orange}{// TODO Die Sprache der Begründungen des LLM irgendwo thematisieren oder noch anpassen. Aktuell ist die Sprache der Begründungen immer Englisch. Vielleicht ändere ich das auf Deutsch ab, da die Mastrarbeit auf Deutsch ist oder ich passe den Code so an, dass die Begründung in der gleichen Sprache wie das Modell ist.}

\begin{figure}
    \centering
    \includegraphics[width=\linewidth]{images/sandbox/sandbox-ai-reasoning}
    \caption{Exemplarische Begürundungen der Klassifikation durch das LLM in der Sandbox.}
    \label{fig:sandbox-frontend-ai-reasoning}
\end{figure}

Um verschiedene \acp{LLM} vergleichen zu können, verfügt die Sandbox auf der rechten Seite über ein Einstellungsmenü mit konfigurierbaren \ac{LLM}-Parametern (siehe Abbildung \ref{fig:sandbox-frontend-analyzed-model}). Diese Parameter sind identisch zu den in Kapitel \ref{sec:api-design} beschriebenen \texttt{llmProps} und werden beim Starten der Analyse in die API-Anfrage überführt.