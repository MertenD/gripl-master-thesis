\chapter{Klassifizierungsalgorithmus (Design und Implementierung)}\label{ch:klassifizierungsalgorithmus-(design-und-implementierung)}

\begin{itemize}
    \item Im gesamten Kapitel werden die genutzten Technologien an geeigneter Stelle aufgezeigt und beschrieben (Camunda-BPMN, LangChain4j, Spring Boot, Kotlin, \ldots)
    \begin{itemize}
        \item TODO: Ggf. extra Unterkapitel für die genutzten Technologien oder doch immer da erwähnen wo es gerade passt
    \end{itemize}
\end{itemize}

\section{Ziel und Annahmen}\label{sec:ziel-und-annahmen}

\begin{itemize}
    \item Eingabe ist BPMN-XML, Ausgabe sind die IDs der kritischen Aktivitäten mit ggf. Erklärung
    \item Getestet wird mit BPMN-Modellen aus Camunda und bpmn.io
    \item Der Algorithmus klassifiziert Aktivitäten in BPMN-Prozessen binär nach ``kritisch'' oder ``nicht kritisch''
    \item Ausgegeben werden ausschließlich die IDs der kritischen Aktivitäten ggf. mit Erklärung, warum sie als kritisch klassifiziert wurde
    \item Das Ziel ist ein robustes und reproduzierbares Verhalten über unterschiedliche Modelle
    \begin{itemize}
        \item Granularität der Prozesse (Bspw. $>$5 Aktivitäten, 20+ Aktivitäten)
        \item Gateways, Datenobjekte, Datenbanken
        \item Mehrere Pools/Lanes, Message Flows
        \item Evtl. (?) verschiedene Sprachen, Text-Annotationen
    \end{itemize}
\end{itemize}
\section{BPMN Preprocessing}\label{sec:bpmn-preprocessing}

Ziel der Vorverarbeitung (Preprocessing) ist es, für jedes Flow-Element einen \emph{strukturierten Kontext} zu erzeugen. Dieser Kontext umfasst die eigenen Attribute, wie \texttt{id}, \texttt{name} und \texttt{documentation}, sowie die Beziehungen zu anderen Elementen im \ac{BPMN}-Diagramm. Dazu gehören vorangehende und nachfolgende Flow-Elemente, Datenobjekte, assoziierte Elemente, sowie Informationen über den Pool und die Lane, in denen sich das Element befindet. Das Parsen des \ac{BPMN}-XML erfolgt mit der \emph{Camunda BPMN Model API}, die das XML in ein Objektmodell überführt \cite{camunda-bpmn-model-api, camunda-bpmn-model-read}. Auf dieser Basis werden die relevanten Informationen extrahiert und in der Datenklasse \texttt{BpmnElement} strukturiert abgelegt. Die Datenklasse ist in Listing \ref{lst:bpmn-element-class} zu sehen. Dadurch entsteht für jedes Flow-Element ein umfassender Kontext, der später im Prompt genutzt wird, um dem \ac{LLM} alle notwendigen Informationen strukturiert bereitzustellen. Außerdem werden durch das Format Tokens eingespart, da irrelevante Informationen, wie die Positionen der Elemente im XML, weggelassen werden. In Abbildung \ref{fig:architecture-diagram} ist dieser Schritt über die Aktivität \enquote{Kontext aller Elemente aufbauen} dargestellt.

\begin{lstlisting}[language=Kotlin,caption={Interne \ac{BPMN}-Repräsentation je Flow-Element.},label={lst:bpmn-element-class}]
data class BpmnElement(
   val type: String,
   val id: String,
   val name: String? = null,
   val documentation: String? = null,
   val poolName: String? = null,
   val laneName: String? = null,
   val outgoingFlowElementIds: List<String> = emptyList(),
   val incomingFlowElementIds: List<String> = emptyList(),
   val outgoingMessageFlowsToElementIds: List<String> = emptyList(),
   val incomingMessageFlowsFromElementIds: List<String> = emptyList(),
   val incomingDataFromElementIds: List<String> = emptyList(),
   val outgoingDataToElementIds: List<String> = emptyList(),
   val associatedElementIds: List<String> = emptyList()
)
\end{lstlisting}
\section{Prompt Engineering}\label{sec:prompt-engineering}

\begin{itemize}
    \item Referenz auf Zero-Shot/Few-Shot (?)
    \item Prompt-Sprache (Passendes Paper referenzieren)
    \item Erklärung des System-Prompt (Anleitung was als kritisch klassifiziert werden soll, Auflistung wichtiger DSGVO-kritischer Inhalte und Definitionen aus der DSGVO wie ``Verarbeitung'', Definition des Ausgabeformats mit LangChain4j) (System-Prompt im Anhang beifügen oder hier direkt integrieren)
    \item Was steht im User-Prompt drin
\end{itemize}
\section{Validierung der Ausgabe}\label{sec:validierung-der-ausgabe}

Zusätzlich zu den in Kapitel \ref{sec:prompt-engineering} beschriebenen Maßnahmen zur Sicherstellung strukturierter Ausgaben wird die Antwort des \ac{LLM} durch LangChain4j validiert. Wenn das erwartete Schema vom \ac{LLM} nicht eingehalten wird, wirft Langchain4j eine \texttt{OutputParsingException}.

\textcolor{orange}{// TODO Eventuell noch Retry Mechanismus wenn am Ende noch Zeit ist mit Nachrichtenverlauf und Parsing Fehlermeldung. Das muss aber auch erst einmal implementiert werden. Aktuell gibt es nur den Parsing Error.}

Ein weiterer Mechanismus zur Validierung der Ausgabe ist die Überprüfung der ausgegebenen \texttt{ids}. Da das \ac{LLM} theoretisch jede beliebige \texttt{id} ausgeben könnte, werden die ausgegebenen \texttt{ids} mit den tatsächlichen \texttt{ids} der Aktivitäten aus dem Prozess abgeglichen. Falls eine ausgegebene Aktivität nicht im Prozess existiert, wird diese aus der Antwort der Klassifizierung entfernt. Dadurch wird sichergestellt, dass nur gültige \texttt{ids} in der finalen Ausgabe enthalten sind.

Nachdem es nun möglich ist, \ac{BPMN}-Prozesse vorzuverarbeiten, zu klassifizieren und die Ausgabe zu validieren, folgt als nächster Schritt die Definition einer Schnittstelle zum Aufruf der Pipeline. Das nächste Kapitel beschreibt dafür das API-Design.
\section{API Design}\label{sec:api}

\begin{itemize}
    \item Klassifizierungspipeline ist über HTTP-Endpunkt aufrufbar, wo das BPMN-XML und ggf. Attribute zum Überschreiben des genutzten LLMs übergeben werden
    \item Klassifizierungspipeline kann außerdem lokal über CLI gestartet werden und legt Ergebnisse in Datei ab
    \item Dadurch kann die Klassifizierung leicht in das Evaluationsframework eingebunden werden, während das Evaluationsframework flexibel bleibt und beliebige Klassifizierungsalgorithmen benutzen kann, welche ebenfalls die gleiche HTTP-Schnittstelle anbieten
    \item Beispielaufruf des Klassifizierungsendpunkts
    \item Irgendwo hier oder auch in einem nächsten Abschnitt die ``Schnittstelle'' mit BPMN-XML rein und JSON mit Liste der kritischen Elemente (evtl. mit Begründung) wieder. Durch diese Vereinheitlichung ist es möglich verschiedene Algorithmen in dem Evaluationsframework zu vergleichen (Stichwort Standardisierung). Vielleicht gehe ich auch soweit, dass ich den Standard für zukünftige Arbeiten propose, damit spätere Arbeiten meinen Standard zum Vergleich nutzen können
\end{itemize}
\section{Nutzung über Webapp}\label{sec:nutzung-uber-webapp}

Zur interaktiven Nutzung der Klassifizierung wurde eine \emph{Sandbox} in Form einer Webapp entwickelt. Sie verbindet einen vollwertigen \ac{BPMN}-Editor auf Basis von \texttt{BPMN.js} \cite{bpmn-js} mit der in Kapitel~\ref{sec:api-design} beschriebenen HTTP-Schnittstelle und macht die Analyse damit ganz einfach bedienbar. In der Sandbox können \ac{BPMN}-Modelle erstellt, verändert, exportiert und importiert sowie auf Datenschutzrelevanz analysiert werden. Als kritisch klassifizierte Aktivitäten werden nach der Analyse direkt im Editor farblich hervorgehoben, wie in Abbildung \ref{fig:sandbox-frontend-analyzed-model} zu sehen.

\begin{figure}
    \centering
    \includegraphics[width=\linewidth]{images/sandbox/sandbox-analyzed-model}
    \caption{Sandbox im Frontend mit hervorgehobenen kritischen Aktivitäten nach Analyse.}
    \label{fig:sandbox-frontend-analyzed-model}
\end{figure}

Außerdem können die vom \ac{LLM} generierten Begründungen zu jeder als kritisch erkannten Aktivität im Editor eingesehen werden. Diese Erläuterungen werden gesammelt in einer aufklappbaren Karte im unteren Bereich des Editors angezeigt, siehe Abbildung \ref{fig:sandbox-frontend-ai-reasoning}.

\begin{figure}
    \centering
    \includegraphics[width=\linewidth]{images/sandbox/sandbox-ai-reasoning}
    \caption{Begründung der Klassifikation durch das LLM in der Sandbox.}
    \label{fig:sandbox-frontend-ai-reasoning}
\end{figure}

Um verschiedene \acp{LLM} vergleichen zu können, verfügt die Sandbox auf der rechten Seite über ein Einstellungsmenü mit konfigurierbaren \ac{LLM}-Parametern (siehe Abbildung \ref{fig:sandbox-frontend-analyzed-model}). Diese Parameter sind identisch zu den in Kapitel \ref{sec:api-design} beschriebenen \texttt{llmProps} und werden beim Absenden der Analyse in die API-Anfrage überführt.

