\section{Ziel und Annahmen}\label{sec:ziel-und-annahmen}

\begin{itemize}
    \item Eingabe ist BPMN-XML, Ausgabe sind die IDs der kritischen Aktivitäten mit ggf. Erklärung
    \item Getestet wird mit BPMN-Modellen aus Camunda und bpmn.io
    \item Der Algorithmus klassifiziert Aktivitäten in BPMN-Prozessen binär nach ``kritisch'' oder ``nicht kritisch''
    \item Ausgegeben werden ausschließlich die IDs der kritischen Aktivitäten ggf. mit Erklärung, warum sie als kritisch klassifiziert wurde
    \item Das Ziel ist ein robustes und reproduzierbares Verhalten über unterschiedliche Modelle
    \begin{itemize}
        \item Granularität der Prozesse (Bspw. $>$5 Aktivitäten, 20+ Aktivitäten)
        \item Gateways, Datenobjekte, Datenbanken
        \item Mehrere Pools/Lanes, Message Flows
        \item Evtl. (?) verschiedene Sprachen, Text-Annotationen
    \end{itemize}
\end{itemize}